\chapter{STOPPING CRITERIA}
% !TEX root = hazy1.tex
\label{sec:StoppingCriteria}

\section{Overview}

\Cloudy\ will stop at some depth into the cloud.
The physics that sets
this depth is important since it can directly affect predicted quantities.
This chapter describes various stopping criteria.

Two geometries, matter bounded and radiation bounded, can be identified
for an ionized gas.

A radiation-bounded cloud is one where the outer edge of the emitting
gas is a hydrogen ionization front.
In this case the calculation stops
because nearly all ionizing radiation has been attenuated and the temperature
falls below \TEMPSTOPDEFAULT, the default lowest-allowed kinetic temperature.
This
choice of lowest temperature was made with optical emission lines in mind.
Setting another outer limit is not necessary unless molecules or lines with
very low ionization and excitation potentials (i.e., the [C II] or [O I]
far infrared lines) are of interest.
You lower
the stopping temperature with the \cdCommand{stop temperature} command.

In a matter-bounded cloud the gas is optically thin to energetic radiation
and the outer radius of the cloud must be specified.
This could be a column
density, physical thickness in cm, or optical depth.
More than one stopping
criterion can be specified and the calculation will stop when the first
one is met.

\Cloudy\ will say why it stopped after the results of the last zone
calculation are printed.
It is very important to make sure that the
calculation stopped for the reason you intended.

If no stopping criteria are set the calculation will usually stop because
the default lowest temperature (\TEMPSTOPDEFAULT) or the default greatest number of
zones (1400) was reached.

\section{Danger!  Understand why the calculation stopped!}

Sometimes the predicted emission-line spectrum will depend strongly on
the thickness of the cloud.
The cloud thickness is set by the stopping
criteria.
The predicted intensity will depend on thickness if the outer
edge of the cloud is within a line's creation region.
This is often the
case for some lines in an X-ray irradiated gas and for any radiation field
and molecular or low-ionization infrared lines.

There are several checks that should be made to confirm that the spectrum
is the one expected and not an artifact of the cloud thickness or stopping
criteria.
The first and most important is to understand \emph{why} the calculation
stopped.
This is explained in the first comment after the last zone is
printed.
First locate the print out for the last zone.
The following
example, from the \cdFilename{pn\_paris} simulation, shows the printout that includes
the last zone's results and the start of the calculation's summary.
The
first line after the ionization distribution of iron gives the title for
the model and the line after that gives the reason that the calculation
stopped.
In this case the calculation stopped because the kinetic
temperature fell below the lowest temperature, which was left at its
default of \TEMPSTOPDEFAULT.
This occurs very near the hydrogen \hplus\ - \hO\ ionization front.
Optical and UV lines form at higher temperatures so the calculation would
include all contributors to the optical/UV spectrum.
Lines such as [C II]
158$\mu$ or [Si~II] 34.8$\mu$ form in cool neutral gas,
so these lines would become
stronger if the calculation went more deeply, into cold neutral gas.

{\setverbatimfontsize{\tiny}
\begin{verbatim}
####150  Te:3.978E+03 Hden:3.000E+03 Ne:1.276E+02 R:4.062E+17 R-R0:3.062E+17 dR:5.658E+13 NTR:  5 Htot:4.094E-19 T912: 9.97e+07###
 Hydrogen      9.70e-01 2.97e-02 H+o/Hden 1.00e+00 3.57e-09 H-    H2 1.57e-07 5.07e-10 H2+ HeH+ 8.59e-08 Ho+ ColD 3.27e+19 8.86e+20
 Helium        8.77e-01 1.23e-01 1.81e-04 He I2SP3 4.78e-08 5.43e-16 Comp H,C 1.39e-26 3.59e-27 Fill Fac 1.00e+00 Gam1/tot 6.57e-01
 Carbon        2.50e-04 9.97e-01 3.12e-03 0.00e+00 0.00e+00 0.00e+00 0.00e+00 H2O+/O   0.00e+00 OH+/Otot 0.00e+00 Hex(tot) 0.00e+00
 Sulphur    0  7.02e-05 9.64e-01 3.56e-02 2.36e-06 0.00e+00 0.00e+00 0.00e+00 0.00e+00 0.00e+00 0.00e+00 0.00e+00 0.00e+00 0.00e+00
 Argon      0  7.82e-01 1.96e-01 2.24e-02 0.00e+00 0.00e+00 0.00e+00 0.00e+00 0.00e+00 0.00e+00 0.00e+00 0.00e+00 0.00e+00 0.00e+00
 Iron       0  9.25e-06 9.97e-01 2.80e-03 3.42e-06 0.00e+00 0.00e+00 0.00e+00 0.00e+00 0.00e+00 0.00e+00 0.00e+00 0.00e+00 0.00e+00
    parispn.in Meudon Planetary nebula
   Calculation stopped because lowest Te reached.    Iteration  1 of  1
   The geometry is spherical.
  !Some input lines contained [ or ], these were changed to spaces.
  !Non-collisional excitation of [OIII] 4363 reached   2.35\% of the total.
  !AGE: Cloud age was not set.  Longest timescale was 3.68e+012 s = 1.17e+005 years.
   Suprathermal collisional ionization of H reached 9.48\% of the local H ionization rate.
   Charge transfer ionization of H reached 8.94\% of the local H ionization rate.
\end{verbatim}
}

Left on its own, the code will probably stop when the temperature falls
below the default lowest temperature of \TEMPSTOPDEFAULT.  This is what happened in
the preceding example.  This temperature was chosen for two reasons; a)
collisionally excited optical and ultraviolet lines generally form in gas
hotter than this (but infrared lines will form at far lower temperatures)
and b) more than one thermal solution is possible for temperatures around
3000~K (\citealp{Williams1967}), so thermal instabilities may result if the gas
extends to cooler temperatures.

It is a good idea to check whether the predictions would change if the
model were made thicker or thinner.  It is safe to assume that a line's
intensity does not depend on the thickness of the cloud if the final
temperature is well below the excitation potential of the line or the gas
is more neutral than the species of interest.

The chapter \cdSectionTitle{Output} in Part 2 of this document
lists all possible reasons for stopping.

\section{Radius inner =18 [thickness =16; parsec; linear]}

The optional
second number on the \cdCommand{radius} command
sets the thickness or outer radius of the cloud.

\section{Stop Av 12.1 [point, extended]}
\label{sec:CommandStopAv}

This stops a calculation at a specified visual extinction $A_V$.
The value
is the extinction in magnitudes at the isophotal wavelength of the $V$ filter
(5500\AA ).
The number is the linear extinction in magnitudes unless it
is negative, when it is interpreted as the log of the extinction.
Note
that there must be spaces before and after the key ``\cdCommand{AV}.''

Properties of grains are determined by the \cdCommand{grains} command.
The distinction between the extinction for a point
versus an extended source is described in
section 7.6 of AGN3.
By default the $A_V$ specified with this command will
be for a point source,
which is the quantity measured in extinction studies
of stars.
The extended-source extinction, appropriate for extinction
of a spatially-resolved emission-line region,,
is specified with the keyword \cdCommand{extended}.

\section{Stop column density = 23 [neutral; ionized; total; \dots]}

This stops the calculation at the specified hydrogen column density
$N$(H) [cm$^{-2}$].
There are several optional keywords which determine whether the
column density is the total (the default), the ionized hydrogen column
density, the neutral hydrogen column density, or several other column
densities.
The default stopping column density is $10^{30}\ \mathrm{cm}^{-2}$.

By default the number on the line is interpreted as the
log of the column density.
The keyword \cdCommand{linear} forces that interpretation.

\subsection{Stop column density 23  }

The number is the log of the total hydrogen column density (atomic, ionic,
and all molecular forms), defined as the integral
\begin{equation}
N\left( {\mathrm{H}} \right) = \int {\left\{ {n\left( {{\mathrm{H}}^0 } \right)
+ n\left( {{\mathrm{H}}^ +  } \right) + 2n\left( {{\mathrm{H}}_2 } \right) +
\sum\limits_{other} {n\left( {{\mathrm{H}}_{other} } \right)} } \right\}}
\,f\left( r \right)\,dr
\,[\mathrm{cm}^{-2}]% (69)
\end{equation}
where $f(r)$ is the filling factor.

\subsection{Stop neutral column density 23  }

The number is the log of the atomic hydrogen column density
\begin{equation}
N\left( {{\mathrm{H}}^0 } \right) = \int {n\left( {{\mathrm{H}}^0 } \right)f\left(
r \right)\,dr}
\, [\mathrm{cm}^{-2}].% (70)
\end{equation}

\subsection{Stop ionized column density 23  }

The number is the log of the ionized hydrogen column density
\begin{equation}
N\left( {{\mathrm{H}}^ +  } \right) = \int {n\left( {{\mathrm{H}}^ +  }
\right)f\left( r \right)\,dr}
\,[\mathrm{cm}^{-2}] .% (71)
\end{equation}

\subsection{Stop atomic column density 21.3}

In some PDR literature the atomic hydrogen column density is defined
as the sum of \hO\ and \htwo.
This command allows the calculation to stop as
this summed column density, defined~as
\begin{equation}
N\left( {{\mathrm{H}}^0  + 2{\mathrm{H}}_2 } \right) = \int {\left[ {n\left(
{{\mathrm{H}}^0 } \right) + 2n\left( {{\mathrm{H}}_2 } \right)} \right]f\left( r
\right)\,dr}
\,[\mathrm{cm}^{-2}] .% (72)
\end{equation}
This command was added by Nick Abel.
Note that this counts each \htwo\ as two
hydrogen atoms.

\subsection{Stop H/TSpin column density 17.2}

This calculation stops at the specified integral
\begin{equation}
N\left( {{\mathrm{H}}^0 } \right)/T_{spin}  = \int {\frac{{n\left( {{\mathrm{H}}^0
} \right)}}{{T_{spin} }}f\left( r \right)\,dr}
\,[\mathrm{K^{-1}\, cm}^{-2}] .
\end{equation}
Note that $T_{spin}$ is computed self-consistently
including the effects of
pumping by the background continuum (usually the CMB)
and the local \la\ radiation field.

\subsection{Stop H2 column density 19.2}

The calculation stops at the specified molecular hydrogen column density
\begin{equation}
N\left( {{\mathrm{H}}_2 } \right) = \int {n\left( {{\mathrm{H}}_2 } \right)f\left(
r \right)\,dr}
\,[\mathrm{cm}^{-2}] .% (74)
\end{equation}
The 2 in \cdCommand{H2} must come before the log of the
column density in the command.
This is really the \htwo\ column density, not twice it.

\subsection{Stop CO column density 19.2}

The calculation stops at the specified column density in CO
\begin{equation}
N\left( {{\mathrm{CO}}} \right) = \int {n\left( {{\mathrm{CO}}} \right)f\left( r
\right)\,dr}
\,[\mathrm{cm}^{-2}] .% (75)
\end{equation}
This command was added by Nick Abel.

\subsection{Stop column density ``OH'' 19.2}
The calculation stops at the specified column density in the species given in quotes.  If the species is not recognized, the command will have no effect.  
Species are described on page \pageref{sec:SpeciesDefine}.

\subsection{Stop effective column density 23  }

This is actually a form of the \cdCommand{stop optical depth} command.
Usually, low-energy cutoffs in X-ray spectra are
parameterized by the equivalent column density of a cold neutral absorber
with cosmic abundances.
Actually what is measured is an optical depth at
some energy, generally around 1.0 keV.
If the gas is ionized then a much
larger column density will be needed to produce the observed absorption.
The difference can be more than an order of magnitude.
This command stops
the calculation when the incident continuum has been attenuated by the
appropriate absorption at 1.0 keV.
The calculation stops when the absorption
optical depth at 1.0 keV (neglecting scattering opacities)
reaches a value of
\begin{equation}
\tau _{abs} \left( {1.0\;{\mathrm{keV}}} \right) = N_{effec} 2.14 \times 10^{
- 22}
\, [\mathrm{Napier}]
\end{equation}
at 73.5 Ryd.
The argument of this command is the log of the effective column
density $N_{effec}$.
The absorption cross-section per proton for cold neutral
gas is taken from \citet{Morrison1983}.
Scattering opacities \emph{are not} included in this optical depth.
No attempt is made to use realistic
physical conditions or absorption cross sections---this command follows
the Morrison \& McCammon paper.

If the gas is highly ionized then the actual column density will be
greater than the effective column density.
It will be less if the abundances
of the heavy elements are greatly enhanced.

\section{Stop continuum flux 200 micron 65 Jansky}

This commands allows you to stop the calculation when an observed continuum
flux or surface brightness at an arbitrary wavelength is reached. Note that
both the keywords \cdCommand{continuum} and \cdCommand{flux} need to be
present on the command line. The first parameter must be the wavelength or
frequency of the observation. The second must be the flux or surface
brightness. The syntax for these parameters is exactly the same as for the
\cdCommand{optimize continuum flux} command described in
Section~\ref{sec:opt:cont:flux}. If the flux is $\leq$ 0, or the keyword
\cdCommand{log} appears on the line, it will be assumed that the flux is
logarithmic (but not the wavelength or frequency!).

The code will implicitly behave as if a \cdCommand{set nFnu add} command had
been given to add the requested frequency point.
Observed fluxes must be dereddened for any extinction inbetween the
source and the observer. The same normalization will be used as in the rest of
the Cloudy output. So if you include the \cdCommand{print line flux at earth}
command, you should enter the observed flux at Earth here. See
Section~\ref{sec:line:flux:earth} for further details. See also
Section~\ref{sec:set:nfnu} for a discussion of the components that are
included in the continuum flux prediction.

There is no limit to the number of \cdCommand{stop continuum flux}
commands. The following gives some examples of its use:
\begin{verbatim}
# make sure we normalize the flux as observed
print line flux seen at earth
# and set the distance
distance 980 linear parsec

# stop when a dust continuum flux, e.g. measured by IRAS is reached
stop continuum flux 60 micron 220 Jy

# stop when a measured radio continuum flux is reached
stop continuum flux 6 cm 50 mJy
# same for an OCRA measurement...
stop continuum flux 30 GHz 0.1 jansky

# this will stop on a flux of 1e-21 erg s^-1 cm^-2 Hz^-1 (100 Jy) at 100 micron
stop continuum flux 100 micron -21 erg/s/sqcm/Hz
\end{verbatim}

\section{Stop depth \dots}

This is another name for the \cdCommand{stop thickness} command
described on page \pageref{sec:CommandStopThckness} below.

\section{Stop eden 3 [linear]}

The calculation stops when the electron density falls below the
indicated value.
The number is the log of the electron density [cm$^{-3}$].
The optional
keyword \cdCommand{linear} will force the argument to be interpreted as the quantity
itself, not its log.
This command is one way to stop constant-temperature
models.
For instance, the calculation can be forced to stop at the
\hplus\ - \hO\
ionization front by setting the stopping electron density to approximately
half of the hydrogen density.

The following examples show a case that will stop
near the
He$^{2+}$-He$^+$
ionization front and a case that will stop near
the \hplus-\hO\ ionization front
for solar abundances.
\begin{verbatim}
#
# stop at the He++ - He+ ionization front
hden 9
stop eden 9.06 # stop when helium (10\% by number) is He+

#
# stop at H+ - H0 ionization front
hden 5
stop eden 4.5 # stop when electron dens falls below H density
\end{verbatim}
The default is an electron density of $-10^{30}$~cm$^{-3}$.
(The negative sign
is not a typo.)

\section{Stop efrac = 1.05}

The model will stop when the electron fraction, defined as the ratio
of electron to total hydrogen densities, falls below the indicated value.
This is another way to stop calculations at ionization fronts.  This is
useful if the hydrogen density there is not known beforehand as occurs in
constant-pressure calculations.  The argument is the fraction itself if
it is greater than zero and the log of the fraction if it is $\le 0$.

The default is an electron fraction of $-10^{37}$~cm$^{-3}$.
(The negative sign is not a typo.)

\section{Stop line "C  2" 157.6m reaches 0.2 rel to  "O  3" 5006.84}

The calculation will stop when the emission line with the label given
within the first pair of quotes and the wavelength given by
the first number
exceeds an intensity given by the second number, relative to an optional
second emission line.
In this example the calculation will stop when the
intensity of [C II] 158 $\mu $m reaches 0.2 relative to [O III] $\lambda$5006.84.
If a second
optional line is not entered it will be H$\beta^{30}$.\footnote{The scaling of the line intensities can be changed with the
\cdCommand{normalize}
command.
That command can change both the
normalization line (usually H$\beta$) and its relative intensity
(usually 1).
If the second line is not set with the \cdCommand{stop line} command
then
H$\beta$ is the
denominator in the ratio.
The \cdCommand{stop line} command always uses the ratio of
the two line intensities on the scale that is set with the
\cdCommand{normalize} command.
In versions C07.02 and before the \cdCommand{normalize} command
did not interact with
the stop line command.}
This can be a useful way
to stop matter-bounded models.
The results of this command are not exact;
the final intensity ratio will be slightly larger than the ratio specified.

Intrinsic intensities are used by default.
Emergent intensities will be used if the keyword
\cdCommand{emergent} appears on the command line.

The line label and wavelength should be entered exactly as it appears
in the emission-line output.
The line label within the quotes must have
the same number of spaces as in the output.
``C~2'' and ``C~~2'' are not the same (the first has only one space
and the second is correct with two spaces).
The wavelength is assumed to be in Angstroms if no letter
follows it.
The wavelength can be changed to microns or centimeters by immediately
following the wavelength with 'm' or 'c' so that
the [\cii ] $157.6\micron$ would be written as ``157.6m''.

Up to 10 different \cdCommand{stop line} commands may be entered.
If more than one
\cdCommand{stop line} command is entered then the code
will stop when any limit is reached.

\section{Stop mass 32.98}

The calculation will stop when the total mass of the computed structure
exceeds the quantity entered.  If the inner radius is specified (the
luminosity case) then the entered number is the log of the mass in gm.
If the inner radius is not specified (the intensity case) then it is the
log of the mass per unit area, gm cm$^{-2}$.

At the current time no attempt is made to make the computed mass exactly
equal to the entered number.
The calculation will stop after the zone where
the mass is first exceeded.

\section{Stop mfrac = 0.5}

The calculation stops when the hydrogen molecular fraction, defined as
$2n(\mathrm{H}2)/n(\mathrm{H}_{tot}$), exceeds the indicated value.
This is a way to stop
calculations within a PDR.
The argument is interpreted as the molecular
fraction itself if it is greater than zero and as the log of the fraction
if it is less than or equal to zero.

The default is a molecular fraction of $10^{37} \mathrm{cm}^{-3}$.

\section{Stop molecule depletion -2}

Nick Abel incorporated the condensation of molecules onto grain surfaces.
Currently CO, \water, and OH condensation are treated.
The density of molecules
on grains is given as a species with the molecule's usual label, ``CO'',
``H2O'', or ``OH'', followed by the string ``gr''.
The rates of UV and
cosmic ray desorption along with accretion come from \citet{Hasegawa1992} and \citet{Hasegawa1993}.
The grain surface chemistry
mentioned in these papers is not included so our treatment of condensation
is closer to the work of \citet{Bergin1995}, who also use
the Hasegawa et al. rates.
Use the \cdCommand{no grain molecules} command
to turn off this condensation.

\section{Stop nTotalIoniz 100}

This is a way to stop a calculation while debugging and is not normally
used.
The variable with this name is incremented each time the ionization
evaluation routine is called.
This command will cause the calculation to
abort after the specified number of calls.

\section{Stop optical depth -1 at 2.3 Ryd}

This command stops the calculation at an arbitrary continuum absorption
optical depth.
The first number is the log of the optical depth.
The optical depth is interpreted as a log by default.
If the \cdCommand{linear} keyword
occurs then the number is interpreted as the linear value.
The optical
depth is only for absorption and does not include scattering opacities.
The second number is the energy in Rydbergs.
It is interpreted as a log
if it is negative, as linear if positive, and must be within the energy
bounds considered by the code (presently \emm\ to
\egamry).
At present only one stopping optical depth can be specified.
If more than
one is entered then only the last is honored.

It is traditional in X-ray astronomy to characterize low-energy cut-offs
as the equivalent \emph{completely neutral} column density
for \emph{solar} abundances.
This is not correct when the gas is ionized
(since the high-energy absorption
opacity is diminished) or when the abundances of the heavy elements are
enhanced (since the high-energy opacity is increased).
For extreme cases
these effects can change the opacity by more than an order of magnitude.
The deduced column density is underestimated by the same amount.
It is
better to convert the deduced column density back into an optical depth
at 0.5 or 1 keV (this is actually the observed quantity)
and use this optical
depth and energy as the stopping criteria than to use the deduced column
density as a stopping criterion.
Either this command, or the \cdCommand{stop effective
column density} command (which is actually a form of the
\cdCommand{stop optical depth} command can be used to stop
the calculation
at an X-ray optical depth corresponding to a certain low-energy absorption.

The optical depth used in this command is the absorption optical depth
and does not include scattering opacities.
In general, the effects of
scattering opacities are much more geometry dependent than absorption
opacities.

\subsection{Stop Balmer optical depth $= -.$3}

This command is a special case of the \cdCommand{stop optical depth}
command in which the energy does not need to be specified
but the keyword \cdCommand{Balmer} is given.
It will stop the calculation when the log of the absorption
optical depth
at the Balmer continuum threshold ($\nu = 0.250$ Ryd)
reaches the specified value.
The default is $\tau_{Bac} = 10^{20}$ and the optical depth
is always interpreted as a log.
This is the \emph{total absorption} optical depth at the Balmer edge
and includes all computed absorption opacity sources such as grains or
free-free absorption, but neglects scattering.

\subsection{Stop Lyman optical depth = 5}

This is a special case of the \cdCommand{stop optical depth} command
in which the
energy does not need to be specified but the keyword \cdCommand{Lyman} is given.
The
number is the log of the Lyman limit optical depth, $\tau_{912}$.
The default value
is $\tau_{912} = 10^{20}$.
The stopping criterion is \emph{really} the
\emph{total} $912$\AA\ \emph{absorption}
optical depth and \emph{not} the hydrogen Lyman limit optical depth at $912$\AA.
These are not exactly the same, especially when grains
are present or the
abundances of the heavy elements are enhanced.

\subsection{Stop 21cm optical depth = linear 3}

This stops the calculation when the \emph{line center} optical depth of \hi\ $\lambda$21~cm reaches the entered value.
The populations of the hyperfine levels
are determined by solving the full level populations with the effects of
pumping by \la\ and the external continuum included.
The calculation stops
when the optical depth from the continuum source to the outer edge of the
cloud exceeds the entered value.
The actual optical depth with be greater
than or equal to the indicated value.

Note that \Cloudy\ always uses the dimensionless \emph{line center}
optical depth.
The \emph{integrated} optical depth is the standard used
in radio astronomy.
The
\emph{integrated} optical depth is called the \emph{mean} optical depth in optical
spectroscopy.
The dimensionless mean optical depth is $\sqrt \pi  $
larger than the line center optical depth.
In radio astronomy, the
integrated optical depth is often given in velocity units, km~s$^{-1}$.

\section{Stop pfrac = 0.23}

The calculation will stop when the proton fraction, defined as the ratio
of proton (ionized hydrogen) to total hydrogen densities, falls below the
indicated value.
This is another way to stop calculations at ionization
fronts and is useful if the hydrogen density there is not known beforehand.
This occurs in constant pressure calculations, for instance.
The argument
is interpreted as the fraction itself if it is greater than zero
and the
log of the fraction if it is less than or equal to zero.

The default is an proton fraction of $-10^{37}$~cm$^{-3}$.
(The negative sign is not a typo.)

\section{Stop radius 17.3 [parsec; linear; 17.6 on sec iter] }

This sets an upper limit to the radius of the cloud.
The argument
is interpreted as the log of the radius unless the keyword
\cdCommand{linear} appears.
The default unit is centimeter but it will be interpreted
as the log
of the radius in parsec if the keyword \cdCommand{parsec}
appears on the line.

The \cdCommand{stop radius} command has the same effect
as the optional second
number on the \cdCommand{radius} command.

Any number of radii may be entered on the command line.
Each will
be the ending radius for consecutive iterations.
The limit to the number
of stopping values is set by the limit to the number of iterations that
can be performed.
If fewer numbers are entered than iterations performed
then the last number will be used for all further iterations.

This command is useful if you want to vary the inner radius in an optimizer
run, but want to keep the outer radius fixed. If you do so, make sure that you
set the upper limit of the inner radius to a value less than the outer radius.
You can do this with the \cdCommand{optimize range} command. If you did not
set an upper limit and the optimizer would try to move the inner radius beyond
the outer radius, the code would abort.

\section{Stop temperature 1e3 K [linear, exceeds]}
\label{sec:CommandStopTemperature}

The calculation will stop when the kinetic temperature drops below
$T_{low}$, the argument of this command.
The argument is interpreted as the log of
the temperature if $\le 10$ and as the linear temperature
if $> 10$ or if the \cdCommand{linear} keyword appears.

The default value is $T_{low} =$~\TEMPSTOPDEFAULT.
Gas cooler than this produces little
optical emission, but may be a strong emitter of infrared lines such as
the [C II] 158 $\mu$m or the [O I]$^3P$ lines.
The lowest temperature allowed,
$T_{low}$, should be adjusted with this command
so that the excitation potential
$h\nu$ is $\cong kT_{low}$ for the lowest excitation potential
transition considered.
Note that more than one temperature is sometimes possible
when $T \approx 10^3$~K
(\citealp{Williams1967}) so thermal stability problems may develop
if $T_{low}$ is lowered
below a few thousand degrees Kelvin.
This issue is discussed further in
the section \cdSectionTitle{Problems} in Part 2 of this document.

If the keyword \cdCommand{exceeds} appears on the line
then the specified temperature
will be the highest allowed temperature.
The calculation stops if the
temperature exceeds the value on the command.
This might be necessary when
a grid of models is computed but those in the high
temperature phase (i.e.,
$T > 10^5$~K) are not of interest.
The other rules for the command are
unchanged.

If no number appears on the command line,
but the keyword \cdCommand{off} does,
then temperature will not be used as a stopping criterion.

\section{Stop thickness 9.3 [parsec; linear; 23 on sec iter] }
\label{sec:CommandStopThckness}

This sets an upper limit to the thickness of the cloud.
The thickness is the distance [cm] between the illuminated
and shielded faces.
The argument
is interpreted as the log of the thickness unless the keyword
\cdCommand{linear} appears.
The default units are centimeters but it will be interpreted
as the log
of the thickness in parsec if the keyword \cdCommand{parsec}
appears on the line.

The \cdCommand{stop thickness} command has the same effect
as the optional second
number on the \cdCommand{radius} command.
This command makes
it possible to set a cloud thickness when the inner radius
is not specified,
such as when the ionization parameter is given.

Any number of thicknesses may be entered on the command line.
Each will
be the ending thickness for consecutive iterations.
The limit to the number
of stopping values is set by the limit to the number of iterations that
can be performed.
If fewer numbers are entered than iterations performed
then the last number will be used for all further iterations.

The keyword \cdCommand{depth} can be used instead of \cdCommand{thickness}.

\section{Stop velocity $<$ 1 km/s}

The calculation will stop if the absolute value of the wind velocity
falls below the value.
The stopping velocity is entered in km s$^{-1}$ but is
converted to cm s$^{-1}$ within the code.

\section{Stop zone 123 [21 on sec iteration,\dots]}

This sets a limit to the number of zones that are computed on each
iteration.
It is not normally used.
In this example the calculation will
stop after computing 123 zones.
The default value is 1400.
Any number
of zones may be entered, each being the ending zone for consecutive
iterations.
This limit is set by the limit to the number of iterations
that can be performed.
If fewer numbers are entered than iterations
performed then the last number will be used for all further iterations.

The code checks that it did not stop because the default number of zones
was reached.
A warning will be generated if this happens since it was
probably not intended.
Use the \cdCommand{set nend} command to increase the default number
of zones while keeping this checking active.

The code allocates memory to store a great deal of information for the
limiting number of zones.
Increasing the number of zones will also increase
the memory needed to run the code.
Don't do this unless you really need
to use the zones.


