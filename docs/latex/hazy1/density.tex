\chapter{DENSITY LAWS}
% !TEX root = hazy1.tex

\section{Overview}

Hydrogen plays a fundamental role in any astrophysical plasma because
of its large abundance.
As a result the hydrogen density [cm$^{-3}$] is a
fundamental parameter.
Commands that specify how the hydrogen density is
set, and how it changes with radius or depth, are described in this section.
Constant density is the default.  In this case the total hydrogen density
(the sum of the protons in atomic, ionic, and molecular form,
given by the
command \cdCommand{hden}) is kept constant.
Many other
density or pressure distributions can also be computed.

A cloud can be isobaric, maintain constant pressure, if the timescale
for changes, for instance in the continuum source or the cooling time, is
short compared with the dynamical or sound-crossing time $t_d$
\begin{equation}
t_d  = \frac{{\Delta r}}{{c_s }}
\ [\mathrm{s}]
\end{equation}
where $\Delta r$ is the cloud thickness and $c_s$ is the sound speed
(AGN3 eq 6.25)
\begin{equation}
c_s  = \left( {\frac{{\gamma kT_0 }}{{\mu _0 m_{\mathrm{H}} }}}
\right)^{\frac{1}{2}}
 [\cmps ].
\end{equation}

\section{Constant density, pressure, gas pressure}

This command specifies how the density changes across the cloud.
The
\cdCommand{hden} command usually specifies the initial hydrogen density.
Several commands determine how the density changes with radius.
These are described next.

\subsection{Constant density}

This is the default.  The hydrogen density, the sum
\begin{equation}
n\left( {\mathrm{H}} \right) = n\left( {{\mathrm{H}}^{\mathrm{0}} } \right) + n\left(
{{\mathrm{H}}^ +  } \right) + 2n\left( {{\mathrm{H}}_2 } \right) +
\sum\limits_{other} {n\left( {{\mathrm{H}}_{other}^{} } \right)}
\,[\mathrm{cm}^{-3}]% (28)
\end{equation}
is kept constant.
This is not quite an isochoric density law because the
total particle density is not constant---the electron and molecular
fractions can vary with depth.
I prefer this type of model because the
homology relations with the ionization parameter (\citealp{Davidson1977}) are
preserved.
The hydrogen nucleon density is set with the \cdCommand{hden} command.

\subsection{Constant gas pressure [index $=-$1.1]}

An isobaric density law is specified with this command.
The gas pressure
\begin{equation}
P_{gas} = n_{tot}kT_e \,  [\mathrm{dyne \,cm}^{-2}]
\end{equation}
where $n_{tot}$ is the total particle density [cm$^{-3}$],
is kept constant.
The optional index $\alpha$ can be any value, positive or negative,
and will force
the pressure to change as a power-law of the radius;
\begin{equation}
P_{gas}(r)=P_{\mathrm{o}}\left(\frac{r}{r_{\mathrm{o}}}\right)^{\alpha}
[\mathrm{dyne\, cm}^{-2}]
\end{equation}
where $P_{\mathrm{o}}$ is the pressure at the illuminated face of the cloud.

\subsection{Constant pressure [no continuum, no abort]}

This holds the total pressure constant.
This includes ram, magnetic,
turbulent, particle, and radiation pressure.
The equation of state,
the relation between gas density, pressure, and temperature,
is given by
\begin{equation}
\begin{array}{ccl}
 P_{tot} \left( r \right)& =& P_{tot} \left( {r_o } \right)
 + \int {a_{rad}\,\rho \,dr}  + \int {g\,\rho \,dr}  \\
&=& P_{gas} + P_{ram} + P_{turb} + P_{mag} + P_{lines} + \Delta P_{rad} + \Delta P_{grav} \\
 \end{array}\,
  [\mathrm{dyne\, cm}^{-2}]% (31)
\end{equation}
where $a_{rad}$ is the radiative acceleration [cm s$^{-2}$] due to the attenuated
incident continuum, $g$ is the (negative) gravitational acceleration
[cm s$^{-2}$], and $\rho$ is the gas density [gm cm$^{-3}$].

$P_{gas}  = n_{tot} kT_e$ is the thermal gas pressure.
The ram pressure is $P_{ram} = \rho u_{wind}^2$, where $u_{wind}$ is the flow velocity.
This is only present for a wind geometry.
The turbulent pressure is $ P_{turb}  = \rho \,u_{turb}^2 /2$, where $u_{turb}$ is the turbulent velocity set with the \cdCommand{turbulence} command.
The magnetic pressure is $P_{mag}  = B^2 /8\pi $, where the magnetic field B is set with the \cdCommand{magnetic field} command.
Both magnetic and turbulent pressures are zero by default.
If either dominates the total pressure then the density will be nearly constant.

$P_{lines}$ is the nearly isotropic radiation pressure due to
trapped emission
lines (\citealp{FerlandElitzur1984}, and \citealp{Elitzur1986}).
\Cloudy\ will
stop if the internal line radiation pressure builds up to more than half
of the total pressure since such clouds would be unstable unless they are
self-gravitating (\citealp{Elitzur1986}).
It is necessary to do at least
a second iteration when radiation pressure is important since the total
line optical depths must be known to compute line widths and level
populations reliably.
If more than one iteration is done then the radiation
pressure will not be allowed to exceed the gas pressure on any except the
last iteration.
If the option \cdCommand{no abort} appears on the command line the
code will never stop because of excessive radiation pressure.
The radiation
pressure is still computed and the simulation will probably become
unstable when $P_{lines} > P_{gas}$.

When the outward force due to the attenuation of the incident
radiation field and the inward force of gravity are balanced by
the pressure gradient, the cloud will be in hydrostatic equilibrium
\citealp{AscasibarDiaz2009}).
The required change to the total pressure is given by the integrals
and referred to as $\Delta P_{rad}$ and $\Delta P_{grav}$, respectively.
The \cdCommand{no continuum} option will turn off the $\Delta P_{rad}$ term.
Gravity must be turned on separately using the \cdCommand{gravity} command
described below.

\subsection{Constant pressure set 6.5}

The \cdCommand{set} option specifies the initial pressure.
A number, the log of
the pressure in ISM $nT$ units (K cm$^{-3}$), must appear.

An initial gas density must still be specified with one of the commands
in this Chapter.  This command works by first solving for the equilibrium
temperature at the density that was specified.
The density is then changed,
using a Newton-method solver, to have the pressure match that specified.
The solver will raise or lower the density assuming that pressure and
density are directly related to one another.

There are possible ambiguities in this solution since the same pressure
can sometimes be achieved with more than one density.
This is the well-known
three-phase stability phenomenon \citep{Field1969}.

\subsection{Constant pressure timescale = 2.3e9 s, alpha=-1}
\label{sec:ConstantPressureTimescale}

The pressure changes as a function of time.
It is given by
\begin{equation}
P_{\mathrm{o}}(t)=P_{\mathrm{o}}(t_{\mathrm{o}})\;\left(1+\frac{t}{t_{\mathrm{scale}}}\right)^{\alpha}
[\mathrm{dyne\ cm}^{-2}]
\end{equation}
The timescale $t_{\mathrm{scale}}$ is the first number on the line
and the power-law index $\alpha$ is the second.
Here $P_{\mathrm{o}}(t_{\mathrm{o}})$ is the static pressure evaluated at 
the illuminated face of the cloud.

\subsection{The reset option}

This changes the behavior of the \cdCommand{constant pressure} command.
By default the gas pressure in the first zone is derived from
the specified hydrogen density and the resulting kinetic temperature.
The hydrogen density is the fundamental quantity.
In later iterations the initial hydrogen density 
is kept constant but the kinetic temperature may change
as details of the line radiative transfer change. 
The resulting gas pressure, $\propto nT$,
may not have exactly the same value on successive iterations.
The total pressure in constant pressure clouds will change 
slightly as a result.
The change in the pressure is small, typically only
a few percent. 

The \cdCommand{reset} option tells the code to keep the total pressure
at the illuminated face constant, from iteration to iteration,
rather than the hydrogen density.
The \cdCommand{reset} option works by allowing the gas density
to change to try to match the total pressure 
found in previous iterations.

\section{Gravity [options] }

The \cdCommand{gravity} command tells the code to balance the gravitational 
acceleration by varying the total pressure as described in 
\citealp{AscasibarDiaz2009}.

\subsection{Gravity [spherical, plane-parallel] }

One of the keywords \cdCommand{spherical} or \cdCommand{plane-parallel} 
must appear.  Each of these commands specifies the symmetry of the mass distributions
and activates self-gravitational pressure.

\subsection{Gravity external [Mass = 10 {\Msun}, extent = 3 parsecs, powerlaw = -2]}

Additional mass components can be specified by any number of 
these commands.
The first number of represents the mass.  
The units depend upon the symmetry
(which must be specified separately -- see above).
In the spherical case, this number represents the mass (in solar masses) 
of a pointlike star located at $r=0$; in the plane-parallel case, 
it corresponds to the surface density (in~M$_\odot$~pc$^{-2}$) of a uniform
mass sheet at $z=0$.
The number is interpreted as linear 
unless the keyword \cdCommand{log} is specified.
Two optional parameters are the physical
extent of the mass component (in parsecs, linear unless \cdCommand{log} appears)
and a power-law index.  Together these specify the mass accumulation function
\begin{equation}
M_{\mathrm{enclosed}} = M (r/r_0)^\alpha.
\end{equation}
The last two parameters can be omitted from right to left.  
Point sources or infinitesimally thin sheets are the default.
The default index is zero.

\begin{shaded}
\section{\experimental Dark [options]}
This command provides a gravitational pressure term due to a dark-matter
distribution.  At present, there is only one option, a standard
\citet{NFW96} (NFW) profile.

\subsection{Dark NFW virial radius = 18 [characteristic radius = 17]}
This command specifies an NFW profile uniquely specified by their 
virial and characteristic radii $r_{200}$ and $r_s$, respectively.
The second parameter can be omitted and has the
default value $r_s=r_{200}/10.$ Both parameters are interpreted as 
logarithms of distances in \cm.
\end{shaded}

\section{Dlaw [options]}

An arbitrary density law, specified by the user, will be used.
There are two forms of this command.
It is possible to either edit the source
to create a new routine that calculates the density at an arbitrary depth
or to interpolate on a table of points.

If the density or density law is specified with both this command and
others, such as \cdCommand{hden}, \cdCommand{constant pressure}, etc,
only the last command will be honored.

It is possible to specify a change in densities so extreme that the code
will have convergence problems.
\Cloudy\ works by linearizing all equations.
If the density changes dramatically over a very small radius the
conditions may change too much for the solvers to converge.  The code uses
adaptive logic to adjust the zoning and should prevent this from happening
but the heuristics may be fooled by drastic density laws.  The code will
generate a warning if the density does change by too much---if
this happens
the cure is to not use such large density contrasts.

\subsection{dlaw p1, p2, p3\dots}

This is the default form of the command.  It passes the parameters on
the command line to a user-provided function  \cdTerm{dense\_fabden},
located in \cdFilename{dense\_fabden.cpp}.  There are up to ten
parameters.  A new function \cdTerm{dense\_fabden}  must be written
by the user and the
version of \cdTerm{dense\_fabden} already in \Cloudy\ must be deleted.
(The code will stop
if the initial version of \cdTerm{dense\_fabden} is not replaced.)
\Cloudy\ will call \cdTerm{dense\_fabden}
as needed to determine the density as a function of depth.
The arguments
of the function are the radius and the depth.
Both are in centimeters and are double precision variables.
The function must return the hydrogen density (cm$^{-3}$)
as a double precision variable.
The code provided in the function must use the ten or fewer
parameters in the structure \cdVariable{dense}
to compute the density at the current position.

The following is an example of a function in which the density is the
product of the first number on the command line and the depth.
\begin{verbatim}
/*dense_fabden implements the dlaw command, returns density using
 * current position and up to ten parameters on dlaw command line */
#include "cddefines.h"
#include "rfield.h"
#include "dense.h"

double dense_fabden(double radius, double depth)
{
    return( depth*dense.DensityLaw[0] );
}
\end{verbatim}

\begin{shaded}
\subsection{\experimental Dlaw wind}

This sets up a density profile consistent with a steady-state wind
and the continuity equation. The steady-state velocity profile is 
parametrized as in \citet{Springmann1994}:
\begin{equation}
v(r) = v_\star + (v_{\infty} - v_0) \sqrt{ \beta_1 x + (1-\beta_1) x^{\beta_2} }.
\end{equation}
where $x\equiv 1 - r_\star/r$ and we set $r_\star$ to the inner radius of the
cloud (except that $x$ has a minimum value 0.01 for
computational stability).
The mass loss rate into 4$\pi$ sterradians ($\dot{M}$) then allows
the density via continuity:
\begin{equation}
n(r) = \dot{M} / ( 4\pi m_H \mu r^2 v(r) ),
\end{equation}
where $\mu$ is the mean molecular weight of the gas.
The parameters must be specified in this order:
$\dot{M}$, $v_{\infty}$, $\beta_2$, $\beta_1$, $v_0$, $v_\star$.
Only the first three are required.  
$\dot{M}$ should be given as \Msunpyr.
All velocities should be given as \kmps.
The final three may be omitted right to left and
take default values $\beta_1 = v_0 = v_\star = 0$.
All values are interpreted as linear values.

This command sets the hydrogen density but does not create a wind.  
It creates a density profile that mimics a wind.  
The  \cdCommand{dlaw} command 
(with or without the  \cdCommand{wind} keyword) 
and the  \cdCommand{wind} command described in section 
\ref{sec:CommandWind} are mutually exclusive, 
and you will just end up with whichever one appeared last.  
There is no way to combine the two.

This command is experimental.
\end{shaded}

\subsection{Dlaw table [depth, radius]}

If the keyword \cdCommand{table} appears on the \cdCommand{dlaw}
command then the code will read
in a set of ordered pairs of radii and densities.
The original form of
this option was added by Kevin Volk.
There must be two numbers per line
as in the example below.
The first number is the log of the radius or depth
[cm] and is followed by the log of the hydrogen density [cm$^{-3}$].
If the
keyword \cdCommand{depth} also appears on the command line then
the first number is
interpreted as the log of the depth from the illuminated face
and the table
must begin with a depth smaller than 10$^{-30}$ cm,
the first point where the
depth is evaluated.
The first number is interpreted as the log of the radius
if \cdCommand{depth} does not appear.
The ordered pairs end with a line with the keyword
\cdCommand{end} in columns 1 through~3.

Linear interpolation in log-log space is done.
The following is an example.
\begin{verbatim}
dlaw table depth
continue -35 4
continue 12 4
continue 13 5
continue 14 6
continue 15 7
end of dlaw
\end{verbatim}

Be sure that the first and last radii or depths extend beyond the computed
geometry---this law is only to be used for interpolation and the code will
stop if extrapolation is necessary.
Note that the first depth must be
smaller than $10^{-30}$~cm,
and also that there must not be a space in the first
column of any lines with numbers---the code will think that an end of file
has been reached.
Alphabetic characters can be placed anywhere on the line
and will be ignored---I placed the word
\cdCommand{continue} in the first four columns
for this reason (it is actually totally ignored).

\section{Fluctuations density period \dots}

This specifies a density that varies as a sine wave.  It was introduced
to investigate the effects of inhomogeneities upon the
emission-line spectrum
(see \citealp{Mihalszki1983}; \citealp{KingdonFerland1995}).
The first number
is the log of the period $P$ of the sine wave in centimeters.
The second
two numbers are the logs of the largest and smallest hydrogen densities
over the sine wave.  Order is important here.
The last optional number
is a phase shift $\varphi$ (in radians) which allows the initial zone to occur at
any part of the sine wave.
If it is omitted the calculation will begin
at the maximum density.
If the phase is set to $\pi$ the calculation will start
at the minimum density.

The density is scaled according to the relation
\begin{equation}
\label{eqn:commandFluctuations}
n\left( r \right) = \left( {\frac{{n_{\max }  - n_{\min } }}{2}} \right)
\times \cos \left( {\Delta r\frac{{\,2\pi }}{P} + \varphi } \right) + \left(
{\frac{{n_{\max }  + n_{\min } }}{2}} \right)\, [\mathrm{cm}^{-3}]% (32)
\end{equation}
where $n_{max}$ and $n_{min}$ are the maximum and minimum densities and
$\Delta r$ is the
depth into the cloud, measured from the illuminated face.

The simulation may result in a large number of zones since the code must
spatially resolve the density fluctuations.
The zone thickness is not
allowed to exceed ${\sim}$0.05 of the period so that each cycle is
divided into at least 20 zones.
This may result in very long execution times.
The total number of zones (this sets the code's execution time)
will be at least 20
times the number of cycles over the nebula.

The keyword \cdCommand{column} will replace
the depth variable in equation \ref{eqn:commandFluctuations} with
the total hydrogen column density.
The period should then be the column
density of one cycle.
This will more evenly weight the final column density
over low- and high-density gas.

The \cdCommand{fluctuations abundances} command proves a mechanism
for varying the gas-phase abundances of the elements.

\section{Globule [density =2, depth =16, power =2]}

This produces a density law that would be appropriate for a power-law
density gradient irradiated from the outside (see, for example, \citealp{Williams1992}).  The total hydrogen density $n(r)$ is given~by
\begin{equation}
n\left( r \right) = n_{\mathrm{o}} \left( {\frac{{R_{scale\,depth}
}}{{R_{scale\,depth}  - \Delta r}}} \right)^\alpha   = n_o \left( {1 -
\frac{{\Delta r}}{{R_{scale\,depth} }}} \right)^{ - \alpha }
\c[\mathrm{cm}^{-3}]% (33)
\end{equation}
where $n_o$ is the background density outside the cloud,
with default value
1 cm$^{-3}$, and $\Delta r$ is the depth into the cloud,
measured from the illuminated face.
The log of $n_o$ is the optional first number on the command line.
The variable \cdVariable{$R_{scale}$} depth is the scale depth
for the cloud and has a default
of one parsec, $R_{scale\, depth} = 3.086\times 10^{18} \; \mathrm{cm}$.
Other scale depths are specified
by the optional second parameter, which must be entered as a log of the
scale depth in cm.
The optional third argument is the index $\alpha$, which has
the default\footnote{The default index was 2 for versions 89 and before.} $\alpha = 1$. The arguments can be omitted from right to left.

\section{hden 5.6, [proportional to R -2,\dots]}

The first number is the log of the total (ionic, atomic, and molecular)
hydrogen density at the illuminated face of the cloud.
This is the sum
\begin{equation}
n\left( {\mathrm{H}} \right) = n\left( {{\mathrm{H}}^{\mathrm{0}} } \right) + n\left(
{{\mathrm{H}}^ +  } \right) + 2n\left( {{\mathrm{H}}_2 } \right) +
\sum\limits_{other} {n\left( {{\mathrm{H}}_{other}^{} } \right)}
\, [\mathrm{cm}^{-3}] .%   (34)
\end{equation}
If the optional keyword \cdCommand{linear} appears then
the number is the density itself and not its log.

For situations where the hydrogen atom is close to LTE and the gas is
hot, there is a problem in defining the neutral hydrogen density because
of the well-known divergence of the partition function, as discussed, for
instance, by \citet{Mihalas1978}.
The atomic hydrogen density is defined as
the total population in all computed levels.
In most circumstances, i.e.,
$n(\mathrm{H}) \leq 10^{15}\; \mathrm{cm}^{-3}$ and $T \le 10^4\; \K$,
the ambiguity is much less than 1\%.

Several options are available to specify optional power-law dependencies
on depth variables.  These are described in the next sub-sections.

\subsection{Power-law dependence on radius}

The second (optional) number is the exponent  for a power-law radial
density dependence, as in the following example:
\begin{verbatim}
hden 9, power =-2
\end{verbatim}
i.e.,
\begin{equation}
n\left( r \right) = n_{\mathrm{o}} \left( {r_{\mathrm{o}} } \right)\,\left(
{\frac{r}{{r_{\mathrm{o}} }}} \right)^\alpha
\,[\mathrm{cm}^{-3}].% (35)
\end{equation}
In this example $n_o$, the density at the illuminated face
of the cloud, will
be $10^9 \, \mathrm{cm}^{-3}$.
The optional power law is relative to radius, the distance
to the central object, not the depth into the cloud.
If $\alpha = -2$ then the
density will be proportional to the inverse square of the radius.
Both
density and photon flux will fall off as an inverse square and the cloud
will tend to have the same ionization parameter (and hence physical
conditions) across the ionized zone.

\subsection{Clouds extending to infinity}

For an inverse-square density law there is a critical value of the number
of ionizing photons emitted by the central object, corresponding to an
ionization front at infinite radius;
\begin{equation}
Q_{crit} \left( {\mathrm{H}} \right) = \alpha _B \left( {T_e }
\right)n_{\mathrm{o}}^2 \,4\,\pi \;r_{\mathrm{o}}^{\mathrm{3}}
\,\, [\mathrm{s}^{-1}] .
\end{equation}
A hydrogen ionization front will not be present and the model will extend
to infinite radius when $Q(\mathrm{H}) \ge Q_{crit}(H)$.
In this expression
$\alpha_B(Te)$ is the
hydrogen Case~B recombination coefficient
and $n_o$ and $r_o$ are the inner density
and radius respectively.
Generally, a hydrogen ionization front will not
be present if the density falls off faster than an inverse square law,
but
rather the level of ionization will tend to \emph{increase}
with increasing radius.
In either case, if a reasonable outer radius is not set, the calculation
will extend to very large radii,
an unphysically small density will result,
and usually the code will crash due to floating point underflow,
followed by division by zero.
It is usually necessary to set an outer radius when
the density falls off with an index $\alpha \le -2$,
since, for most circumstances,
the cloud will remain hot and ionized to infinite radius and zero density.

\subsection{Power-law dependence on depth}

The density will depend on the depth into the cloud rather than the radius
if both the optional exponent \emph{and} the keyword \cdCommand{depth} appear:
\begin{verbatim}
hden 9, power =-2, scale depth = 13
\end{verbatim}
The density is given by
\begin{equation}
n\left( r \right) = n_{\mathrm{o}} \left( {r_{\mathrm{o}} } \right)\left( {1 +
\frac{{\Delta r}}{{R_{scale} }}} \right)^\alpha  \, [\mathrm{cm}^{-3}]% (37)
\end{equation}
where \cdVariable{$R_{scale}$} is the scale depth and $\Delta r$ is the depth.
The log of \cdVariable{$R_{scale}$} [cm] is the
third number on the line and must be set.

\subsection{Power-law dependence on column density }

The local hydrogen density will depend on the total hydrogen column
density if both the optional exponent and the keyword
\cdCommand{column} appear;
\begin{verbatim}
hden 9, power =-2, scale column density = 21
\end{verbatim}
Here the density is given by
\begin{equation}
n\left( r \right) = n_{\mathrm{o}} \left( {r_{\mathrm{o}} } \right)\left( {1 +
\frac{{N\left( {\mathrm{H}} \right)}}{{N\left( {\mathrm{H}} \right)_{scale} }}}
\right)^\alpha  \, [\mathrm{cm}^{-3}]% (38)
\end{equation}
where $N(\mathrm{H})$ is the total hydrogen column density [cm$^{-2}$] from the illuminated
face to the point in question, and $N(\mathrm{H})_{scale}\mathrm{[cm}^{-2}$] is the scale column
density.
The log of $N(\mathrm{H})_{scale}$ is the third number.

