\chapter{OUTPUT}
% !TEX root = hazy2.tex
\label{sec:output}

\section{Overview}

This section defines the output produced by \Cloudy.  Each section begins
with a sample of the output described, and then goes on to describe the
meaning of the printout in greater detail.  The output actually shown is
from the Orion \hii\ Region / PDR / molecular cloud test case
(\cdFilename{orion\_hii\_pdr\_pp.in)}.

\section{Header Information}

Several lines of output echo the input commands and outline some
properties of the initial continuum.

{\setverbatimfontsize{\tiny}
\begin{verbatim}
                                  Cloudy 06.01.02

**************************************06Jan02**************************************
*                                                                                 *
* title the Orion HII Region / PDR / Molecular cloud with an open geometry        *
* #                                                                               *
* # commands controlling continuum =========                                      *
* # the incident continuum is two parts                                           *
* # kurucz continuum and flux of photons striking cloud                           *
* # this is the photosphere of the OVI star, its temperature and phi(H)           *
* table star kurucz 39600K                                                        *
* phi(H) 13                                                                       *
* # this adds the observed hot brems                                              *
* # its temperature (as log of T) and the flux of                                 *
* # photons striking the cloud                                                    *
* brems 6                                                                         *
* phi(h) 10                                                                       *
* #                                                                               *
* # cosmic rays are important for pdr chemistry                                   *
* cosmic rays, background                                                         *
* #                                                                               *
* # commands controlling geometry  =========                                      *
* # this turns off the stop temperature option                                    *
* # so the sim will not stop due to temperature                                   *
* stop temperature off                                                            *
* # this sets the thickness of the HII region & PDR                               *
* stop thickness 0.5 linear parsec                                                *
* # this will result in a milli gauss B-field in molecular region                 *
* magnetic field -5 gauss                                                         *
* # assume constant pressure                                                      *
* constant pressure                                                               *
* set nend 2000                                                                   *
* #                                                                               *
* # other commands for details     =========                                      *
* failures 3                                                                      *
* # mimic existence of unmodeled molecular gas                                    *
* double                                                                          *
* # iterate since lines optically thick                                           *
* iterate                                                                         *
* # set microturbulence in equipartition with B field                             *
* turbulence equipartition                                                        *
* # set the line width so lines appear on the save continuum                     *
* set SaveLwidth 10 km/s                                                         *
* #                                                                               *
* # commands for density & abundances =========                                   *
* # this is the log of the initial H density, cm-3                                *
* hden 4                                                                          *
* # this will speed up the calculation a bit                                      *
* init file="ism.ini"                                                             *
* # this uses HII region abundances, but no grains                                *
* abundances hii region no grains                                                 *
* # this uses orion grains                                                        *
* grains orion                                                                    *
* >>>> mie_read_opc reading file -- graphite_orion_10.opc                    <<<< *
* >>>> mie_read_opc reading file -- silicate_orion_10.opc                    <<<< *
* # turn on PAHs, with an abundance that depends on H0 fraction,                  *
* # as suggested by long-slit observations of Orion bar,                          *
* # with an abundance 3x larger than default built into the code                  *
* grains pah function 3                                                           *
* >>>> mie_read_opc reading file -- pah1_bt94_10.opc                         <<<< *
* #                                                                               *
* # commands controlling output    =========                                      *
* # print lots of faint CO lines                                                  *
* print line faint -4                                                             *
* # normalize to Ha                                                               *
* normalize to "H  1" 6563                                                        *
* #                                                                               *
* # orion hii pdr pp.in                                                           *
* # class hii pdr                                                                 *
*                                                                                 *
***********************************************************************************
\end{verbatim}
}

This begins with the version number of \Cloudy, the date that the version
was released, in the form yy.mm.dd.
The following line gives this date
in another form.

All of the input command lines, with the exception of those starting
with a \#, \%, or *, are echoed before the calculation begins,
and are saved
to be reprinted after the calculation is completed.
{\setverbatimfontsize{\tiny}
\begin{verbatim}
           3198CellPeak1.00E+00   Lo 9.99e-09=912.21cm   Hi-Con:1.20E+02 Ryd   E(hi):7.35E+06Ryd     E(hi):     100.01 MeV
           I(nu>1ryd):   2.4792   Average nu:1.382E+00   I( X-ray):  -1.6194   I(BalC):   2.8255     Phi(BalmrC):  13.7081
           phi(1.0-1.8):12.9508   phi(1.8-4.0): 12.033   phi(4.0-20):  9.388   phi(20--):  7.634     Ion pht flx:1.001E+13
           I(gam ray):   0.0000   phi(gam r):   0.0000   I(Infred):   1.5064   Alf(ox):   0.0000     Total inten:   3.0012
           U(1.0----):3.339E-02   U(4.0----):8.293E-06   T(En-Den):4.586E+01   T(Comp):3.532E+04     nuJnu(912A):5.422E+02
           Occ(FarIR):2.911E+08   Occ(H n=6):1.326E-11   Occ(1Ryd):2.470E-14   Occ(4R):5.400E-20     Occ (Nu-hi):1.201E-32
           Tbr(FarIR):4.835E+05   Tbr(H n=6):5.816E-08   Tbr(1Ryd):3.905E-09   Tbr(4R):3.413E-14     Tbr (Nu-hi):2.269E-25
\end{verbatim}
}
This block of information describes the continuum that strikes the
illuminated face of the cloud.
The full block of information is shown above,
and in the following discussion each line is given again
just before it is described.
{\setverbatimfontsize{\tiny}
\begin{verbatim}
           3198CellPeak1.00E+00   Lo 9.99e-09=912.21cm   Hi-Con:1.20E+02 Ryd   E(hi):7.35E+06Ryd     E(hi):     100.01 MeV
\end{verbatim}
}

This gives the number of numerical frequency cells in the continuum
followed by the energy (in Ryd) of the peak of hydrogen-ionizing
continuum.
This is the point with the largest flux density per unit energy interval
($J_\nu$).
Next is the energy of the low-energy limit of the continuum, in both
Ryd and cm.
The last two numbers are the energies of the high-energy limit
of the continuum in Ryd and MeV.
{\setverbatimfontsize{\tiny}
\begin{verbatim}
                 I(nu>1ryd):   2.4792   Average nu:1.382E+00   I( X-ray):  -1.6194   I(BalC):   2.8255     Phi(BalmrC):  13.7081
\end{verbatim}
}

This line gives the intensity or luminosity of the continuum source.
Luminosities are printed if the inner radius of the cloud is specified.
The units will be energy radiated by the central object into
$4\pi$~sr [erg s$^{-1}$].
If an inner radius is not set then the code will compute the intensity
case and give the emission per unit area of cloud surface.
This is loosely
called the intensity but is more formally $4\pi J$ where $J$ is the proper mean
intensity [erg cm$^{-2}$ s$^{-1}$ sr$^{-1}$ for an emission line; AGN3
Appendix~1].

The line gives the log of the energy (erg s$^{-1}$ cm$^{-2}$
or erg s$^{-1}$, depending
on whether it is the intensity or luminosity case) in the hydrogen-ionizing
continuum (1 Ryd $\le h\nu < 100$~MeV), and the average energy of the
hydrogen-ionizing continuum, in Ryd, weighted by photon number;
\begin{equation}
\left\langle {h\nu } \right\rangle  = \frac{{\int_{1\,Ryd}^\infty  {4\,\pi
\;{J_\nu }\;d{\kern 1pt} \nu } }}{{\int_{1\;Ryd}^\infty  {4\,\pi \;{J_\nu
}/h\nu \;d{\kern 1pt} \nu } }}\quad [\mathrm{Ryd}] .% (400)
\end{equation}
The log of the energy in the X-ray continuum (20.6 Ryd $\le h\nu\le  7676$~Ryd) is
followed by the log of the energy (erg s$^{-1}$ cm$^{-2}$ or erg s$^{-1}$) and the number
of photons (cm$^{-2}$ s$^{-1}$ or s$^{-1}$) in the Balmer continuum (0.25 Ryd to 1.0 Ryd).
{\setverbatimfontsize{\tiny}
\begin{verbatim}
                 phi(1.0-1.8):12.9508   phi(1.8-4.0): 12.033   phi(4.0-20):  9.388   phi(20--):  7.634     Ion pht flx:1.001E+13
\end{verbatim}
}

The third line gives the log of the number of photons
(cm$^{-2}$~s$^{-1}$ or s$^{-1}$)
in four frequency bins (1.0 Ryd $\le h\nu < 1.807$ Ryd, 1.807 Ryd
$\le h\nu < 4.0$ Ryd, 4.0 Ryd $\le h\nu < 20.6$ Ryd,
and 20.6 Ryd $\le h\nu < 7676$ Ryd).
The last number ``Ion pht flx'' is the flux of hydrogen ionizing photons;
\begin{equation}
\Phi \left( {{{\mathrm{H}}^{\mathrm{0}}}} \right) = \frac{{Q\left(
{{{\mathrm{H}}^{\mathrm{0}}}} \right)}}{{4\pi \,{r^2}}}\quad
  [\mathrm{cm}^{-2} \mathrm{s}^{-1}].% (401)
\end{equation}
In this equation $Q(\mathrm{H}^0$) is the total number of
hydrogen-ionizing photons
emitted by the central object (s$^{-1}$), and $r$ is the
separation between the
center of the central object and the illuminated face of the cloud.  Unlike
the majority of the quantities printed in the header,
$\Phi(\mathrm{H}^0$) (per unit area)
is always printed, never $Q(\mathrm{H}^0$) (into 4$\pi$~sr).
{\setverbatimfontsize{\tiny}
\begin{verbatim}
                 I(gam ray):   0.0000   phi(gam r):   0.0000   I(Infred):   1.5064   Alf(ox):   0.0000     Total inten:   3.0012
\end{verbatim}
}

The fourth line of the header gives some information about the low and
high energy portions of the incident continuum.
The first number is the
log of the luminosity or intensity in the gamma-ray ($\sim $100 keV  to
$\sim$100 MeV)
continuum.
The second number is the log of the number of photons over this
energy range.
The third number is the log of the luminosity in the continuum
between 0.25 Ryd and the lowest energy considered,
presently an energy of \emm .
All of these entries are either per unit area, or radiated
into $4\pi$~sr, depending on whether the intensity or luminosity case was
specified.

The entry ``Alf(ox)'' is the spectral index $\alpha_{ox}$,
defined as in Zamorani
et al. (1981), except for the difference in sign convention.
This is the
spectral index which would describe the continuum between 2 keV (147 Ryd)
and 2500\AA\ (0.3645 Ryd) if the continuum could be described
as a single power-law, that is,
\begin{equation}
\frac{{{f_\nu }\left( {2\;{\mathrm{keV}}} \right)}}{{{f_\nu }\left(
{2500\;{\mathrm{{\AA}}}} \right)}} = {\left( {\frac{{{\nu
_{2\;{\mathrm{keV}}}}}}{{{\nu _{2500\;{\mathrm{{\AA}}}}}}}} \right)^\alpha } =
{403.3^\alpha }.% (402)
\end{equation}
The definition of $\alpha_{ox}$ used here is slightly different from that of Zamorani
et al. since implicit negative signs are \emph{never} used by \Cloudy.  Typical
AGN have $\alpha_{ox} \sim  -1.4$.
If no X-rays are present then
$\alpha_{ox} = 0$.
The last number
on the line is the log of the total energy in the continuum between
\emm\ and \egamrymev .
{\setverbatimfontsize{\tiny}
\begin{verbatim}
           log L/Lsun:   3.9743   Abs bol mg:  -5.1858   Abs V mag:   2.4170   Bol cor:  -7.6028     nuFnu(Bbet):  34.5867
\end{verbatim}
}

This line is optional, depending on whether the luminosity or intensity
case is specified.  (It was not printed in this model since we are working
in the intensity case but a sample is shown).  This is printed in the
luminosity case.  First comes the log of the total luminosity in the
continuum in solar units.  The absolute bolometric magnitude, absolute V
magnitude, and the bolometric correction, are then given, followed by the
log of the continuum specific luminosity $[\nu F_\nu(\mathrm{H}\beta)]$
at the wavelength of H$\beta [\mathrm{erg\,s}^{-1}]$.
{\setverbatimfontsize{\tiny}
\begin{verbatim}
                 U(1.0----):3.339E-02   U(4.0----):8.293E-06   T(En-Den):4.586E+01   T(Comp):3.532E+04     nuJnu(912A):5.422E+02
\end{verbatim}
}
This line begins with two ionization parameters.
The first is the
dimensionless ratio of ionizing photon to hydrogen densities,
defined as
\begin{equation}
U \equiv \frac{{\Phi \left( {{{\mathrm{H}}^{\mathrm{0}}}}
\right)}}{{{n_{\mathrm{H}}}c}},%  (403)
\end{equation}
where $n_H$ is the total hydrogen density.
The second number is defined in
a similar way, but the numerator is the number of photons with energies
greater than 4 Ryd (i.e., helium-ionizing).
The third number is the
equivalent black-body temperature corresponding to the energy density $u$
at the illuminated face of the cloud, from the incident continuum and
Stefan's radiation density constant $a$;
\begin{equation}
{T_u} \equiv {\left( {L/4\pi \,{r^2}ac} \right)^{1/4}}\quad
 [\mathrm{K}] .% (404)
\end{equation}
T(Comp) is the Compton temperature of the incident radiation
field.\footnote{For a blackbody radiation field $T_{Compton}$ is roughly 4\% lower than
the blackbody color temperature $T_{color}$ when the energy density temperature
$T_u$ is $> T_{color}$.
Only when $T_u \equiv T_{color}$ does induced Compton heating cause
$T_{Compton} \equiv T_{color}$.
If $T_u > T_{color}$  then $T_{Compton} >
T_{color}$ because of induced
Compton heating.
All of the relevant physics is included in the Compton
temperature printed here.}  The
last number is $4\pi \nu J_\nu(912 \AA )$, the flux at 912\AA\ (erg
cm$^{-2}$ s$^{-1}$), where
$J_\nu$ is the mean intensity of the incident continuum (\citealp{Mihalas1978}).
{\setverbatimfontsize{\tiny}
\begin{verbatim}
           Occ(FarIR):2.911E+08   Occ(H n=6):1.326E-11   Occ(1Ryd):2.470E-14   Occ(4R):5.400E-20     Occ (Nu-hi):1.201E-32
           Tbr(FarIR):4.835E+05   Tbr(H n=6):5.816E-08   Tbr(1Ryd):3.905E-09   Tbr(4R):3.413E-14     Tbr (Nu-hi):2.269E-25
\end{verbatim}
}

These lines give dimensionless photon occupation numbers $\eta(\nu)$,
for the incident continuum at several energies.
The occupation number is defined as
\begin{equation}
{\eta _\nu } \equiv {J_\nu }\left( \nu  \right)\,{\left( {\frac{{2h{\nu
^3}}}{{{c^2}}}} \right)^{ - 1}},% (405)
\end{equation}
and the incident continuum brightness temperature ${T_b}( \nu  )$, [K] is defined as
\begin{equation}
{T_b}\left( \nu  \right) \equiv {J_\nu }\left( \nu  \right){\left(
{\frac{{2k{\kern 1pt} {\nu ^2}}}{{{c^2}}}} \right)^{ - 1}}\quad  [\mathrm{K}].% (406)
\end{equation}
These energies correspond to the lowest frequency considered
(presently \emmmhz );
the ionization potential of the $n = 6$ level of hydrogen
(1/36 Ryd); one Rydberg; four Rydbergs, and the high-energy limit of the
incident continuum.
The energy where the last number is evaluated depends
on the continuum shape.
The energy is given by the fifth number on the
first line of the continuum output.

\section{Chemical composition}
{\setverbatimfontsize{\tiny}
\begin{verbatim}
                                                  Gas Phase Chemical Composition
        H :  0.0000  He: -1.0223  C : -3.5229  N : -4.1549  O : -3.3979  Ne: -4.2218  Mg: -5.5229  Si: -5.3979  S : -5.0000
                                               Cl: -7.0000  Ar: -5.5229  Fe: -5.5229

                                                    Grain Chemical Composition
                                  C : -3.6259  O : -3.9526  Mg: -4.5547  Si: -4.5547  Fe: -4.5547

                                                  Number of grains per hydrogen
                                            Carbonaceous: -14.166  silicate: -14.103
\end{verbatim}
}

The chemical composition of the cloud comes next.
The three blocks of
numbers give the gas-phase abundances of the elements, the abundances
contained in grains, and the number of each type of grains per unit hydrogen.
The numbers are the logs of the number densities of the elements, relative
to the gas-phase hydrogen abundance of unity (so, 0 on the log scale).
Only the active elements are included (those turned off with the \cdCommand{elements off} command are not printed).
If grains are not present then the second
two blocks are not printed.

\section{Comments before or during the calculation}

The code may print comments as the calculation proceeds.
These are
printed with the start of the comment in capital letters to make them easy
to find with a script.
The comments fall into three categories:
\begin{description}
\item[DISASTER] something has unexpectedly caused the calculation to stop.
The results are bogus and should not be trusted.
The code should immediately
return to the calling program.

\item[PROBLEM] something did not go as expected but the calculation is
continuing.  This might be a convergence failure at a point in the cloud.
This is an indication that the code was having difficulties and a few of
these can occur in a normal set of calculations.
The predictions can be
used if only a few occur and if the comments at the end of the calculation
(see page \pageref{sec:CommentsAfterCalculation} below)
do not identify other problems.

\item[NOTE] this gives advice about the calculation.  These can be ignored
if you prefer.
\end{description}

\section{Zone Results}

Next comes a summary of the conditions in the first and last zone.
This print out is done in routine \cdRoutine{PrtZone} which should
be consulted if there are any questions.
The following is the output produced for one zone.
Details follow.
{\setverbatimfontsize{\tiny}
\begin{verbatim}
####  1  Te:9.361E+03 Hden:1.000E+04 Ne:1.101E+04 R:1.000E+30 R-R0:5.223E+11 dR:1.045E+12 NTR:  3 Htot:3.461E-16 T912: 1.48e-05###
 Hydrogen      2.15e-04 1.00e+00 H+o/Hden 1.00e+00 1.00e-11 H-    H2 5.27e-19 3.99e-13 H2+ HeH+ 2.03e-14 Ho+ ColD 2.38e+14 1.11e+18
 Deuterium     2.15e-04 1.00e+00 D+o/Dden 1.00e+00 5.61e-12 D-    HD 6.70e-16 2.92e-12 HD+ HeD+ 1.40e-14 Do+ ColD 3.93e+09 1.83e+13
 Helium        6.57e-04 9.45e-01 5.47e-02 He I2SP3 3.52e-06 Comp H,C 1.75e-22 4.64e-23 Fill Fac 1.00e+00 Gam1/tot 1.00e+00
 He singlet n  6.54e-04 2.35e-11 6.59e-18 2.09e-18 3.18e-18 2.52e-18 He tripl 3.52e-06 9.32e-16 7.41e-18 4.77e-17 7.56e-18
 Pressure      NgasTgas 2.06e+08 P(total) 2.84e-08 P( gas ) 2.84e-08 P(Radtn) 1.79e-11 Rad accl 3.51e-05 ForceMul 3.33e+03
               Texc(La) 4.21e+03 T(contn) 4.59e+01 T(diffs) 2.03e+00 nT (c+d) 1.16e+07 Prad/Gas 6.30e-04 Pmag/Gas 2.80e-04
 gra-orion01*  DustTemp 2.12e+02 Pot Volt 5.32e+00 Chrg (e) 1.21e+02 drf cm/s 4.83e+03 Heating: 4.26e-18 Frac tot 1.23e-02
 gra-orion02*  DustTemp 2.02e+02 Pot Volt 5.08e+00 Chrg (e) 1.43e+02 drf cm/s 5.26e+03 Heating: 3.69e-18 Frac tot 1.07e-02
 gra-orion03*  DustTemp 1.91e+02 Pot Volt 4.86e+00 Chrg (e) 1.70e+02 drf cm/s 5.60e+03 Heating: 3.21e-18 Frac tot 9.26e-03
 gra-orion04*  DustTemp 1.81e+02 Pot Volt 4.66e+00 Chrg (e) 2.01e+02 drf cm/s 5.86e+03 Heating: 2.79e-18 Frac tot 8.06e-03
 gra-orion05   DustTemp 1.70e+02 Pot Volt 4.47e+00 Chrg (e) 2.39e+02 drf cm/s 6.07e+03 Heating: 2.43e-18 Frac tot 7.03e-03
 gra-orion06   DustTemp 1.60e+02 Pot Volt 4.30e+00 Chrg (e) 2.85e+02 drf cm/s 6.22e+03 Heating: 2.13e-18 Frac tot 6.16e-03
 gra-orion07   DustTemp 1.50e+02 Pot Volt 4.15e+00 Chrg (e) 3.40e+02 drf cm/s 6.33e+03 Heating: 1.87e-18 Frac tot 5.41e-03
 gra-orion08   DustTemp 1.40e+02 Pot Volt 4.02e+00 Chrg (e) 4.07e+02 drf cm/s 6.42e+03 Heating: 1.65e-18 Frac tot 4.77e-03
 gra-orion09   DustTemp 1.31e+02 Pot Volt 3.91e+00 Chrg (e) 4.89e+02 drf cm/s 6.48e+03 Heating: 1.46e-18 Frac tot 4.21e-03
 gra-orion10   DustTemp 1.22e+02 Pot Volt 3.80e+00 Chrg (e) 5.89e+02 drf cm/s 6.54e+03 Heating: 1.29e-18 Frac tot 3.73e-03
 sil-orion01*  DustTemp 1.55e+02 Pot Volt 2.85e+00 Chrg (e) 6.44e+01 drf cm/s 1.45e+04 Heating: 2.44e-18 Frac tot 7.06e-03
 sil-orion02*  DustTemp 1.49e+02 Pot Volt 2.70e+00 Chrg (e) 7.58e+01 drf cm/s 1.62e+04 Heating: 2.09e-18 Frac tot 6.03e-03
 sil-orion03*  DustTemp 1.43e+02 Pot Volt 2.57e+00 Chrg (e) 8.92e+01 drf cm/s 1.77e+04 Heating: 1.78e-18 Frac tot 5.15e-03
 sil-orion04   DustTemp 1.37e+02 Pot Volt 2.44e+00 Chrg (e) 1.05e+02 drf cm/s 1.91e+04 Heating: 1.53e-18 Frac tot 4.42e-03
 sil-orion05   DustTemp 1.31e+02 Pot Volt 2.33e+00 Chrg (e) 1.24e+02 drf cm/s 2.02e+04 Heating: 1.32e-18 Frac tot 3.81e-03
 sil-orion06   DustTemp 1.26e+02 Pot Volt 2.24e+00 Chrg (e) 1.47e+02 drf cm/s 2.11e+04 Heating: 1.14e-18 Frac tot 3.30e-03
 sil-orion07   DustTemp 1.20e+02 Pot Volt 2.15e+00 Chrg (e) 1.76e+02 drf cm/s 2.18e+04 Heating: 9.93e-19 Frac tot 2.87e-03
 sil-orion08   DustTemp 1.15e+02 Pot Volt 2.08e+00 Chrg (e) 2.10e+02 drf cm/s 2.23e+04 Heating: 8.67e-19 Frac tot 2.50e-03
 sil-orion09   DustTemp 1.10e+02 Pot Volt 2.01e+00 Chrg (e) 2.52e+02 drf cm/s 2.26e+04 Heating: 7.60e-19 Frac tot 2.20e-03
 sil-orion10   DustTemp 1.06e+02 Pot Volt 1.96e+00 Chrg (e) 3.03e+02 drf cm/s 2.28e+04 Heating: 6.69e-19 Frac tot 1.93e-03
 pah-bt9401 *  DustTemp 3.41e+02 Pot Volt 6.60e+00 Chrg (e) 1.06e+00 drf cm/s 1.97e+02 Heating: 9.29e-22 Frac tot 2.68e-06
 pah-bt9402 *  DustTemp 3.44e+02 Pot Volt 6.64e+00 Chrg (e) 1.28e+00 drf cm/s 2.13e+02 Heating: 1.01e-21 Frac tot 2.92e-06
 pah-bt9403 *  DustTemp 3.47e+02 Pot Volt 6.61e+00 Chrg (e) 1.49e+00 drf cm/s 2.37e+02 Heating: 1.07e-21 Frac tot 3.08e-06
 pah-bt9404 *  DustTemp 3.50e+02 Pot Volt 6.54e+00 Chrg (e) 1.70e+00 drf cm/s 2.67e+02 Heating: 1.09e-21 Frac tot 3.16e-06
 pah-bt9405 *  DustTemp 3.53e+02 Pot Volt 6.44e+00 Chrg (e) 1.92e+00 drf cm/s 3.02e+02 Heating: 1.09e-21 Frac tot 3.16e-06
 pah-bt9406 *  DustTemp 3.55e+02 Pot Volt 6.45e+00 Chrg (e) 2.22e+00 drf cm/s 3.29e+02 Heating: 1.13e-21 Frac tot 3.26e-06
 pah-bt9407 *  DustTemp 3.57e+02 Pot Volt 6.47e+00 Chrg (e) 2.54e+00 drf cm/s 3.60e+02 Heating: 1.17e-21 Frac tot 3.38e-06
 pah-bt9408 *  DustTemp 3.59e+02 Pot Volt 6.43e+00 Chrg (e) 2.87e+00 drf cm/s 4.00e+02 Heating: 1.18e-21 Frac tot 3.40e-06
 pah-bt9409 *  DustTemp 3.61e+02 Pot Volt 6.46e+00 Chrg (e) 3.27e+00 drf cm/s 4.35e+02 Heating: 1.20e-21 Frac tot 3.46e-06
 pah-bt9410 *  DustTemp 3.63e+02 Pot Volt 6.48e+00 Chrg (e) 3.71e+00 drf cm/s 4.74e+02 Heating: 1.22e-21 Frac tot 3.52e-06
 Carbon        6.76e-07 1.91e-02 9.70e-01 1.10e-02 1.77e-04 2.36e-09 0.00e+00 H2O+/O   0.00e+00 OH+/Otot 0.00e+00 Hex(tot) 0.00e+00
 Nitrogen      1.60e-06 1.72e-02 9.69e-01 1.41e-02 4.77e-05 1.34e-07 9.81e-14 0.00e+00 O2/Ototl 0.00e+00 O2+/Otot 0.00e+00
 Oxygen        9.23e-06 1.17e-01 8.59e-01 2.47e-02 6.97e-05 9.60e-08 4.83e-11 0.00e+00 0.00e+00
 Neon          5.77e-05 3.93e-01 5.95e-01 1.17e-02 7.25e-05 5.08e-08 1.99e-12 1.47e-16 0.00e+00 0.00e+00 0.00e+00
 Magnesium     6.41e-06 7.31e-03 9.65e-01 2.74e-02 1.76e-04 5.41e-07 2.86e-10 3.72e-14 9.96e-19 0.00e+00 0.00e+00 0.00e+00 0.00e+00
 Silicon    0  1.13e-07 8.39e-03 8.16e-01 1.56e-01 1.86e-02 4.68e-05 3.08e-08 5.13e-12 1.63e-16 0.00e+00 0.00e+00 0.00e+00 0.00e+00
 Sulphur    0  7.25e-08 7.46e-03 9.03e-01 8.81e-02 1.60e-03 1.67e-04 5.91e-06 9.11e-10 3.49e-14 3.20e-19 0.00e+00 0.00e+00 0.00e+00
 Chlorine   0  2.51e-07 1.32e-02 9.57e-01 2.92e-02 7.57e-04 5.50e-05 7.07e-07 9.54e-09 2.94e-13 2.17e-18 0.00e+00 0.00e+00 0.00e+00
 Argon      0  9.50e-08 2.83e-03 9.69e-01 2.73e-02 4.35e-04 2.65e-05 3.24e-07 1.53e-09 1.16e-11 8.21e-17 0.00e+00 0.00e+00 0.00e+00
 Iron       0  1.78e-08 2.88e-04 1.08e-01 8.56e-01 3.26e-02 2.33e-03 2.33e-05 2.30e-07 1.28e-10 1.94e-14 9.07e-19 0.00e+00 0.00e+00
\end{verbatim}
}

The results of calculations for the first and last zones are always
printed.
Results for intermediate zones can be printed if desired (see
the \cdCommand{print every} command).
The following is a line-by-line description of
the output produced for each printed zone.
{\setverbatimfontsize{\tiny}
\begin{verbatim}
####  1  Te:9.361E+03 Hden:1.000E+04 Ne:1.101E+04 R:1.000E+30 R-R0:5.223E+11 dR:1.045E+12 NTR:  3 Htot:3.461E-16 T912: 1.48e-05###
\end{verbatim}
}

The line begins with a series of \#\#\#\# characters to make it easy
to locate with an editor.
The zone number is the first number and it is
followed by the electron temperature of the zone (``Te'').
A lower case
``u'' will appear before the ``Te'' if the temperature solution is possibly
thermally unstable.
This occurs when the derivative of the net cooling
with respect to temperature is negative.
This is discussed further in the
section on thermal stability problems starting on page
\pageref{sec:ThermalStabilityProblems} below.
The total hydrogen (``Hden'')
and electron (``Ne'') densities (cm$^{-3}$) follow.
The next number (``R'')
is the distance from the center of the central object to the center of the
zone.
The depth, the distance between the illuminated face of the cloud
and the center of the zone, (``R-R0'', or ``r-r$^{\mathrm{o}}$''),
and the thickness of
the zone (``dR'', or $\delta r$), (all are in cm), follow.
The inner edge of the
zone is $( {r - {r_o}} ) - \delta r/2$
from the illuminated face of the cloud.
The line ends with a number
indicating how many ionization iterations were needed for this zone to
converge (NTR), followed by the total heating\footnote{\Cloudy\ defines heating as the energy input by the freed photoelectron,
or $h\nu - $IP, where IP is the ionization potential of the atom or ion, and
$h\nu$ is the energy of the photon.  See AGN3 for more details.} (``Htot''; photoelectric
and otherwise, erg cm$^{-3}$ s$^{-1}$),
and the optical depth between the
\emph{illuminated}
face of the cloud and the \emph{outer} edge of the zone at the Lyman limit (T912;
the number is the \emph{total absorption} optical depth at 912\AA,
and not the
hydrogen Lyman-limit optical depth).
{\setverbatimfontsize{\tiny}
\begin{verbatim}
WIND; V:-7.000e+00km/s G:-0.00e+00 Accel: 8.62e-06 Fr(cont): 1.000 Fr(line): 0.000 Fr(dP): 0.000
\end{verbatim}
}

A line describing the velocity and acceleration of the zone is printed
if the cloud is a wind.
The numbers are the wind velocity at the outer
edge of the current zone (km s$^{-1}$), inward gravitational acceleration (cm s$^{-2}$), total outward radiative acceleration (cm s$^{-2}$),
and the fraction of
this acceleration caused by the incident continuum, line driving, and the
gradient of the radiation pressure.
{\setverbatimfontsize{\tiny}
\begin{verbatim}
P(Lines):(Mg 2  2796A 0.32) (Mg 2  2803A 0.20) (H  1  6563A 0.13) (H  1 1.875m 0.10) (Fe 2  1786A 0.06) (H  1  4861A 0.05)
\end{verbatim}
}
A line describing the source of the radiation pressure is generated if
the ratio of line radiation to gas pressure, $P_{rad}/P_{gas}$,
is greater than 5\%.
The line begins with the label ``P(Lines)'' and
continues with the fraction
of the total radiation pressure produced by that emission line, the
spectroscopic designation of the line, and its wavelength.
Up to twenty
lines can be printed, although in most cases only \la\
and a few others will dominate.
{\setverbatimfontsize{\tiny}
\begin{verbatim}
 Hydrogen      2.15e-04 1.00e+00 H+o/Hden 1.00e+00 1.00e-11 H-    H2 5.27e-19 3.99e-13 H2+ HeH+ 2.03e-14 Ho+ ColD 2.38e+14 1.11e+18
 Deuterium     2.15e-04 1.00e+00 D+o/Dden 1.00e+00 5.61e-12 D-    HD 6.70e-16 2.92e-12 HD+ HeD+ 1.40e-14 Do+ ColD 3.93e+09 1.83e+13
\end{verbatim}
}

The line begins with the ratios $n(\mathrm{H}^0)/n(H_{tot})$ and
$n(\mathrm{H}^+)/n(\mathrm{H}_{tot})$ where
$\mathrm{H}_{tot}$ is the total density in H all forms (including molecular).
If \cdCommand{print h-like departure coefficients} has been specified
then departure coefficients
are also printed on the following line.
Neutral hydrogen H$^0$ is defined
to be the total population of atomic hydrogen in all explicitly computed
bound levels.
Next comes ``H+o/Hden'', the ratio
$[n(\mathrm{H}^0) + n(\mathrm{H}^+)]/n(\mathrm{H}_{tot})$.

The following five numbers give densities of the negative hydrogen ion
and several molecules (\hminus, \htwo, H$_2^+$, and HeH$^+$)
relative to the total hydrogen density.
Note that, with this definition of the hydrogen density, a fully
molecular gas will have $n(\mathrm{H}2)/n(\mathrm{H})=0.5$.
These molecular abundances are
also expressed as departure coefficients if the
\cdCommand{print departure coefficients}
command occurs.
The last number is the H$^0$ and H$^+$ column densities
(cm$^{-2}$).

The next line gives very similar information for deuterium.

{\setverbatimfontsize{\tiny}
\begin{verbatim}
 H  1 1S-12    1.39e-01 3.43e-04 1.02e-03 1.02e-03 1.24e-03 1.62e-03 2.12e-03 2.73e-03 3.43e-03 4.23e-03 5.12e-03 6.12e-03 7.20e-03
 H  1  rest    8.39e-03 9.66e-03 1.10e-02 1.25e-02 1.41e-02 1.57e-02 1.75e-02 1.93e-02 2.13e-02 2.33e-02 2.54e-02 2.77e-02 3.00e-02
               3.24e-02 3.49e-02 3.75e-02 4.02e-02 4.30e-02 4.59e-02 4.89e-02 5.20e-02 5.51e-02 5.84e-02 6.18e-02 6.52e-02 6.88e-02
               7.24e-02 7.62e-02 8.00e-02 8.39e-02 8.79e-02 9.21e-02 9.63e-02 1.01e-01 1.05e-01 1.09e-01 1.14e-01 1.19e-01
\end{verbatim}
}
This information is only printed if the
\cdCommand{print H-like populations} command occurs.
The numbers give the populations of the \hO\ levels relative to the
ionized hydrogen density.
All of these populations usually are relative
to the ionized hydrogen density, but can also be printed as LTE departure
coefficients if the \cdCommand{print departure coefficients} command
is given.
{\setverbatimfontsize{\tiny}
\begin{verbatim}
Helium        6.57e-04 9.45e-01 5.47e-02 He I2SP3 3.52e-06 Comp H,C 1.75e-22 4.64e-23 Fill Fac 1.00e+00 Gam1/tot 1.00e+00
\end{verbatim}
}

The first three numbers are the total populations of the three ionization
stages of helium relative to the total helium abundance.
The population
of atomic helium is the sum of the total population in the triplets and
singlets, including the population of all explicitly computed levels of
each.
These populations can also be expressed as departure coefficients
if this option is set with the \cdCommand{print departure coefficients} command.
The
population of He~2$^3$~S, relative to the total helium abundance, follows.
The Compton heating and cooling rates (both erg cm$^{-3}$~s$^{-1}$) are next, followed
by the gas filling factor.
The last number is the fraction of the total
hydrogen ionizations that are caused by photoionization from the ground
state.
{\setverbatimfontsize{\tiny}
\begin{verbatim}
He singlet n  6.54e-04 2.35e-11 6.59e-18 2.09e-18 3.18e-18 2.52e-18 He tripl 3.52e-06 9.32e-16 7.41e-18 4.77e-17 7.56e-18
\end{verbatim}
}

The first numbers are the level populations of the $l$-levels within
$n=1$ to 3 of the He$^0$ singlets.
The next group consists of He$^0$ triplet populations
of $2S$, the three $2^3P_j$, levels and the $3S$, $3P$, and $3D$ levels.  All populations
are relative to the total helium abundance.
Departure coefficients are
also printed if requested.
{\setverbatimfontsize{\tiny}
\begin{verbatim}
Pressure      NgasTgas 2.06e+08 P(total) 2.84e-08 P( gas ) 2.84e-08 P(Radtn) 1.79e-11 Rad accl 3.51e-05 ForceMul 3.33e+03
              Texc(La) 4.21e+03 T(contn) 4.59e+01 T(diffs) 2.03e+00 nT (c+d) 1.16e+07 Prad/Gas 6.30e-04 Pmag/Gas 2.80e-04
\end{verbatim}
}

Some information concerning the pressure is printed.
The gas equation
of state includes thermal gas pressure, the radiation pressure due to trapped
line emission, magnetic and turbulent pressure, and the radiation pressure
due to absorption of the incident continuum.
The first number is the gas
pressure $n_{gas}\, T_{gas}$ (with units cm$^{-3}$~K), followed by the total pressure
(dynes cm$^{-2}$), and is followed by the gas pressure
$(n_{gas}\, kT_{gas})$ in dynes cm$^{-2}$.
The radiation pressure follows.  The second to last number is the
radiative acceleration (cm s$^{-2}$) at the inner edge of this zone.
The
radiative acceleration is computed with all continuous scattering and
absorption opacities included.  The last number is a force multiplier,
defined as in \citet{Tarter1973}, and is the ratio of total opacity
to electron scattering opacity.

The second line gives more information.
The line starts with
``Texc(La)'', the excitation temperature $T_{exc}$ of \la,
defined as
\begin{equation}
\frac{{n\left( {2p} \right)/g\left( {2p} \right)}}{{n\left( {1s}
\right)/g\left( {1s} \right)}} = \exp \left[ { - h\nu /k{T_{exc}}\left(
{L\alpha } \right)} \right].% (407)
\end{equation}
This is followed by the temperature corresponding to the energy density
of the attenuated incident continuum (``T(contn)'')
and the diffuse continua (``T(diffs)'').
This includes all trapped lines and diffuse continuous
emission.
The entry ``nT (c+d)''is the energy density of the sum of these
two continua expressed as an equivalent pressure $nT$ [cm$^{-3}$~K].
The line
ends with the ratios of the radiation to gas pressure ``Prad/Gas'' and the
ratio of magnetic to gas pressure ``Pmag/Gas''.

{\setverbatimfontsize{\tiny}
\begin{verbatim}
 gra-orion01*  DustTemp 2.12e+02 Pot Volt 5.32e+00 Chrg (e) 1.21e+02 drf cm/s 4.83e+03 Heating: 4.26e-18 Frac tot 1.23e-02
 gra-orion02*  DustTemp 2.02e+02 Pot Volt 5.08e+00 Chrg (e) 1.43e+02 drf cm/s 5.26e+03 Heating: 3.69e-18 Frac tot 1.07e-02
 gra-orion03*  DustTemp 1.91e+02 Pot Volt 4.86e+00 Chrg (e) 1.70e+02 drf cm/s 5.60e+03 Heating: 3.21e-18 Frac tot 9.26e-03
 gra-orion04*  DustTemp 1.81e+02 Pot Volt 4.66e+00 Chrg (e) 2.01e+02 drf cm/s 5.86e+03 Heating: 2.79e-18 Frac tot 8.06e-03
 gra-orion05   DustTemp 1.70e+02 Pot Volt 4.47e+00 Chrg (e) 2.39e+02 drf cm/s 6.07e+03 Heating: 2.43e-18 Frac tot 7.03e-03
 gra-orion06   DustTemp 1.60e+02 Pot Volt 4.30e+00 Chrg (e) 2.85e+02 drf cm/s 6.22e+03 Heating: 2.13e-18 Frac tot 6.16e-03
 gra-orion07   DustTemp 1.50e+02 Pot Volt 4.15e+00 Chrg (e) 3.40e+02 drf cm/s 6.33e+03 Heating: 1.87e-18 Frac tot 5.41e-03
 gra-orion08   DustTemp 1.40e+02 Pot Volt 4.02e+00 Chrg (e) 4.07e+02 drf cm/s 6.42e+03 Heating: 1.65e-18 Frac tot 4.77e-03
 gra-orion09   DustTemp 1.31e+02 Pot Volt 3.91e+00 Chrg (e) 4.89e+02 drf cm/s 6.48e+03 Heating: 1.46e-18 Frac tot 4.21e-03
 gra-orion10   DustTemp 1.22e+02 Pot Volt 3.80e+00 Chrg (e) 5.89e+02 drf cm/s 6.54e+03 Heating: 1.29e-18 Frac tot 3.73e-03
 sil-orion01*  DustTemp 1.55e+02 Pot Volt 2.85e+00 Chrg (e) 6.44e+01 drf cm/s 1.45e+04 Heating: 2.44e-18 Frac tot 7.06e-03
 sil-orion02*  DustTemp 1.49e+02 Pot Volt 2.70e+00 Chrg (e) 7.58e+01 drf cm/s 1.62e+04 Heating: 2.09e-18 Frac tot 6.03e-03
 sil-orion03*  DustTemp 1.43e+02 Pot Volt 2.57e+00 Chrg (e) 8.92e+01 drf cm/s 1.77e+04 Heating: 1.78e-18 Frac tot 5.15e-03
 sil-orion04   DustTemp 1.37e+02 Pot Volt 2.44e+00 Chrg (e) 1.05e+02 drf cm/s 1.91e+04 Heating: 1.53e-18 Frac tot 4.42e-03
 sil-orion05   DustTemp 1.31e+02 Pot Volt 2.33e+00 Chrg (e) 1.24e+02 drf cm/s 2.02e+04 Heating: 1.32e-18 Frac tot 3.81e-03
 sil-orion06   DustTemp 1.26e+02 Pot Volt 2.24e+00 Chrg (e) 1.47e+02 drf cm/s 2.11e+04 Heating: 1.14e-18 Frac tot 3.30e-03
 sil-orion07   DustTemp 1.20e+02 Pot Volt 2.15e+00 Chrg (e) 1.76e+02 drf cm/s 2.18e+04 Heating: 9.93e-19 Frac tot 2.87e-03
 sil-orion08   DustTemp 1.15e+02 Pot Volt 2.08e+00 Chrg (e) 2.10e+02 drf cm/s 2.23e+04 Heating: 8.67e-19 Frac tot 2.50e-03
 sil-orion09   DustTemp 1.10e+02 Pot Volt 2.01e+00 Chrg (e) 2.52e+02 drf cm/s 2.26e+04 Heating: 7.60e-19 Frac tot 2.20e-03
 sil-orion10   DustTemp 1.06e+02 Pot Volt 1.96e+00 Chrg (e) 3.03e+02 drf cm/s 2.28e+04 Heating: 6.69e-19 Frac tot 1.93e-03
 pah-bt9401 *  DustTemp 3.41e+02 Pot Volt 6.60e+00 Chrg (e) 1.06e+00 drf cm/s 1.97e+02 Heating: 9.29e-22 Frac tot 2.68e-06
 pah-bt9402 *  DustTemp 3.44e+02 Pot Volt 6.64e+00 Chrg (e) 1.28e+00 drf cm/s 2.13e+02 Heating: 1.01e-21 Frac tot 2.92e-06
 pah-bt9403 *  DustTemp 3.47e+02 Pot Volt 6.61e+00 Chrg (e) 1.49e+00 drf cm/s 2.37e+02 Heating: 1.07e-21 Frac tot 3.08e-06
 pah-bt9404 *  DustTemp 3.50e+02 Pot Volt 6.54e+00 Chrg (e) 1.70e+00 drf cm/s 2.67e+02 Heating: 1.09e-21 Frac tot 3.16e-06
 pah-bt9405 *  DustTemp 3.53e+02 Pot Volt 6.44e+00 Chrg (e) 1.92e+00 drf cm/s 3.02e+02 Heating: 1.09e-21 Frac tot 3.16e-06
 pah-bt9406 *  DustTemp 3.55e+02 Pot Volt 6.45e+00 Chrg (e) 2.22e+00 drf cm/s 3.29e+02 Heating: 1.13e-21 Frac tot 3.26e-06
 pah-bt9407 *  DustTemp 3.57e+02 Pot Volt 6.47e+00 Chrg (e) 2.54e+00 drf cm/s 3.60e+02 Heating: 1.17e-21 Frac tot 3.38e-06
 pah-bt9408 *  DustTemp 3.59e+02 Pot Volt 6.43e+00 Chrg (e) 2.87e+00 drf cm/s 4.00e+02 Heating: 1.18e-21 Frac tot 3.40e-06
 pah-bt9409 *  DustTemp 3.61e+02 Pot Volt 6.46e+00 Chrg (e) 3.27e+00 drf cm/s 4.35e+02 Heating: 1.20e-21 Frac tot 3.46e-06
 pah-bt9410 *  DustTemp 3.63e+02 Pot Volt 6.48e+00 Chrg (e) 3.71e+00 drf cm/s 4.74e+02 Heating: 1.22e-21 Frac tot 3.52e-06
\end{verbatim}
}

Some properties of the grain populations are printed if they are present.
Each line gives the results of calculations for a specific type and size
of grain.
Graphite and silicate are normally included when grains are
present.
Each line begins with the name of the grain and an asterisk appears
if quantum heating was important for the species.
Quantum heating is only
computed if it is significant due to its computational expense.
The
remainder of the line gives the equilibrium temperature of the grain, the
potential in volts, the charge, the drift velocity, the gas heating (erg
cm$^{-3}$ s$^{-1}$) due to grain electron photoemission, and the dimensionless fraction
of the total gas heating due to grain electron photoemission.
For
quantum-heated grains the temperature is the average weighted by $T^4$.

{\setverbatimfontsize{\tiny}
\begin{verbatim}
Molecules     CH/Ctot: 3.65e-04 CH+/Ctot 5.82e-13 CO/Ctot: 5.52e-01 CO+/Ctot 1.95e-14 H2O/Otot 3.27e-09 OH/Ototl 4.18e-11
\end{verbatim}
}

A line giving relative abundances of some molecules is printed if the
molecular fraction is significant.
All molecular abundances are relative
to either the total carbon or total oxygen abundance (this is indicated
in the label for each).
In order, the molecules are CH, CH$^+$, CO,
CO$^+$, H$_2$O, and OH.
{\setverbatimfontsize{\tiny}
\begin{verbatim}
Lithium       8.94e-02 9.11e-01 9.53e-10 0.00e+00 Berylliu 9.99e-01 6.40e-04 6.38e-05 7.06e-10 0.00e+00 sec ion: 7.66e-15

Abundances of each stage of ionization of lithium and beryllium relative
to the total gas-phase abundance of the element are followed by the
suprathermal secondary ionization rate [s$^{-1}$].
{\setverbatimfontsize{\tiny}
\begin{verbatim}
Carbon        6.76e-07 1.91e-02 9.70e-01 1.10e-02 1.77e-04 2.36e-09 0.00e+00 H2O+/O   0.00e+00 OH+/Otot 0.00e+00 Hex(tot) 0.00e+00
\end{verbatim}
}
The abundances of the seven stages of ionization of carbon relative to
the total gas-phase carbon abundance begin the line.
The abundance of H$_2$O$^+$
and OH$^+$ relative to the total gas-phase oxygen abundance are given.  These
are followed by ``Hex(tot)'', the extra heat  (erg cm$^{-3}$ s$^{-1}$) due to fast
neutrons, dissipation of turbulence, or added with the
\cdCommand{hextra} command.
{\setverbatimfontsize{\tiny}
\begin{verbatim}
Nitrogen      1.60e-06 1.72e-02 9.69e-01 1.41e-02 4.77e-05 1.34e-07 9.81e-14 0.00e+00 O2/Ototl 0.00e+00 O2+/Otot 0.00e+00
\end{verbatim}
}

The relative populations of the eight ionization stages of nitrogen are
printed first.
The abundance of $O_2$ and $O_2^+$, relative to the total oxygen
abundance, follows.
{\setverbatimfontsize{\tiny}
\begin{verbatim}
 Silicon    0  1.13e-07 8.39e-03 8.16e-01 1.56e-01 1.86e-02 4.68e-05 3.08e-08 5.13e-12 1.63e-16 0.00e+00 0.00e+00 0.00e+00 0.00e+00
 Sulphur    0  7.25e-08 7.46e-03 9.03e-01 8.81e-02 1.60e-03 1.67e-04 5.91e-06 9.11e-10 3.49e-14 3.20e-19 0.00e+00 0.00e+00 0.00e+00
 Chlorine   0  2.51e-07 1.32e-02 9.57e-01 2.92e-02 7.57e-04 5.50e-05 7.07e-07 9.54e-09 2.94e-13 2.17e-18 0.00e+00 0.00e+00 0.00e+00
 Argon      1  9.50e-08 2.83e-03 9.69e-01 2.73e-02 4.35e-04 2.65e-05 3.24e-07 1.53e-09 1.16e-11 8.21e-17 0.00e+00 0.00e+00 0.00e+00
 Iron       3  1.78e-08 2.88e-04 1.08e-01 8.56e-01 3.26e-02 2.33e-03 2.33e-05 2.30e-07 1.28e-10 1.94e-14 9.07e-19 0.00e+00 0.00e+00
\end{verbatim}
}

There are too many ionization stages to print across the line for elements
more massive than neon.
Although all stages with non-trivial abundances
are computed, only the highest twelve stages of ionization are printed.
The first number is an integer indicating how many stages are ``off the
page to the left''.
If the number is 2, then the first printed stage of
ionization is twice ionized, i.e., Fe$^{+2}$.

\section{Comments after the calculation}
\label{sec:CommentsAfterCalculation}

{\setverbatimfontsize{\tiny}
\begin{verbatim}
    the Orion HII Region / PDR / Molecular cloud with an open geometry
   Calculation stopped because outer radius reached. Iteration 1 of 2
\end{verbatim}
}
A series of messages appear after the printout of the last zone.
The
first will say why the calculation stopped.
In a valid calculation the
model will stop because one of the specified stopping criteria specified
was met.
If no other criteria are specified then the calculation usually
stops when the default lowest temperature of
\TEMPSTOPDEFAULT\ is reached.
If the
code stops because of an unintended reason (i.e., internal errors, or
reaching the default limit to the number of zones) then a warning is printed
saying that the calculation may have halted prematurely.

Only one stopping criterion message will be printed.
The possible
messages, and their interpretations, are:
\begin{description}
\item[ \dots because of radiation pressure]  By default a cloud will have constant
density.  \Cloudy\ will keep the total pressure, particle and radiation,
constant if constant pressure is specified with the \cdCommand{constant
pressure}
command.  The radiation pressure is small at the boundaries of the cloud,
so the cloud will be unstable if the ratio of radiation to total pressure
exceeds 0.5.  The calculation stops, and this message is generated, if
${{\mathop{\mathrm P}\nolimits} _{rad}}/{P_{tot}} > 0.5$
    occurs after the first iteration.

\item[\dots because lowest EDEN reached]  The calculation can be forced to stop
when the electron density (\cdCommand{eden}) falls below the value set by the
\cdCommand{stop eden}
command.  This can be used to stop the calculation at an ionization front.
The default lowest electron density is negative, so this stopping criterion
occurs only when the command is entered.

\item[\dots because low electron fraction]  The calculation can be forced to
stop when the ratio of electron to hydrogen densities falls below a certain
value, as set by the \cdCommand{stop efrac} command.  This can be used to stop the
calculation at an ionization front when the hydrogen density there is not
known (for instance, in a constant pressure model).  The default lowest
electron density is negative, so this stopping criterion applies only when
the command is entered.

\item[\dots because large \htwo/H fraction]  The calculation can be forced
to stop when the ratio of densities of molecular hydrogen to total hydrogen
rises above the value set by the \cdCommand{stop mfrac} command.  The molecular fraction
is defined as $2n(\mathrm{H}_2)/n(\mathrm{H}_{tot})$.  This can be used to stop the calculation
at some depth into a PDR.  The default highest molecular density is $>> 1$,
so this stopping criterion occurs only when the \cdCommand{stop mfrac} command is
entered.

\item[\dots because wind veloc too small]  The code can perform a wind calculation
which includes the outward force due to radiation pressure and the inward
force of gravity.  This message is printed if the gas decelerates to a stop.

\item[\dots because code returned BUSTED]  The calculation stopped because
something bad happened.  Please post the input script and version number
on the discussion board.

\item[\dots because DRAD small -  set DRMIN]  The Str\"omgren radius of the
H$^+$
zone is estimated at the start of the calculation and the smallest allowed
zone thickness is set to a very small fraction of this.  The calculation
will stop if the zone thickness falls below this smallest thickness.  This
can occur because of any of several logical errors within \Cloudy\ (adaptive
logic is used to continuously adjust the zone thickness), although it can
rarely occur for physical reasons as well.  The smallest thickness can be
reset to any number with the set \cdCommand{drmin} command but it should not be necessary
to do this.  Please post the input script and version number on the
discussion board.

\item[\dots because DR small rel to thick]  The depth into the cloud is stored
as the double precision variable \emph{depth} and the zone thickness is stored
as the double precision variable \emph{drad}.  If the zone size becomes too small
relative to the depth (\emph{drad/depth} $<$ 10$^{-14}$) then the depth variable will
underflow such that \emph{depth + drad = depth}.  The calculation will stop if
this problem prevents the density from being properly evaluated.

\item[\dots because optical depth reached]  The largest allowed continuous
absorption optical depth can be set with the \cdCommand{stop optical depth} command.
The command specifies both the absorption optical depth, and the energy
at which it is to be evaluated.  Scattering opacities are not included since
their effects are very geometry dependent.  If the calculation stops because
the largest continuum optical depth is reached, then this line is printed.
This line is also printed if the \cdCommand{stop effective column density} command is
used to stop the calculation, since this command is actually a form of the
\cdCommand{stop optical depth} command.

\item[\dots because outer radius reached]  The default outer radius is
unphysically large, but can be changed with the \cdCommand{radius} or
\cdCommand{stop thickness}
commands.  If the calculation stops because the outer radius set by one
of these commands is reached, then this line is printed.

\item[ because column dens reached]  a limit to the largest allowed neutral,
ionized, and total hydrogen column densities is set with the \cdCommand{stop column
density}, \cdCommand{stop neutral column density}, or \cdCommand{stop ionized
column density}
commands.  This message will be printed if one of these criteria stops the
calculation.

\item[\dots because lowest Te reached]  The default value of the lowest
temperature allowed is \TEMPSTOPDEFAULT.  This is reasonable when only emission from
warm ionized gas is of interest.  The limit can be changed with the
\cdCommand{stop
temperature} command.  This message is printed if the calculation stops
because the lowest temperature is reached.

\item[\dots because highest Te reached]  The default value of the highest
temperature allowed is \TEMPLIMITHIGH.  The limit can be changed with the
\cdCommand{stop
temperature exceeds} command.  This message is printed if the calculation
stops because the highest allowed temperature is exceeded.

\item[\dots because freeze out fraction]  Nick Abel incorporated the
condensation of molecules onto grain surfaces. Currently CO, H$_2$O, and OH
condensation are treated. The chemistry network will become unstable when
oxygen is highly depleted from the gas phase.  By default the code stops
when 99\% of the oxygen abundance has condensed out of the gas phase.

\item[\dots because NZONE reached]  By default the code will stop after computing
1400 zones.  This can be reset with the \cdCommand{stop zone} command.  This message
is printed if the calculation stops because the limiting number of zones
is reached.  A warning is printed at the end of the calculation since this
was probably not intended.

The default limit to the number of zones can be increased, while retaining
the check that the default limit is not hit, by using the \cdCommand{set nend} command.

\item[\dots because volume too large for this cpu] This indicates that the
 effective volume of the last zone was too large to be represented as a single
 precision floating point number. This happens when $R/R_{\rm in}$ becomes
 really large. Usually this indicates that none of the intended stopping
 criteria were hit, so you should check those. If you really intend for the
 simulation to integrate this far, you should increase the inner radius.

\item[\dots because line ratio reached]  It is possible to set a limit to the
largest value of an emission-line intensity ratio with the \cdCommand{stop
line} command.
This message is printed if the calculation stops because the largest value
of the ratio is reached.

\item[\dots because internal error - DRAD]  An internal logical error caused
this message to be printed. Please report the problem, including the command
lines and the version number of \Cloudy, on the discussion board at the code's
web site, \href{www.nublado.org}.

\item[\dots because initial conditions out of bounds]  The temperature of the
first zone was not within the temperature bounds of the code.  This is
probably due to the incident continuum not being set properly.

\item[\dots because zero electron density]  The electron density fell to zero
because there was no source of ionization.  This is unphysical and usually
occurs because the cloud boundary conditions were not set properly.  Consider
adding at least galactic background cosmic rays with the \cdCommand{cosmic ray
background} command and perhaps the galactic or extragalactic background.

\item[\dots because reason not specified] This is an internal error. Please post
the input and version number on the code's discussion board.
\end{description}

\section{Geometry}
{\setverbatimfontsize{\tiny}
\begin{verbatim}
   The geometry is plane-parallel.
\end{verbatim}
}

The code will next say whether the geometry is plane parallel $(\Delta
  r/r_{\mathrm{o}} <
0.1 )$, a thick shell $(\Delta r/r_{\mathrm{o}} < 3 )$, or spherical  $(\Delta
r/r_{\mathrm{o}}\ge  3 )$, where $r_{\mathrm{o}}$ is
the inner radius and $\Delta r$ is the thickness of the cloud.

\section{Warnings, Cautions, Surprises, and Notes}

{\setverbatimfontsize{\tiny}
\begin{verbatim}
 C-Cloud thicker than smallest Jeans length=3.51e+16cm; stability problems? (smallest Jeans mass=2.58e-01Mo)
  !Magnetic field & cosmic rays both present.  Their interations are not treated.
  !Some input lines contained underscores, these were changed to spaces.
  !Suprathermal collisional ionization of H reached 83.84% of the local H ionization rate.
  !H2 vib deexec heating reached 6.68% of the local heating.
  !Charge transfer ionization of H reached 95.8% of the local H ionization rate.
  !The largest continuum brightness temperature was 4.835e+05K at 1.052e-08 Ryd.
  !Both constant pressure and turbulence makes no physical sense???
  !AGE: Cloud age was not set.  Longest timescale was 8.43e+15 s = 2.67e+08 years.
  !The excitation temp of Lya exceeded the electron temp, largest value was  4.60e+03K (gas temp there was  1.01e+03K, zone 310)
  !Line absorption heating reached 13.45% of the local heating - largest by level 1 line Si 2 34.8m
  !Some infrared fine structure lines are optically thick:  largest tau was  2.05e+03
  !Local grain-gas photoelectric heating rate reached  98.8% of the total.
  !The local H2 photodissociation heating rate reached 12.8% of the total heating.
  !The CMB was not included.  This is added with the CMB command.
  !The fraction of H  in H2    reached 100.0% at some point in the cloud.
  !The fraction of C  in CO     reached 100.0% at some point in the cloud.
  !The fraction of N  in N2     reached 99.7% at some point in the cloud.
  !The fraction of S  in CS     reached 100.0% at some point in the cloud.
  !The fraction of S  in SO     reached 56.9% at some point in the cloud.
  !The fraction of S  in OCS    reached 36.3% at some point in the cloud.
  !The fraction of Cl in HCl    reached 24.5% at some point in the cloud.
  !The fraction of O  in H2Ogrn reached 58.3% at some point in the cloud.
  !A signficant amount of molecules condensed onto grain surfaces.
  !These are the molecular species with "grn" above.
  !The optical depth in the H I 21 cm line is 6.84e-01
  !The optical depth in the 12CO J=1-0 line is 2.79e+05
  !The radiation pressure jumped by 123% at zone 334, from 5.28e-11 to 3.67e-11 to 3.73e-11
  !The H2 density varied by 12.6% between two zones
   Continuum fluorescent production of H-beta was significant.
   Te-ne bounds of Case B lookup table exceeded, H I Case B line intensities set to zero.
   Te-ne bounds of Case B lookup table exceeded, He II Case B line intensities set to zero.
   Destruction of He 2TriS reached 3.4% of the total He0 dest rate at zone 236, 3.4% of that was photoionization.
   Critical density for l-mixing of He I not reached.  More resolved levels are needed for accurate He I line ratios.
   The largest continuum occupation number was 2.911e+08 at 1.052e-08 Ryd.
   The continuum occupation number fell below 1 at 1.944e+04 microns.
   The continuum brightness temperature fell below 10000K at 1.433e+06 microns.
   Ratio of computed diffuse emission to case B reached 2.30168 for iso 1 element 2
   Global grain photoelectric heating of gas was  9.9% of the total.
   Local grain-gas cooling of gas rate reached  90.4% of the total.
   The local H2 cooling rate reached 7.2% of the local cooling.
   Local CO rotation cooling reached 83.0% of the local cooling.
   The Balmer continuum optical depth was  3.68e+03.
   The Balmer continuum stimulated emission correction to optical depths reached  7.64e-02.
   The Paschen continuum optical depth was 1.69e+03.
   The continuum optical depth at the lowest energy considered (1.052e-08 Ryd) was  3.213e+03.
   The optical depth to Rayleigh scattering at 1300A is 4.36e-02
   3 body rec coef outside range 1
   The fraction of Cl in CCl+   reached 0.103% at some point in the cloud.
   The fraction of N  in HNC    reached 0.312% at some point in the cloud.
   The fraction of C  in C2H    reached 2.60% at some point in the cloud.
   The fraction of C  in COgrn  reached 0.379% at some point in the cloud.
\end{verbatim}
}

The next messages fall into four categories: warnings beginning with
``W-''; cautions beginning with ``C-''; surprising results beginning with an
explanation mark (``!''), and notes.

\Cloudy\ does many internal sanity checks to confirm that its range of
validity was not exceeded (\citealp{Ferland2001b}).
Warnings are issued to indicate
that the program has not treated an important process correctly.
For
instance, warnings occur if the temperature was high enough for the electrons
to be relativistic, if the global heating - cooling balance is off by more
than 20\%, or if the code stopped for an unintended reason.
We would like
to hear about warnings - the web site has a discussion board to place
comments.
Cautions are less severe, and indicate that \Cloudy\ is on thin
ice.
Examples are when the optical depths in excited states of hydrogen
change during the last iteration.
Surprises indicate
that, while the physical process has been treated correctly, the result
is surprising.
An example is when induced Compton heating is more that
5 percent of the total Compton heating.
Notes indicate interesting features
about the model, such as maser effects in lines or continua, or if the fine
structure lines are optically thick.
The messages are usually
self-explanatory.

\section{Optional Plot}
{\setverbatimfontsize{\tiny}
\begin{verbatim}
    l-                      Log(nu fnu) os log(nu) "NEW" PARIS MEETING PLANETARY NEBULA
     l    o
     l-         o         o      o ooo   o                     ..............
     l           oooo  o     oo  ooo oooo  ooo     ............             .........
     l-   oo        ooooooooooooooooooooooooooo....                                 .....
     loooooooooooooooo            ........                                              ...
     oo                   .........                                                        ...
     l             ........                                                                  ...
     l-    ........                                                    o                       ..
     .......                                                          oo                 ooooooo ..
     l-                                                              oo                ooo     oooo.
     l                                               o              oo                oo          ooo.
     l-                                                           ooo                o              oo.
     l                                                             o                o                oo.
     l-                                                           oo               oo                 oo.
     l                                              oo       o    o               oo                    o.
     l-                                        oooooooo          o                o                      o.
     l                                        oo      o      ooooo               oo                      oo.
     l-                                                ooooooo                   o                        oo
     l                                                oo                        oo                         oo
     l-                                                                         o                           oo
     l                                                                          o                            o.
     l-                                                                        o                             oo
     l                                                                         o                              oo
     l-                                                                      ooo                               o.
     l                                                                        o                                 o
     l-                                                                 oo    o                                 oo
     l                                                                        o                                  o
     l-                                                                      o                                   oo
     l                                                                   oo  o                                    o
     l-                                                                o     o                                    oo
     l                                                                       o                                     o
     l-                                                                 o   o                                      oo
     l                                                                      o                                       o
     l-                                                                     o                                       oo
     l                                                                     oo                                        o
     l-                                                                    o                                         o.
     l                                                                                                                o
     l-                                                                    o                                          o
     l                                                                     o                                          oo
     l-                                                                    o                                           o
     l                                                                    o                                            o.
     l-                                                                o  o                                             o
     l                                                                    o                                             o
     l-                                                                   o                                             o.
     l                                                                                                                   o
     l-                                                                 oo                                               o
     l                                                                  oo                                               oo
     l-                                                                                                                   o
     l                                                                                                                    o
     l-                                                                                                                   o.
     l                                                                                                                     o
     l-                                                                                                                    o
     l                                                                                                                     o
     l-                                                                                                                    oo
     l                                                                                                                      o
     l-                                       l                                         l                                   o
     -----------------------------------------------------------------------------------------------------------------------o-
      0.1                                     1                                        10

\end{verbatim}
}

If any of the optional plots are requested with a \cdCommand{plot - -}
command then they will appear next.
This option is seldom used today since it is much
easier to create data files with save commands and then use other
software to make plots.

\section{Final Printout}

{\setverbatimfontsize{\tiny}
\begin{verbatim}
                       ********************************> Cloudy 06.01.02 <********************************
                       * title the Orion HII Region / PDR / Molecular cloud with an open geometry        *
                       * #                                                                               *
                       * # commands controlling continuum =========                                      *
               -
               -
               -
               -
                       * assert line luminosity "12CO" 235.4m  -2.80 error 0.15                          *
                       * assert line luminosity "12CO" 215.7m  -2.84 error 0.15                          *
                       * assert nzone < 1400                                                             *
                       * assert itrzn < 24                                                               *
                       * # orion hii pdr pp.in                                                           *
                       * # class hii pdr                                                                 *
                       * # ========================================                                      *
                       *********************************> Log(U): -1.48 <*********************************
                       >>>>>>>>>> Cautions are present.
\end{verbatim}
}

The final printout begins by reprinting the input commands.
The box
surrounding it gives both the version number of \Cloudy\ (at the top) and
the log of the ionization parameter (the ratio of ionizing photon
to hydrogen densities) at the bottom.
{\setverbatimfontsize{\tiny}
\begin{verbatim}
                     Emission Line Spectrum.  Constant  Pressure Model.    Open geometry.  Iteration 2 of 2.
                                                    Intensity (erg/s/cm^2)
\end{verbatim}
}

This line summarizes some properties of the model and output.
The first
part indicates whether the energy in the emission lines is given as the
luminosity case (the energy radiated by a spherical shell covering
$\Omega$ sr [erg s$^{-1}$] where $\Omega/4\pi$ is the covering factor)
or the intensity case (emission
produced by a unit area of gas [erg s$^{-1}$ cm$^{-2}$])).
Which of the two choices
is printed is determined by whether the luminosity of the continuum was
specified as the luminosity radiated by the central object into $4\pi$ sr
or
the intensity $(4\pi J)$ of the incident continuum (erg cm$^{-2}$ s$^{-1}$) at the
illuminated face of the cloud.
If the cloud is spherical and the intensity
case is computed then the emergent emission-line spectrum will be per unit
area in units of the inner radius ro (that is, the total line luminosity
radiated by a shell covering $4\pi$ sr will be the listed intensity
$4\pi J \times 4\pi r_{\mathrm{o}}^2)$.
The second part of this line indicates the density structure (i.e.,
wind, constant density, constant pressure, constant gas pressure, power-law
density distribution, etc).
The next section tells whether the geometry
was open or closed (these are defined in Part I of this document).
The last part indicates the iteration number.
{\setverbatimfontsize{\tiny}
\begin{verbatim}
                                                   Intrinsic line intensities
 general properties..........      H  1 1.500m  -2.042   0.0010      +Col 1.278m  -1.666   0.0024      O  2 833.8A  -1.615   0.0027
 H  1  4861A   0.473   0.3311      H  1 7.458m  -1.094   0.0090      Ca B 1.870m  -1.760   0.0019      O 2r  4651A  -2.135   0.0008
 H  1  1216A   0.740   0.6119      H  1 4.653m  -1.299   0.0056      +Col 1.870m  -1.757   0.0019      O 2r  4341A  -2.413   0.0004
 Inci     0    3.001 111.7791      H  1 3.740m  -1.487   0.0036      Ca B 1.279m  -2.153   0.0008      TOTL  4341A  -2.413   0.0004
 TotH     0    1.727   5.9400      H  1 3.296m  -1.650   0.0025      +Col 1.279m  -2.149   0.0008      TOTL  1665A  -1.363   0.0048
-
 -
 -
 H  1 1.513m  -2.287   0.0006      Ca B 1.869m  -1.283   0.0058      O 2r  2471A  -2.363   0.0005      hyperfine structure.........
 H  1 1.508m  -1.964   0.0012      +Col 1.869m  -1.275   0.0059      O 2r  7323A  -2.536   0.0003      inner shell.................
 H  1 1.504m  -1.999   0.0011      Ca B 1.278m  -1.676   0.0024      O 2r  7332A  -2.657   0.0002
                                                   Emergent line intensities
 H  1  4861A   0.132   1.0000      12CO 517.8m  -3.749   0.0001      12CO 431.5m  -3.474   0.0002      12CO 369.8m  -3.268   0.0004
 13CO 353.6m  -3.709   0.0001      13CO 309.4m  -3.605   0.0002      13CO 275.0m  -3.547   0.0002      13CO 247.5m  -3.544   0.0002
 13CO 225.0m  -3.641   0.0002      H  1  1216A   0.547   2.6028           H  1  1026A  -1.178   0.0490      H  1 972.5A  -1.350   0.0330
-
-
-
 C  3 398.4A  -3.773   0.0001      N  1 980.7A  -3.567   0.0002      N  2 916.3A  -2.513   0.0023      N  2 646.4A  -2.307   0.0036
 N  3 348.7A  -3.763   0.0001      O  1  7950A  -2.869   0.0010      O  2  4188A  -3.756   0.0001      O  2 386.3A  -1.774   0.0124
 O  2 385.7A  -1.955   0.0082      O  3 300.5A  -3.653   0.0002
\end{verbatim}
}

A series of predicted quantities follow.
These are mainly emission-line
intensities although the output also includes other predicted quantities.

Some continua and various indications of contributors to lines and
continua are included.
The Chapter of this document describing observed
quantities (starting on page \pageref{sec:ObservedQuantities})
tells how to convert these into some observed quantities.
Not all
are printed by default.
The discussions on the \cdCommand{print} commands in Part I
tell how to get more or fewer predictions.
This list
of emission lines can also be sorted by wavelength or intensity, and can
be printed as a single column so that they can be entered into a spreadsheet.

The organization and meaning of the different of lines in the printout
is discussed in the Chapter \cdSectionTitle{The Emission Lines}
starting on page \pageref{sec:EmissionLines}.

A list of emission lines with negative relative intensities 
may follow the main block of lines.
These are lines which heat rather than cool the gas (heating
is negative cooling).
This is not a problem but occurs if the line's
collisional de-excitation rate exceeds its collisional excitation rate.
This usually occurs when the line is radiatively excited but collisionally
de-excited.

\subsection{Intrinsic line intensities and luminosities}

There are two blocks of predicted intensities.
The block ``Intrinsic line intensities'' is the intrinsic emission
from the cloud.
The intrinsic
emission includes all processes that affect the line formation and transfer.
This includes collisional processes, fluorescence, line destruction by
background opacities such as dust or the Lyman continuum of hydrogen, and
recombination.
The intrinsic intensities do not include the effects of
absorbers or scatters that do not lie within the line-formation region.

\subsection{Emergent line intensities and luminosities}

The block ``Emergent line intensities'' is the emission observed from
outside the cloud.
In an open geometry the inward part of the line includes
the effects of extinction between the line-forming region and the illuminated
face.
There is an additional contribution due to reflection off the gas
in the outward direction.
The outward part of the line includes the effects
of extinction between the shielded face and the point where the line forms.
In a closed geometry the emergent intensity is the emission escaping to
the outer edge of the slab.
All of this is very geometry dependent.
For an open geometry, in particular, the observed
emission will depend strongly upon the viewing angle.

If the intrinsic and emergent values are very different then it is
important to understand the geometry and what is actually observed.
This
is an indication of the uncertainty if the predictions have a strong
geometry dependence.

The following is an example of the emission-line output.
The first column indicates the species and the second is the wavelength.
``A'' indicates Angstroms and ``m'' indicates microns.
The third column is the log of the intensity [erg cm$^{-2}$ s$^{-1}$] or luminosity [erg s$^{-1}$]
depending on which case was used.
The last column is the intensity of the line relative to the normalization line, H$\beta$ by default.
\begin{verbatim}
Blnd      4725.00A  -17.316    0.1805
Ar 4      4740.12A  -17.639    0.0858
He 2      4859.18A  -17.839    0.0541
H  1      4861.33A  -16.572    1.0000
Fe 7      4893.37A  -18.376    0.0157
He 1      4921.93A  -19.511    0.0012
Fe 7      4942.48A  -18.787    0.0061
O  3      4958.91A  -16.542    1.0722
Fe 6      4967.15A  -17.661    0.0815
Fe 6      4972.48A  -17.239    0.2153
Fe 7      4988.55A  -18.106    0.0293
Pcon      5000.00A    1.373 *********
O  3      5006.84A  -16.067    3.1990
Blnd      5007.00A  -15.942    4.2715
O3bn      5007.00A   -0.164 *********
Fe 6      5097.82A  -19.193    0.0024
\end{verbatim}

\subsection{Air vs vacuum wavelengths}
The emission line wavelengths follow the convention that vacuum wavelengths
are used for $\lambda < 2000$\AA\ and STP air wavelengths are used
for $\lambda \ge 2000$\AA.
The \cdCommand{print line vacuum} command tells the code use use vacuum wavelengths throughout.
The continuum is always reported in vacuum wavelengths to avoid 
a discontinuity at 2000\AA.

\subsection{Some physical properties of the cloud}

{\setverbatimfontsize{\tiny}
\begin{verbatim}
 the Orion HII Region / PDR / Molecular cloud with an open geometry
 Cooling:  HFBc     0 :0.110 HFFc     0 :0.077 TOTL  3727A:0.074 O  3  5007A:0.210 O  3  4959A:0.070 S  3  9532A:0.074
 Heating:  BFH1     0 :0.817 BFHe     0 :0.074 GrGH     0 :0.099
\end{verbatim}
}

\cdCommand{Cooling}:  This line indicates the fraction of the total cooling (defined
here as the energy of the freed photoelectron, see AGN3 Chapter 3) carried
by the indicated emission lines.
The line label is followed by the ratio
of the energy in the line to the total cooling.
This is an important
indication of the fundamental power-losses governing conditions in the model.
The labels used are the same as those in the line array.

\cdCommand{Heating}:  This line indicates the fraction of the total heating produced
by various processes.  The format is the same as the line giving the cooling.
{\setverbatimfontsize{\tiny}
\begin{verbatim}
 IONIZE PARMET:  U(1-) -1.4764  U(4-): -5.0813  U(sp): -4.46  Q(ion):   -4.574  L(ion)-13.505   Q(low): 15.158      L(low)  1.121
\end{verbatim}
}

The line begins with the log of the H ``U(1-)'' and He$^+$ ``U(4$_-$)''
ionization parameters.
The third number ``U(sp)'' is the log of a spherical
ionization parameter often used in spherical geometries,
such as \hii\ regions or planetary nebulae.
It is defined as
\begin{equation}
{U_{sph}} = \frac{{Q\left( {{{\mathrm{H}}^{\mathrm{0}}}} \right)}}{{4\pi
\,R_s^2{n_{\mathrm{H}}}c}}% (408)
\end{equation}
where $R_s$ is the Str\"omgren radius,
defined as the point where the hydrogen
neutral fraction falls to H$^0$/H $=$ 0.5 .
If no ionization front is present
then $U_{sph}$ is evaluated at the outer edge of the cloud.
The next two numbers
are the log of the number of hydrogen-ionizing photons
$(h\nu \ge 1$ Ryd) exiting
the nebula ``Q(ion)'' and the log of the energy
in this continuum ``L(ion)''.
The next two numbers are the equivalent quantities for
non-ionizing $(h\nu < 1$ Ryd) radiation.
These are either per unit area or by a shell covering
$4\pi$~sr.
These have been corrected for the $r^{-2}$ dilution if per unit area,
and so are directly comparable with the numbers given at the start of the
calculation.
{\setverbatimfontsize{\tiny}
\begin{verbatim}
ENERGY BUDGET:  Heat:   1.725  Coolg:   1.725  Error:  0.0%  Rec Lin:   1.462  F-F  H -5.612   P(rad/tot)mx:1.10E-01      R(F Con):1.737e+05
\end{verbatim}
}

This line gives an indication of the energy budget of the nebula.
The
first number ``Heat'' is the log of the total heating (in ergs s$^{-1}$, but
again either into $4\pi$~sr or cm$^{-2}$).
The second number ``Coolg'' is the log
of the total cooling, in the same units.
Cooling is the total energy in
collisionally excited lines and part of the recombination energy,
but \emph{does not} include recombination lines (AGN3 Chapter 3).
The percentage error
in the heating-cooling - cooling match ``Error'' follows.
The next numbers
give ``Rec Lin'', the log of the total luminosity in recombination lines,
``F-F  H'', the log of the amount of energy deposited by
free-free heating,
and ``P(rad/tot)mx'', the largest value of the ratio of
radiation to gas pressures that occurred.
The line ends with \cdCommand{R(F~Con)}, the resolving power in the fine continuum.

{\setverbatimfontsize{\tiny}
\begin{verbatim}
     Col(Heff):      1.031E+25  snd travl time  5.06E+13 sec  Te-low: 1.69E+01  Te-hi: 1.08E+04 G0TH85:3.54E+05   G0DB96:4.54E+05
\end{verbatim}
}

The effective column density ``Col(Heff)'', as defined in the section
in Part 1 on the \cdCommand{stop effective column density} command, is printed.
This is followed by ``snd travl time'',
the sound travel time across the nebula
in seconds.
Constant pressure is only valid if the cloud is static for
times considerably longer than this.
The last two numbers are the lowest
``Te-low'' and highest ``Te-hi'' electron kinetic temperatures found in
the computed structure.
The last numbers ``G0TH85'' and ``GHBD96'' give
the intensity of the ultraviolet radiation field relative to the background
Habing value, as defined by \citet{Tielens1985a} and \citet{Bertoldi1996}.
{\setverbatimfontsize{\tiny}
\begin{verbatim}
  Emiss Measure    n(e)n(p) dl       2.205E+25  n(e)n(He+)dl         1.962E+24  En(e)n(He++) dl      1.402E+21
\end{verbatim}
}

This gives several line-of-sight emission measures.
The definition of
the line of sight emission measure of a species X is
(AGN3 section 5.4)
\begin{equation}
E\left( X \right) = \int {n\left( e \right)} \,\,n\left( X
\right)\,\,f(r)\;dr\quad   [\mathrm{cm}^{-5}]% (409)
\end{equation}
where $f(r)$ is the filling factor.  This is given for H$^+$, He$^+$, and
He$^{2+}$.
{\setverbatimfontsize{\tiny}
\begin{verbatim}
He/Ha:9.61E-02  =   1.01*true  Lthin:1.81E+01  itr/zn: 9.78  H2 itr/zn:  0.00  File Opacity: F MassTot  1.12e32  (gm)
\end{verbatim}
}

This line includes some quantities deduced from the predicted emission-line
spectrum.
The first (``He/Ha'') number is the apparent helium abundance He/H,
measured from the emission-line intensities using techniques described in
AGN3 (Chapter 5);
\begin{equation}
{\left( {\frac{{{\mathrm{He}}}}{{\mathrm{H}}}} \right)_{apparent}} = \frac{{0.739
\times I(5876) + 0.078 \times I(4686)}}{{I({\mathrm{H}}\beta )}}.% (410)
\end{equation}
The intensities of all lines are the total predicted intensities and include
contributions from collisional excitation and radiative transfer effects.
The second number (i.e., ``1.07*true'') is the ratio
of this deduced abundance
to the true abundance.
This provides a simple way to check whether
ionization correction factors, or other effects,
would upset the measurement
of the helium abundance of the model nebula.
This is followed by the longest
wavelength in centimeters ``Lthin'' at which the nebula is optically thin.
Generally the largest FIR opacity source is bremsstrahlung and the number
will be $10^{30}$ if the nebula is optically thin across the full continuum.
The number ``itr/zn'' is the average number of iterations needed to converge
each zone while ``H2 itr/zn'' is the number of iterations per zone required
to converge the large H2 model if it is included.
``MassTot'' gives the
total mass of the computed structure in grams if the inner radius
was specified.
If the inner radius was not specified, and the cloud is plane parallel,
then the mass per unit area [gm~\pscm] is reported.

{\setverbatimfontsize{\tiny}
\begin{verbatim}
   Temps(21 cm)   T(21cm/Ly a)  8.40E+02        T(<nH/Tkin>)  1.04E+03          T(<nH/Tspin>)   1.29E+03          TB21cm       5.32E+02
                  N(H0/Tspin)   5.22E+18        N(OH/Tkin)    2.01E+13
\end{verbatim}
}

This line gives various quantities related to the \hi\ 21 cm line.
``T(21cm/Ly~a)''  gives the temperature deduced from the ratio of the 21
cm to \la\ line optical depths (AGN3 Section 5.5).
The opacity within the
21 cm line is proportional to
$n\left( {{{\mathrm{H}}^0}} \right)\chi /kT$
where $\chi$ is the excitation energy of the line.
``T($<$nH/Tkin$>$)'' gives the harmonic mean  temperature
\begin{equation}
 \left\langle T\right\rangle  = \frac{{\int {Tn\left( {{{\mathrm{H}}^0}} \right)\chi /kT\;dr}
}}{{\int {n\left( {{{\mathrm{H}}^0}} \right)\chi /kT\;dr} }}%   (411)
\end{equation}
where $T$ is the electron or gas kinetic temperature.
``T(nH/Tspin)''  is
the temperature derived from the $n/T_{spin}$ ratio using the 21 cm spin
temperature.
$T_{spin}$ is calculated from the ratio of populations of the
ground fine structure levels, which is computed including the effects of
\la\ scattering.
The spin and kinetic temperatures are often assumed to be
equal although they are not in practice.
The number ``TB21cm`` is an
estimate of the brightness temperature of the 21 cm line as viewed from
the illuminated face of the cloud.
It is the spin temperature at a depth
where the 21 cm line becomes optically thick at line center.

On the next line 
the next two numbers are the H$^0$ and OH column densities
divided by the 21 cm spin temperature in the case of H$^0$
and by the kinetic
temperature in the case of OH.
These ratios are proportional to the optical
depth of a line at radio frequencies.
{\setverbatimfontsize{\tiny}
\begin{verbatim}
   <a>:0.00E+00  erdeFe0.0E+00  Tcompt1.90E+06  Tthr1.19E+13  <Tden>: 2.38E+01  <dens>:2.00E-17 <Mol>:2.31E+00
\end{verbatim}
}

The mean radiative acceleration ``$<$a$>$'' [cm s$^{-2}$] is printed if the geometry
is a wind model and zero otherwise.
This is followed by some time scales.
``erdeFe'' is the time scale, in seconds, to photoerode Fe
\citep{Boyd1987}.
This is 0s if the  $\gamma$-ray flux is zero.
The next gives the Compton
equilibrium timescale ``Tcompt'' and the thermal
cooling timescale ``Tthr'' [s].
The density (gm cm$^{-3}$) weighted mean temperature ``$<$Tden$>$'',
radius-weighted mean density ``$<$dens$>$'' (gm cm$^{-3}$),
and mean molecular weight ``$<$Mol$>$'', follow.
{\setverbatimfontsize{\tiny}
\begin{verbatim}
     Mean Jeans  l(cm)4.19E+16  M(sun)4.36E-01  smallest:     len(cm):3.53E+16  M(sun):2.61E-01 Alf(ox-tran):     0.0000
\end{verbatim}
}
This gives the mean Jeans length ``l(cm)'' (cm) and Jeans mass ``M(sun)''
(in solar units).
This is followed by the smallest Jeans length ``smallest
len(cm)'' and the smallest Jeans mass ``M(sun)'' which occurred in the
calculation.
The last quantity ``Alf(ox-tran)'' is the spectral index
$\alpha_{ox}$,
defined as in the header,
but for the transmitted continuum (attenuated
incident continuum plus emitted continuum produced by the cloud).
{\setverbatimfontsize{\tiny}
\begin{verbatim}
 Rion:1.001E+17  Dist:6.17E+21  Diam:6.689E+00
\end{verbatim}
}
This gives the radius of the hydrogen ionization front (in cm), or the outer radius in case
the hydrogen ionization front is never reached. This is followed by the distance (in cm), or
zero in case the \cdCommand{distance} command was not used. Finally the angular diameter of
the ionized region is given (in arcsec), or zero if the distance was not set. The angular diameter
is defined as {\tt 2*Rion/Dist}.
{\setverbatimfontsize{\tiny}
\begin{verbatim}
Hatom level 26  NHtopoff:  22  HatomType: add  HInducImp  F  He tot level: 63  He2 level:   26 ExecTime 2233.72
\end{verbatim}
}

This line gives the number of levels of the model hydrogen atom, the
``topoff'' level, above which the remainder of the recombination coefficient
is added, the type of top off used for this calculation, and the number
of levels used for the atomic helium.
The last number on the line is the
execution wall-clock time in seconds.

{\setverbatimfontsize{\tiny}
\begin{verbatim}
 ConvrgError(%)  <eden>  0.075  MaxEden  0.543  <H-C>   0.20  Max(H-C)    0.50  <Press>   0.042 MaxPres  2.535
  Continuity(%)  chng Te   3.9  elec den   5.8  n(H2)   12.8  n(CO)        8.5
\end{verbatim}
}

The first line gives some estimates of the errors that occurred in several
quantities that the code converges.
A pair of numbers gives the mean and
largest percentage errors for the electron density, the heating-cooling
balance, and the pressure.
The second line gives the percentage changes
that occurred from one zone to the next for the temperature, the electron
density, and the \htwo\ and CO densities.

\subsection{Average temperatures and densities}
{\setverbatimfontsize{\tiny}
\begin{verbatim}
                                                        Averaged Quantities
             Te      Te(Ne)   Te(NeNp)  Te(NeHe+)Te(NeHe2+) Te(NeO+)  Te(NeO2+)  Te(H2)     N(H)     Ne(O2+)   Ne(Np)
 Radius:  6.86e+02  9.19e+03  9.27e+03  9.18e+03  9.09e+03  9.65e+03  8.86e+03  2.28e+01  8.30e+06  1.43e+04  1.50e+04
 \end{verbatim}
}

 This begins with several temperature and density averages, over either
radius or volume.
The volume averages are only printed if the
\cdCommand{sphere} command
is entered.
The quantity which is printed is indicated at the top of each
column.
The quantity being averaged is the first part of the label, and
the weighting used is indicated by the quantity in parenthesis.
For instance ``Te(NeO2+)'' is the electron temperature averaged
with respect to the product
of the electron and O$^{2+}$ densities.
{\setverbatimfontsize{\tiny}
\begin{verbatim}
Peimbert T(OIIIr)9.08E+03 T(Bac)0.00E+00 T(Hth)0.00E+00 t2(Hstrc) 6.01e-03 T(O3-BAC)0.00E+00 t2(O3-BC) 0.00e+00 t2(O3str) 1.81e-03
 Be careful: grains exist.  This spectrum was not corrected for reddening before analysis.
\end{verbatim}
}

This series of quantities deal with temperature fluctuations ($t^2$, \citealp{Peimbert1967}; AGN3 section 5.11) .
The code analyzes the predicted emission line
and continuum spectrum using the same steps that Manuel outlined in this
paper.
The code does not attempt to correct the predicted emission-line
intensities for collisional suppression or reddening, so this line is only
printed if the density is below the density set with the
\cdCommand{set tsqden} command---the default is $10^7$~cm$^{-3}$.
This code does not attempt to deredden the
spectrum: a caution is printed if grains are present.

The nature of temperature fluctuations is, in my option, the biggest
open question in nebular astrophysics.
Theory (\Cloudy\ too) predicts that
they should be very small because of the steep dependence of the cooling
function on the temperature, while some observations indicate a very large
value of $t^2$ (see \citealp{Liu1995}, \citealp{KingdonFerland1995}, and \citealp{Ferland2003}).
If something is missing from our current understanding of the energy source
of photoionized nebulae then the entire nebular abundance scale (for both
the Milky Way and the extragalactic nebulae) is in error by as much as 0.5
dex.

Two fundamentally different $t^2$s enter here---the ``structural''
$t^2$ and the ``observational'' $t^2$.
The structural value comes from the computed
ionization and thermal structure of the nebula while the observational value
comes from an analysis of the predicted emission-line spectrum following
the methods outlined in Peimbert's 1967 paper.

The structural $t^2$ for the H$^+$ ion is defined as
\begin{equation}
{t^2}\left( {{{\mathrm{H}}^ + }} \right) = \left\langle {{{\left[
{\frac{{T\left( r \right) - \left\langle T \right\rangle }}{{\left\langle
T \right\rangle }}} \right]}^2}} \right\rangle  = \frac{{\int {{{\left[
{T\left( r \right) - \left\langle T \right\rangle } \right]}^2}{n_e}n\left(
{{{\mathrm{H}}^ + }} \right)f\left( r \right)\,dV} }}{{{{\left\langle T
\right\rangle }^2}\int {{n_e}n\left( {{{\mathrm{H}}^ + }} \right)f\left( r
\right)\,dV} }}% (412)
\end{equation}
where $<T>$ is the density-volume weighted mean temperature
\begin{equation}
\left\langle T \right\rangle  = \frac{{\int {T\left( r \right){n_e}n\left(
{{{\mathrm{H}}^ + }} \right)f\left( r \right)\,dV} }}{{\int {{n_e}n\left(
{{{\mathrm{H}}^ + }} \right)f\left( r \right)\,dV} }}.% (413)
\end{equation}
This quantity is given in the averaged quantities block as the column
``Te(NeNp).''

The observational $t^2$---related quantities are the following:  ``T(OIIIr)''
is the electron temperature indicated by the predicted [OIII] 5007/4363
ratio in the low-density limit.
This number is meaningless for densities
near or above the critical density of the [O III] lines.
``T(Bac)'' is
the hydrogen temperature resulting from the predicted Balmer jump and
H$\beta$.
``T(Hth)'' is the same but for optically thin Balmer continuum and case
B H$\beta$ emission.
``t2(Hstrc)'' is the structural \hii\ $t^2$.
The entries
``T(O3-BAC)'' and t2(O3-BC)'' are the mean temperature and $t^2$ resulting
from the standard analysis of the [\oiii] and \hi\ spectra (\citealp{Peimbert1967}).
Finally ``t2(O3str)'' is the structural $t^2$ over the O2$^+$ zone.
Only the
structural $t^2$s are meaningful for high densities.
This section was developed
in association with Jim Kingdon, and
\citet{KingdonFerland1995} provide
more details.
{\setverbatimfontsize{\tiny}

\subsection{Average grain properties}
\begin{verbatim}
Average Grain Properties (over radius):
        gra-orion01* gra-orion02* gra-orion03* gra-orion04* gra-orion05* gra-orion06* gra-orion07  gra-orion08  gra-orion09  gra-orion10
    nd:      0            1            2            3            4            5            6            7            8            9
 <Tgr>: 3.659e+01    3.592e+01    3.522e+01    3.450e+01    3.377e+01    3.306e+01    3.236e+01    3.170e+01    3.106e+01    3.045e+01
 <Vel>: 4.177e+02    4.780e+02    5.483e+02    6.279e+02    7.223e+02    8.327e+02    9.572e+02    1.095e+03    1.241e+03    1.396e+03
 <Pot>: 3.072e-01    2.905e-01    2.736e-01    2.575e-01    2.423e-01    2.281e-01    2.151e-01    2.036e-01    1.932e-01    1.842e-01
 <D/G>: 1.202e-04    1.337e-04    1.486e-04    1.653e-04    1.837e-04    2.043e-04    2.271e-04    2.525e-04    2.808e-04    3.122e-04

        sil-orion01* sil-orion02* sil-orion03* sil-orion04* sil-orion05  sil-orion06  sil-orion07  sil-orion08  sil-orion09  sil-orion10
    nd:     10           11           12           13           14           15           16           17           18           19
 <Tgr>: 3.303e+01    3.261e+01    3.216e+01    3.173e+01    3.130e+01    3.090e+01    3.051e+01    3.015e+01    2.981e+01    2.950e+01
 <Vel>: 1.093e+03    1.237e+03    1.385e+03    1.532e+03    1.676e+03    1.817e+03    1.953e+03    2.086e+03    2.215e+03    2.342e+03
 <Pot>: 1.693e-01    1.598e-01    1.509e-01    1.427e-01    1.352e-01    1.283e-01    1.220e-01    1.163e-01    1.109e-01    1.062e-01
 <D/G>: 2.031e-04    2.259e-04    2.511e-04    2.792e-04    3.104e-04    3.451e-04    3.837e-04    4.267e-04    4.744e-04    5.274e-04

        pah-bt9401 * pah-bt9402 * pah-bt9403 * pah-bt9404 * pah-bt9405 * pah-bt9406 * pah-bt9407 * pah-bt9408 * pah-bt9409 * pah-bt9410 *
    nd:     20           21           22           23           24           25           26           27           28           29
 <Tgr>: 4.620e+01    4.634e+01    4.650e+01    4.662e+01    4.675e+01    4.683e+01    4.691e+01    4.697e+01    4.702e+01    4.706e+01
 <Vel>: 9.565e+00    1.100e+01    1.255e+01    1.426e+01    1.617e+01    1.819e+01    2.029e+01    2.261e+01    2.516e+01    2.779e+01
 <Pot>: 2.697e+00    2.424e+00    2.181e+00    1.964e+00    1.768e+00    1.594e+00    1.442e+00    1.307e+00    1.188e+00    1.084e+00
 <D/G>: 9.119e-08    9.556e-08    1.002e-07    1.050e-07    1.100e-07    1.153e-07    1.208e-07    1.266e-07    1.327e-07    1.391e-07
 Dust to gas ratio (by mass): 5.457e-03, A(V)/N(H)(pnt):5.543e-22, (ext):4.024e-22, R:3.647e+00 AV(ext):5.152e+03 (pnt):7.097e+03
\end{verbatim}
}

The next lines give some information concerning grains if these were
included in the calculation.
These lines give the mean temperature, drift
velocity, and potential, for all of the grain populations included in the
calculation.
An asterisk will appear to the right of the name of any
species with quantum heating included.
In this case the mean temperature
is weighted by $T^4$.

The last line gives some information related to the grain abundance and
optical properties.
The first number is the dust to gas ratio by mass.
The next two are the total visual extinction per unit hydrogen column density
for a point and extended source.
These are different because of the
different effects of forward scattering (AGN3 section 7.6).
Next comes
the ratio of total to selective extinction.
The line ends with \Av\ for both
an extended and point source.

\subsection{Optical depths}
{\setverbatimfontsize{\tiny}
\begin{verbatim}
 Contin Optical Depths: COMP: 1.07e-03     H-: 1.90e-04     R(1300): 4.53e-02  H2+: 1.12e-06  Bra:1.55E+02
                        Pa: 7.89e+02    Ba: 3.98e+03      Hb: 5.31e+03    La: 1.07e+04     1r:1.151E+05  1.8:3.79E+07 4.:4.382E+06
  Line Optical Depths: 10830:  8.12e+02  3889: 3.49e+01    5876: 1.91e-04 7065: 2.62e-05  2.06m: 1.29e-02  21c: 5.29e-01
\end{verbatim}
}

The first two lines give the continuum optical depths at various energies.
These are the total optical depths, including the correction for stimulated
emission, and will be negative if maser action occurs. All opacity sources
are included.
The labels, and their interpretation, are as follows. COMP
is the Thomson scattering optical depth.
``H-'' is the optical depth at the
wavelength where the negative hydrogen ion has its greatest maximum cross
section.
``R(1300)'' is the optical depth due to Rayleigh scattering by
H$^0$ at 1300\AA .
``H2$^+$'' is the optical depth at the dissociation threshold
of the molecular hydrogen ion.
``Bra'' is the optical depth at the
wavelength of the Brackett $\alpha$ transition (5-4).

The next line gives total continuous optical depths at the energies of
various hydrogen and helium ionization edges and 
mean line optical depths\footnote{Line center optical depths were reported through version C10.
Mean line optical depths are now given, although the code works
with line center optical depths internally.}. These are evaluated
at the energies of the Paschen $\alpha$, Balmer $\alpha$ and $\beta$, and
\la\ lines, and the
ionization edges of hydrogen, atomic helium, and the helium ion.

The third line gives optical depths of some \hei\ lines.  These are
computed with a full model of the He$^0$ atom (\citealp{Porter2005}).

{\setverbatimfontsize{\tiny}
\begin{verbatim}
Old, new H  1 continuum optical depths:
     1 1.13e+05     2 3.68e+03     3 1.69e+03     4 8.24e+02     5 4.03e+02     6 2.12e+02     7 1.24e+02     8 8.47e+01
     9 7.60e+01    10 3.02e+02    11 2.24e+02    12 1.05e+02    13 9.80e+01    14 1.31e+02    15 1.16e+02    16 9.21e+01
    17 7.61e+01    18 6.54e+01    19 5.68e+01    20 4.88e+01    21 4.19e+01    22 3.55e+01    23 3.00e+01    24 2.55e+01
    25 2.17e+01
     1 1.13e+05     2 3.61e+03     3 1.66e+03     4 8.08e+02     5 3.95e+02     6 2.08e+02     7 1.22e+02     8 8.31e+01
     9 7.46e+01    10 2.96e+02    11 2.20e+02    12 1.03e+02    13 9.61e+01    14 1.28e+02    15 1.14e+02    16 9.03e+01
    17 7.46e+01    18 6.41e+01    19 5.57e+01    20 4.78e+01    21 4.11e+01    22 3.48e+01    23 2.95e+01    24 2.50e+01
    25 2.12e+01

 Old, new H  1 line optical depths:
  2- 1 2.59e+09  3- 2 1.58e-01  4- 3 9.14e-08  5- 4 5.27e-08  6- 5-1.66e-08  7- 6-1.39e-07  8- 7-5.70e-07
  9- 8-1.24e-06 10- 9-1.79e-06 11-10-3.40e-06 12-11-4.47e-06 13-12-4.36e-06 14-13-3.91e-06 15-14 9.70e-06 16-15-6.04e-05
 17-16-1.08e-04 18-17-1.88e-04 19-18-3.44e-04 20-19-6.90e-04 21-20 1.37e-02 22-21 2.07e-02 23-22 6.52e-02 24-23 8.91e-02
 25-24 1.18e-01
  2- 1 1.16e+09  3- 2 7.08e-02  4- 3 6.02e-08  5- 4 2.67e-08  6- 5-2.15e-08  7- 6-1.51e-07  8- 7-5.76e-07
  9- 8-1.23e-06 10- 9-1.78e-06 11-10-3.35e-06 12-11-4.41e-06 13-12-4.30e-06 14-13-3.81e-06 15-14 4.83e-06 16-15-5.97e-05
 17-16-1.07e-04 18-17-1.85e-04 19-18-3.39e-04 20-19-6.74e-04 21-20 6.72e-03 22-21 1.01e-02 23-22 3.20e-02 24-23 4.38e-02
 25-24 5.79e-02

 Old, new He 2 continuum optical depths:
     1 4.47e+06     2 1.12e+05     3 4.60e+03     4 3.68e+03     5 2.54e+03     6 1.69e+03     7 1.17e+03     8 8.24e+02
     9 5.77e+02    10 4.03e+02    11 2.88e+02    12 2.12e+02    13 1.61e+02    14 1.24e+02    15 1.01e+02    16 8.47e+01
    17 7.13e+01    18 7.60e+01    19 1.43e+02    20 3.02e+02    21 3.11e+02    22 2.24e+02    23 1.54e+02    24 1.05e+02
    25 8.75e+01
     1 4.38e+06     2 1.13e+05     3 4.52e+03     4 3.61e+03     5 2.50e+03     6 1.66e+03     7 1.15e+03     8 8.08e+02
     9 5.66e+02    10 3.95e+02    11 2.82e+02    12 2.08e+02    13 1.57e+02    14 1.22e+02    15 9.93e+01    16 8.31e+01
    17 7.00e+01    18 7.46e+01    19 1.40e+02    20 2.96e+02    21 3.05e+02    22 2.20e+02    23 1.51e+02    24 1.03e+02
    25 8.58e+01

 Old, new He 2 line optical depths:
  2- 1 7.60e+06  3- 2 6.42e-07  4- 3 2.62e-13  5- 4-1.24e-12  6- 5-5.54e-12  7- 6-1.45e-11  8- 7-2.98e-11
  9- 8-5.02e-11 10- 9-7.50e-11 11-10-9.49e-11 12-11-9.32e-11 13-12-3.54e-11 14-13 2.08e-10 15-14 1.07e-09 16-15-1.24e-09
 17-16-2.30e-09 18-17-3.80e-09 19-18-6.67e-09 20-19-1.30e-08 21-20 2.38e-07 22-21 3.59e-07 23-22 1.20e-06 24-23 1.63e-06
 25-24 2.15e-06
  2- 1 4.03e+06  3- 2 3.21e-07  4- 3 1.31e-13  5- 4-1.23e-12  6- 5-5.53e-12  7- 6-1.44e-11  8- 7-2.97e-11
  9- 8-5.01e-11 10- 9-7.48e-11 11-10-9.47e-11 12-11-9.29e-11 13-12-3.54e-11 14-13 1.04e-10 15-14 5.32e-10 16-15-1.23e-09
 17-16-2.30e-09 18-17-3.80e-09 19-18-6.66e-09 20-19-1.30e-08 21-20 1.18e-07 22-21 1.78e-07 23-22 5.96e-07 24-23 8.09e-07
 25-24 1.07e-06

 Old He Is optical depths:   1 3.86e+07   2 4.04e+03   3 3.94e+03   4 3.80e+03   5 3.80e+03   6 3.80e+03   7 3.67e+03   8 2.16e+03
 New He Is optical depths:   1 3.79e+07   2 3.96e+03   3 3.86e+03   4 3.72e+03   5 3.72e+03   6 3.72e+03   7 3.60e+03   8 2.12e+03
          Old He Is Lines: 2-1 9.19e+11 3-2 6.78e-09
          New He Is Lines: 2-1 4.51e+11 3-2 3.44e-09
\end{verbatim}
}

Hydrogen and helium optical depths in continua and
$\alpha(n \to n-1)$ transitions follow.
The first block of lines are the optical depths assumed at the
start of the present iteration and the second block gives the newly computed
total optical depths.
Negative optical depths indicate maser action.
For
each of the pairs the first block is the optical depth at thresholds of
levels of hydrogen.
The optical depths in the $\alpha(n \to n-1)$ transitions of
hydrogen or helium follow.
{\setverbatimfontsize{\tiny}
\begin{verbatim}
   Line Optical Depths: 10830: 3.52e+01  3889: 1.51e+00    5876: 2.11e-08 7065: 2.88e-09  2.06m: 2.86e-05  21c: 5.35e-05
   H  1 1215A 5.15e+05 H  1 1025A 8.26e+04 H  1  972A 2.87e+04 H  1  949A 1.35e+04 H  1  937A 7.45e+03 H  1  930A 4.56e+03
   H  1  926A 3.00e+03 H  1  923A 2.08e+03 H  1  920A 1.51e+03 H  1  919A 1.12e+03 H  1  918A 8.62e+02 H  1  917A 6.75e+02
   H  1  916A 5.39e+02 H  1  915A 4.37e+02 H  1  915A 3.59e+02 H  1  914A 2.99e+02 H  1  914A 2.52e+02 H  1  914A 2.14e+02
   H  1  914A 1.83e+02 H  1  913A 1.58e+02 H  1  913A 1.37e+02 H  1  913A 1.20e+02 H  1  913A 1.06e+02 H  1  913A 9.34e+01
   --------
\end{verbatim}
}

 Line optical depths are not normally printed, but will be if the
  \cdCommand{print line optical depths} command is entered.
  
  \subsection{Column densities}
{\setverbatimfontsize{\tiny}
\begin{verbatim}
                                                     Log10 Column density (cm^-2)
   Htot  : 25.107   HII   : 21.167   HI    : 21.830   H-    : 12.687   H2g   : 24.806   H2*   : 16.421   H2+   : 11.203   HeH+  : 11.485
   H3+   : 13.820
   CH    : 17.799   CH+   : 12.136   OH    : 14.764   OH+   : 11.763   O2    : 17.909   CO    : 21.489   CO+   : 10.423   H2O   : 17.459
   H2O+  : 11.049   O2+   : 11.735   H3O+  : 14.429   CH2+  : 11.869   CH2   : 17.534   HCO+  : 15.394   CH3+  : 13.437   SiH2+ : 11.835
   SiH   : 15.523   HOSi+ : 14.069   SiO   : 17.953   SiO+  :  9.222   CH3   : 20.296   CH4   : 20.207   CH4+  : 11.132   CH5+  : 13.562
   N2    : 20.644   N2+   : 12.546   NO    : 15.954   NO+   : 12.873   S2    :  4.242   S2+   :  0.565   OCN   : 12.648   OCN+  :  9.954
   NH    : 14.634   NH+   :  9.266   NH2   : 16.135   NH2+  :  9.125   NH3   : 16.966   NH3+  : 12.225   NH4+  : 13.243   CN    : 18.325
   CN+   :  7.814   HCN   : 18.107   HCN+  :  8.175   HNO   :  8.826   HNO+  : 10.277   HS    : 14.673   HS+   : 13.458   CS    : 19.929
   CS+   : 10.371   NO2   : 10.136   NO2+  :  4.356   NS    : 13.418   NS+   : 12.254   SO    : 19.303   SO+   : 14.554   SiN   : 13.857
   SiN+  : 10.516   N2O   : 10.072   HCS+  : 15.996   OCS   : 19.077   OCS+  : 14.312   C2    : 20.023   C2+   : 11.133   CCl   : 14.010
   ClO   :  2.369   HCl+  : 11.968   HCl   : 17.061   H2Cl+ : 12.002   CCl+  : 12.258   H2CCl+: 10.100   ClO+  : -1.084   HNC   : 18.151
   HCNH+ : 14.265   C2H   : 19.590   C2H+  : 10.670   C2H2  : 16.355   C2H2+ : 15.324   C3H   :  4.558   C3H+  : -0.848   C2H3+ : 11.612
   C3    :  3.013   C3+   : -6.584   COgrn : 17.793   H2Ogrn: 21.202   OHgrn :  8.957


               1      2      3      4      5      6      7      8      9     10     11     12     13     14     15     16     17

 Hydrogen   21.830 21.167 25.107 (H2)                Log10 Column density (cm^-2)
 Helium     24.085 20.117 17.105
 Carbon     17.950 18.459 17.563 14.609 13.216  3.870
 Nitrogen   19.044 16.241 16.933 14.065 11.478  5.692 -1.143
 Oxygen     20.608 17.460 17.476 14.327 11.487  6.302  2.306
 Neon       20.885 16.857 16.224 13.324 10.890  5.202  0.097
 Magnesium  19.584 16.541 15.619 12.650 10.152  4.929  0.953
 Silicon    19.701 16.932 15.711 14.515 13.010 10.127  3.809 -0.668
 Sulphur    19.036 17.151 16.136 14.583 12.601 11.685 10.014  2.678 -2.437
 Chlorine   18.066 14.800 14.129 11.924 10.392  9.305  4.190  1.698 -3.512
 Argon      19.584 14.672 15.606 13.401 11.627 10.513  5.326  2.378 -0.441
 Iron       19.584 16.655 15.236 15.417 12.498 11.847  9.292  5.178  1.303 -3.214
 Exc state    He1* 14.756   C+1[2] 18.248   C[1] 17.606   C[2] 17.569   C[3] 17.067   O[1] 20.607   O[2] 18.042   O[3] 17.426
              Si2* 16.187   C+2[2] 12.885   C+2[3]  8.857   C+2[3] 13.149
\end{verbatim}
}

This lists the column densities (cm$^{-2}$) of some atoms,
ions, and molecules.
The first number ``Htot'' is the total column density of hydrogen in all
forms (including atoms, ions, and molecules).
The following two numbers
are the column densities in H$^+$ and H$^0$.
The last four numbers are column
densities in four ions and molecules (\hminus, H$_{2^{g}}$, H$_2^*$,
H$_2^+$, H$_3^+$, and HeH$^+$).
The remaining lines give column densities in various molecules.
Molecules with names ending in ``grn'' are solids that have
condensed onto grain surfaces.

The next block gives column densities in atoms and ions of the heavy
elements.
For hydrogen the last number is the \htwo\ column density.
Column
densities within certain excited states of the heavy elements,
listed in Table \ref{tab:cdColm_labels} on page \pageref{tab:cdColm_labels}, are also printed.
The label gives the
element, ionization stage, and level within the ground term.
These upper
levels are photoexcited by the so-called ``level 2 lines''.
Their pumping
will not be included and excited state column densities will not be predicted
if the level 2 lines are disabled with the \cdCommand{no level 2 lines} command.

\subsection{Mean ionization and temperature}
{\setverbatimfontsize{\tiny}
\begin{verbatim}
               1      2      3      4      5      6      7      8      9     10     11     12     13     14     15     16

 Hydrogen   -3.278 -3.940 -0.000 (H2)                 Log10 Mean Ionisation (over radius)
 Helium     -0.000 -3.968 -6.980
 Carbon     -3.635 -3.126 -4.021 -6.975 -8.368-17.715
 Nitrogen   -1.908 -4.712 -4.019 -6.887 -9.474-15.261-22.096
 Oxygen     -1.101 -4.249 -4.234 -7.382-10.222-15.407-19.403
 Neon       -0.000 -4.028 -4.662 -7.561 -9.996-15.684-20.788
 Magnesium  -0.000 -3.044 -3.966 -6.934 -9.432-14.656-18.631
 Silicon    -0.008 -2.778 -3.999 -5.194 -6.700 -9.582-15.900-20.377
 Sulphur    -1.071 -2.956 -3.971 -5.524 -7.507 -8.423-10.093-17.429-22.545
 Chlorine   -0.041 -3.307 -3.978 -6.184 -7.716 -8.802-13.918-16.410-21.619
 Argon      -0.000 -4.912 -3.979 -6.183 -7.957 -9.071-14.259-17.206-20.025
 Iron       -0.001 -2.929 -4.348 -4.167 -7.086 -7.737-10.292-14.406-18.282-22.799

               1      2      3      4      5      6      7      8      9     10     11     12     13     14     15     16

 Hydrogen   -1.292 -0.090 -0.864 (H2)         Log10 Mean Ionisation (over radius*electron density)
 Helium     -0.622 -0.119 -3.265
 Carbon     -2.105 -0.549 -0.176 -3.188 -4.552-14.000
 Nitrogen   -0.770 -0.843 -0.174 -3.101 -5.664-11.546-18.381
 Oxygen     -0.813 -0.378 -0.405 -3.646 -6.501-11.692-15.688
 Neon       -0.730 -0.171 -0.855 -3.828 -6.272-11.969-17.073
 Magnesium  -1.269 -0.741 -0.117 -3.184 -5.702-10.941-14.916
 Silicon    -1.764 -0.643 -0.148 -1.382 -2.932 -5.840-12.185-16.662
 Sulphur    -2.018 -0.696 -0.122 -1.719 -3.712 -4.627 -6.332-13.714-18.829
 Chlorine   -0.856 -0.966 -0.130 -2.390 -3.910 -4.994-10.203-12.695-17.904
 Argon      -0.732 -1.160 -0.130 -2.396 -4.146 -5.259-10.544-13.491-16.310
 Iron       -1.472 -0.778 -0.478 -0.331 -3.339 -3.971 -6.562-10.691-14.567-19.083
\end{verbatim}
}

The next blocks of output give the log of the mean ionization, averaged
over volume, area (if the model is spherical), and over radius.
The
volume-averaged ionization fraction for ion $i$ of element $a$ is given by
\begin{equation}
\left\langle \frac{n_a^i}{n_a}\right\rangle_{vol}
= \frac{\int n_a^i f (r)\,r^2\,dr}{\int n_af(r)\,r^2\, dr}.% (414)
\end{equation}
the area average by
\begin{equation}
\left\langle \frac{n_a^i}{n_a}\right\rangle_{area}
= \frac{\int n_a^i f (r)\,r\,dr}{\int n_af(r)\,r\, dr}.
\end{equation}
and the radius average by
\begin{equation}
\left\langle \frac{n_a^i}{n_a}\right\rangle_{rad}
= \frac{\int n_a^i f (r)\,dr}{\int n_af(r)\, dr}.% (415)
\end{equation}
Where $n_a$ is the total gas-phase density and
$n_a^i$ is the density in ionization
stage~$i$.
Similar blocks of information will give the mean ionization
weighted by electron density and volume, area, or radius.

{\setverbatimfontsize{\tiny}
\begin{verbatim}
               1      2      3      4      5      6      7      8      9     10     11     12

 Hydrogen    3.127  3.965  1.357 (H2)                 Log10 Mean Temperature (over radius)
 Helium      1.372  3.961  3.937
 Carbon      1.485  3.097  3.957  3.945  3.948  3.972
 Nitrogen    1.949  4.004  3.956  3.945  3.951  3.972  3.972
 Oxygen      1.586  3.983  3.946  3.956  3.963  3.972  3.972
 Neon        1.370  3.968  3.944  3.956  3.961  3.972  3.972
 Magnesium   1.357  2.953  3.963  3.951  3.957  3.972  3.972
 Silicon     1.354  2.714  3.963  3.945  3.946  3.950  3.972  3.972
 Sulphur     1.463  2.867  3.964  3.943  3.946  3.946  3.944  3.972  3.972
 Chlorine    1.348  3.198  3.962  3.944  3.947  3.947  3.972  3.972  3.972
 Argon       1.370  3.922  3.960  3.945  3.948  3.948  3.972  3.972  3.972
 Iron        1.357  2.822  3.984  3.950  3.953  3.947  3.957  3.972  3.972  3.972

               1      2      3      4      5      6      7      8      9     10     11     12

 Hydrogen    3.223  3.967  2.109 (H2)    Log10 Mean Temperature (over radius*electron density)
 Helium      3.430  3.963  3.959
 Carbon      2.234  3.742  3.958  3.945  3.949  3.972
 Nitrogen    2.664  4.006  3.958  3.945  3.952  3.972  3.972
 Oxygen      2.826  3.985  3.947  3.956  3.963  3.972  3.972
 Neon        2.675  3.971  3.944  3.956  3.961  3.972  3.972
 Magnesium   2.041  3.498  3.965  3.951  3.957  3.972  3.972
 Silicon     1.496  3.474  3.965  3.945  3.946  3.950  3.972  3.972
 Sulphur     1.721  3.369  3.966  3.944  3.946  3.946  3.943  3.972  3.972
 Chlorine    2.140  3.842  3.964  3.944  3.947  3.948  3.972  3.972  3.972
 Argon       2.631  4.016  3.962  3.945  3.949  3.949  3.972  3.972  3.972
 Iron        1.803  3.140  3.986  3.952  3.953  3.947  3.956  3.972  3.972  3.972
\end{verbatim}
}

The next blocks give the mean temperature weighted by volume, area, and radius.
These are followed by the mean temperature weighted by volume, area, or radius and
electron density. In all cases the volume and area averages will only be shown in
spherical models.

\subsection{Convergence statistics}
{\setverbatimfontsize{\tiny}
\begin{verbatim}
---------------Convergence statistics---------------
      1.24 mean iterations/state convergence
      2.67 mean cx acceleration loops/iteration
      1.02 mean iso convergence loops/ion solve
         1 mean steps/chemistry solve
         2 mean step length searches/chemistry step
----------------------------------------------------
\end{verbatim}
}

\subsection{Final report}
{\setverbatimfontsize{\tiny}
\begin{verbatim}
 Cloudy ends: 196 zones,   1 iteration,  1 caution.  ExecTime(s) 86.97
 [Stop in maincl, Cloudy exited OK]
\end{verbatim}
}

The code ends by listing the number of zones and iterations that were
performed and the number of warnings and cautions that occurred.
Next comes
the elapsed wall-clock time [s].
The last line will say ``Cloudy exited
OK'' if the calculation is successful.

%     [gjf1]Add C+ when revising for C08
