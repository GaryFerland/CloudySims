\chapter{CLOUDY AS A STANDALONE PROGRAM}
% !TEX root = hazy2.tex

\Cloudy\ can be used to run a single model, to create large grids of
calculations, or to compute a number of simulations while
varying one or more parameters to match an observed spectrum.
This Chapter describes how to use \Cloudy\ as a self-contained
program to read in the parameters for
the simulation and compute the result.
The next Chapter discusses the case in which the code is called
as a subroutine of another larger code.

\section{Running a single model with a shell script}

The code reads from an input file and can create a large number of
output files.  The latter include both the main output (described in
Chapter~\ref{sec:output}, \cdSectionTitle{\refname{sec:output}}) and
the ancillary ``save'' files (described in the
Chapter~\ref{Hazy1-sec:ControllingOutput} of Hazy 1,
\cdSectionTitle{\refname{Hazy1-sec:ControllingOutput}}).

It is a good idea to follow a naming convention for these files.
The
convention I use is the style ``\cdFilename{basename.type}''
where \cdFilename{basename} explains
the astrophysical context (for instance, ``quasar'' or ``IGM'')
and \cdFilename{type}
gives the type of information in the file.
For instance, a model of a
planetary nebula may have the base name ``\cdFilename{pn\_halo}'',
the input script might be
\cdFilename{pn\_halo.in}, the output would be in
\cdFilename{pn\_halo.out}, and the file created
by the \cdCommand{save overview} command might be
\cdFilename{pn\_halo.ovr}.
Then all of these files
could be located with a simple ``\cdCommand{ls pn}.*''
and all overview files with a
``\cdCommand{ls *.ovr}'' on a Linux system.

The \cdFilename{pn.in} file contains the input commands
that tell the program what
to do.
A typical example might be the following:
\begin{verbatim}
# log of the hydrogen density (cm^-3)
hden 4
# log of the inner radius (cm)
radius 17
# black body temperature and total luminosity
black body 1e5 K, luminosity 38
save overview "pn.ovr"
\end{verbatim}
\Cloudy\ stops reading the input steam when it reaches
either an empty line or the end of file.
Nothing special is needed at the end of the input file.

I have a shell script named \cdFilename{run} which is in my
``\cdFilename{bin}'' directory, which
I include on my path.
The shell script \cdFilename{run} consists of the following on
Linux:
\begin{verbatim}
cloudy.exe < $1.in > $1.out
\end{verbatim}
Under Windows it would have the name \cdFilename{run.bat} and would contain the following
\begin{verbatim}
cloudy.exe < %1.in > %1.out
\end{verbatim}
If \cdFilename{run} is executed by typing
\begin{verbatim}
run pn
\end{verbatim}
it would read the input stream in \cdFilename{pn.in} and create
an output file called \cdFilename{pn.out}.

\section{Running a single model from the command line}
\label{sec:RunCommandLine}

The code also has two command-line options that will accomplish the same thing
as the shell script described in the previous section. If you create an
executable called \cdFilename{cloudy.exe}, then the command
\begin{verbatim}
cloudy.exe model.in
\end{verbatim}
will read input from \cdFilename{model.in}, write output to
\cdFilename{model.out}, and add the prefix \cdFilename{model} to all the save
files.  This can also be written
\begin{verbatim}
cloudy.exe -p model
\end{verbatim}
Alternatively, you can use the command
\begin{verbatim}
cloudy.exe -r model
\end{verbatim}
which does the same thing, except that it will {\em not} add the prefix
\cdFilename{model} to all the save files (mnemonic: \cdFilename{-p} will
Prefix Save file names, while \cdFilename{-r} will only Redirect input
and output).

By typing the command
\begin{verbatim}
cloudy.exe -h
\end{verbatim}
you can a list of all the command line flags that \Cloudy\ supports.
There are two additional flags, \cdCommand{-a} and \cdCommand{-g},
that are used for debugging the code and internally in grid runs.
They should never be used in normal situations.

\section{Running grids of simulations or optimizing a simulation}

The greatest insight is gained by creating grids of simulations
which vary one or more of the input parameters to show how
various predictions change, or to optimize the agreement between the
predicted and observed values.
One example, the predicted \civ\ $lambda 1549$ equivalent width as a
function of the flux of ionizing photons and cloud density,
is shown in Figure \ref{fig:CIV_EW}.
See \citet{BaldwinEtAl95} for more details.

Such grids of simulations can be made parallel on distributed memory
parallel machines because each point in the grid is independent of the
other points.

\subsection{Grids of simulations}

The \cdCommand{grid} command is described in
Chapter~\ref{Hazy1-sec:CommandGrid}
\cdSectionTitle{\refname{Hazy1-sec:CommandGrid}} of Hazy 1.

\subsection{Optimizing a simulation}

The \cdCommand{optimize} command is described in
Chapter~\ref{Hazy1-sec:CommandOptimize}
\cdSectionTitle{\refname{Hazy1-sec:CommandOptimize}} of Hazy 1.


\section{Parallel processing with MPI}
\label{sec:ParallelMPI}

Peter van Hoof created a wrapper for the 
\cdCommand{grid} and \cdCommand{optimize} commands.
\Cloudy\ will automatically run these commands using
all available processors if the code is compiled with 
an MPI aware compiler and run on an appropriate computer.
