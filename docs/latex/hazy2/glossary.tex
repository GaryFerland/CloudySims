\chapter{GLOSSARY OF SYMBOLS}
% !TEX root = hazy2.tex

As far as possible, the notation used by \Hazy\ follows
standard texts (AGN3; \citealp{Mihalas1978}).
This is a summary of some of the symbols used.

The fundamental constants used by the code are from the
\href{http://physics.nist.gov/cuu/Constants/index.html}{
CODATA recommended values}.
Constants are contained in the header \cdFilename{physconst.h}.

\begin{tabbing}
  \textbf{Symbol}\quad \= \textbf{Description}\hspace{10em} \= \textbf{Units}\hspace{8em} \=
  \textbf{Notes}\quad\quad \\
  $a$ \>  Stefan radiation density \>   erg cm$^{-3}$ K$^{-4}$ \>
$7.56464\times 10^{-15}$\\
$a$ \>  damping constant \>   ---\\
$a_o$ \>  Bohr radius \>  $\hbar
/{m_e}{c^2}$ cm \>  $0.5291775\times 10^{-8}/Z$\\
$A_{rad}$ \>   radiative acceleration \>  cm s$^{-2}$\\
$A_{ul}$ \>  radiative rate from level $u$ to $l$ \>  s$^{-1}$\\
 $b_n$ \>  departure coefficient \> ---\\
$B$ \>  magnetic field \>  esu \>  \\
$B_\nu$ \>  Planck function \>  erg cm$^{-2}$ s$^{-1}$ Hz$^{-1}$ sr$^{-1}$\\
$c$ \>   speed of
light \>  cm s$^{-1}$ \>   $2.997925\times 10^{10}$\\
$C$ \>  collisional rate \>  s$^{-1}$\\
$C_{ul}$ \>  line collision rate \>  s$^{-1}$\\
$D_{ul}$ \>  line destruction probability \> ---\\
$f$ \>  oscillator strength\> ---\\
$f(r)$ \>  filling factor \> --- \> \\
$f_\nu$ \>  flux density \>  erg cm$^{-2}$ s$^{-1}$ Hz$^{-1}$\\
$F_\nu$ \>  flux density \>  erg s$^{-1}$
Hz$^{-1}$\\
$g$ \>  grain asymmetry factor \> ---\\
$g_i$ \>  statistical weight \> ---\\
$g_{III}$ \>  T
aver free-free gaunt factor \> ---\\
$g_\odot$ \>  Solar surface gravity \>  cm
s$^{-2}$ \>  $2.74\times 10^4$\\
$G$ \> gravitational constant \>  dyne cm$^2$ g$^{-2}$ \>  $6.673\times 10^{-8}$\\
$G$ \>  energy gains,
heating \>  erg cm$^{-3}$ s$^{-1}$\\
$h$ \>  Planck's constant \>  erg s \>  $6.6262\times 10^{-27}$\\
$\hbar $ \>  Planck's constant \>  erg s \>   $1.0546\times 10^{-27}$\\
$I$ \>  integrated intensity \>  erg s$^{-1}$ sr$^{-1}$ Hz$^{-1}$\\
$I_n$ \>  ionization potential of level n \>   erg Ryd\\
$I_\nu$ \>  intensity \>  erg s$^{-1}$ sr$^{-1}$ Hz$^{-1}$\\
$J$ \>  integrated mean intensity \>  erg s$^{-1}$ sr$^{-1}$\\
$J_\nu$ \>  mean intensity \>  erg s$^{-1}$ sr$^{-1}$ Hz$^{-1}$\\
$k$ \>  Boltzmann constant \>  eV deg$^{-1}$ \> $8.6171\times 10^{-5}$\\
$k$ \>  Boltzmann constant \>  erg deg$^{-1}$ \>  $1.38062\times 10^{-16}$\\
$L_\odot$ \>  luminosity of sun \>  erg s$^{-1}$ \>  $3.826\times10^{33}$\\
$m_A$ \>  mass  of atom A \>  gm\\
$m_{\mathrm{AMU}}$ \>  atomic mass unit \>  gm \>  $1.6605402\times10^{-24}$\\
$m_e$ \>  electron mass \>  gm \>  $9.10956\times 10^{-28}$\\
$m_e$ c$^2$ \>  electron energy \>  Ryd \>  $3.75584\times10^4$\\
$m_p$ \>  proton mass \>  gm \>  $1.6726231\times 10^{-24}$\\
$M_J$ \>  Jeans' mass \>  gm\\
$M_\odot$ \>  mass of the sun \>  gm \>  $1.989\times 10^{33}$\\
$M_\oplus$ \>  mass of the Earth \>  gm \>  $5.977\times 10^{27}$\\
$n_e$ \>  electron
density \>  cm$^{-3}$\\
$n_j$ \>  population of level $j$ \>  cm$^{-3}$\\
$n_p$ \>  proton density \> cm$^{-3}$\\
$n(H)$ \>  total
H density, all forms \>  cm$^{-3}$\\
$n(x)$ \>  density of species x \>  cm$^{-3}$\\
$n(cr)$ \>  cosmic ray
density \>  cm$^{-3}$\\
$n$ \>  atom's level \> --- \\
$n(\mathrm{H}_{tot})$ \>  H density, all forms \>  cm$^{-3}$\\
$N$(x) \>  column density
of species x \>  cm$^{-2}$\\
$N(\mathrm{H}_{tot})$ \>  total H col den, all forms \>   cm$^{-2}$\\
$N_{\mathrm{eff}}$ \>  effective H
column density \>  cm$^{-2}$\\
$ P*$(x) \>  LTE relative population \>  cm$^3$\\
$P_{gas}$ \>  gas pressure \>  dyn
cm$^{-2}$\\
$P_{lines}$ \>  line radiation pressure \> dyn cm$^{-2}$\\
$P_{tot}$ \>  total pressure \>  dyn cm$^{-2}$\\
$P_{ul}$ \>  line
escape probability \> ---\\
$P_{\tau x}(n)$ \>  continuum escape prob \> ---\\
pc \>  parsec \>  cm \>  $3.085678\times 10^{18}$\\
$q_{ij}$ \>  line collisional rate coefficient \>  cm$^3$ s$^{-1}$\\
$q_n$ \>
collisional rate coefficient \>  cm$^3$ s$^{-1}$\\
$q_e$ \>  electron charge \>   esu \>
$4.80325\times 10^{-10}$\\
$Q_{\mathrm{abs}}$ \>  grain absorption efficiency \> ---\\
$Q$(H) \>  hydrogen ionizing photons \>
s$^{-1}$\\
$r$ \>  radius \>  cm\\
$r_{l,u}$ \>  rate \>  s$^{-1}$\\
$R_0$ \>  inner radius \>  cm\\
$R$ \>  total to selective extinction \> ---\\
$R_{\mathrm{H}}$ \>  Rydberg unit for H \> ---\\
$R_\infty$ \>  Rydberg unit for inf mass \> ---\\
$R_{\mathrm{AU}}$ \>  radius of Earth's orbit \>  cm \>  $1.4959\times 10^{13}$\\
$R_+$ \>  radius of the
Earth \>  cm \>  $6.378\times10^{18}$\\
$R_\odot$ \>  radius of the sun \>  cm \>  $6.9599\times 10^{10}$\\
$T_e$ \>  electron temperature \>
cm$^{-3}$\\
$T_{eff}(\odot)$ \>  Sun's effective temperature \>  K \>  $5770 \K$\\
$T_{exc}$ \>  excitation
temperature \>  K \> \\
$T_{color}$ \>  color temperature \>  K\\
$T_{low}$ \>  lowest temp allowed \>  K \>   \TEMPLIMITLOW\\
$T_{u}$ \>  energy density temperature \>  K\\
$u$ \>  energy density \>  erg cm$^{-3}$\\
$U_g$ \>  grain
potential \>  volt\\
$u$ \>  velocity (mean or projected) \>  cm s$^{-1}$\\
$\bar u$ \>  mean particle speed \>  cm s$^{-1}$\\
$u_{Dop}$ \>  Doppler velocity \>  cm s$^{-1}$\\
$u_{exp}$ \>  expansion
velocity \>  cm s$^{-1}$\\
$u_{th}$ \>  thermal velocity \>  cm s$^{-1}$\\
$u_{turb}$ \>  turbulent velocity \>  cm s$^{-1}$\\
$V_g$ \>  grain potential \>  eV\\
$V_n$ \>  grain work function \>  eV\\
$W$ \>  geometric dilution factor \> ---\\
$x$ \>  relative shift from line center \> ---\\
$X_c$ \>  continuous to total opacity \> ---\\
$\hat Y$ \>
grain photoelectric yield \> ---\\
year \>  \>  s \>  $3.156\times 10^7$\\
$z$ \>  redshift \> ---\\
$Z$ \> nuclear
charge \> ---\\
$\alpha$ \>  Fine structure constant \>  $q_e^2/( {\hbar c} )$ \>
1/137.036\\
$\alpha(n, T)$ \>  recombination coefficient \>  cm$^3$ s$^{-1}$\\
$ \bar \alpha (n,T)$ \>
effec recomb coefficient \>  cm$^3$ s$^{-1}$\\
$\alpha_\nu$ \>  continuous abs cross section \>
cm$^2$\\
$\alpha_{lu}$ \>   line absorption cross section \>  cm$^2$\\
$\alpha_\beta$ \>  Case B recomb rate coef \>  cm$^3$
s$^{-1}$\\
$\beta$ \>  recombination cooling coef \>  cm$^3$ s$^{-1}$\\
$\eta_\nu$ \>  photon occupation number \> ---\\
$\delta r$ \>  zone thickness \>  cm\\
$\Delta r$ \>  depth into cloud \>  cm\\
$\gamma_u,l$ \>  continuum pumping probability \> ---\\
$\Gamma_n$ \>  photoionization rate \>  s$^{-1}$\\
$\Gamma$ \>  reciprocal
lifetime of up level \>  s$^{-1}$\\
$\Gamma_{OTS}$ \>  OTS photoionization rate \>   s$^{-1}$\\
$\kappa$ \>  absorption opacity \>
cm$^{-1}$\\
$\kappa_{lu}$ \>  line absorption opacity \>  cm$^{-1}$\\
$\kappa_s$ \>  continuous scattering opacity \>  cm$^{-1}$\\
$\kappa_\nu$ \>  continuous absorption opacity \>  cm$^{-1}$\\
$\lambda_J$ \>  Jeans' length \>   cm\\
$\Lambda$ \>  energy loss,
cooling \>  erg cm$^{-3}$ s$^{-1}$\\
$\mu$ \>  mean molecular weight \> ---\\
$\Omega$ \>  energy-specific collision strength \> --- \\
$\Omega$ \>  shell coverage \>  sr\\
$\Omega/4\pi$ \>  covering factor \> ---\\
$\Phi$(H) \>  flux of ionizing
photons \>   cm$^{-2}$ s$^{-1}$\\
$\phi\nu$ \>  photon flux density \>  cm$^{-2}$ s$^{-1}$ Ryd$^{-1}$\\
$\phi_{OTS}$ \>  flux of OTS photons \>
cm$^{-2}$ s$^{-1}$\\
$\rho$ \>  mass density \>  gm cm$^{-3}$\\
$\pi a_o^2$ \>  area of first Bohr
orbit \>  cm$^2$ \>  $87.9737\times10^{-18}$\\
 \> Classical electron radius \> $q_e^2/( {{m_e}{c^2}}
)$
 cm \>  $2.818\times10^{-13}$\\
$\sigma_T$ \>  Thomson cross section \>  $8\pi /3 \times {\left[ {q_e^2/(
{{m_e}{c^2}})} \right]^2}$ \>
 $6.6524\times 10^{-25}$\\
 \>  \> cm$^2$\\
$\sigma_\nu$ \>  scattering cross section \>   cm$^2$\\
$\sigma_{\mathrm{Ray}}$ \>  Rayleigh scat cross section \>  cm$^2$\\
$\sum$ \>  projected grain area \>  cm$^2$\\
$\tau$ \>  optical depth \> ---\\
$\tau_{abs}$ \>
absorption optical depth \> ---\\
$\tau_{scat}$ \>  scattering optical depth \> ---\\
$\tau_{u,l}$ \>  line
optical depth \> ---\\
$\Upsilon$ \>  thermal averaged collision strength\> --- \\
$\nu$ \>  frequency \>  Hz\\
$\nu_{\mathrm{Ryd}}$ \>
frequency \>  Ryd\\
$\delta_\nu$ \>  line width \> Hz\\
$\delta\nu_{\mathrm{Dop}}$ \>  Doppler width \>  Hz\\
$\chi_s$ \>  h$\nu$/kT \> ---\\
\end{tabbing}
