\chapter{LINE DETAILS }
% !TEX root = hazy2.tex

\section{Overview }

The effects of optical depths, continuum pumping, collisions, and
destruction by background opacity,
are computed for \emph{all} permitted and
intercombination lines.
The cooling is usually distributed among many lines
in high-density models, and these lines are usually optically thick.   This
section describes the methods and data structures used within the code to
accomplish this.

\section{Line Boltzmann factors }

The Boltzmann factor $h\nu/kT$ for a line with a known wavelength or
energy is given by Table \ref{tab:LineBoltzmann}.
The table lists the ratio $h\nu/k$ for various units of
the line energy.
Vacuum, not air, wavelengths, must be used for all
quantities involving wavelengths.

\begin{table}
\centering
\caption{\label{tab:LineBoltzmann}Line Boltzmann Factors }
\begin{tabular}{ll}
\hline
Line Energy Units& $h\nu/k$ (K)\\
\hline
 Angstroms& 1.43877(+8)/$\lambda$(\AA)\\
microns& 1.43877(+4)/$\lambda(\mu)$\\
 wavenumbers& 1.43877$\times \sigma$\\
 Rydbergs& 1.5788866(+5) $\times$E\\
\hline
\end{tabular}
\end{table}

\section{Air vs vacuum wavelengths }

The convention across physics and astronomy is to give line wavelengths
in vacuum for $\lambda\le 2000$~\AA\ and in air for $\lambda > 2000$~\AA.
There is no choice---if you observe visible light with
HST the wavelengths must be quoted in air.

Air wavelengths are smaller than vacuum wavelengths because the wavefronts
are crushed as then enter the denser medium with its higher index of
refraction.
The frequency is unchanged.

\section{The line escape probability functions }

\subsection{Escape probability formalism vs exact radiative transfer }

Radiative transport effects are approximated with the escape probability
formalism (EPF).  This includes line pumping by the incident continuum,
photon destruction by collisional deactivation or by continuous opacities,
and line overlap in special cases.  This section describes how the escape
probability is related to the net radiative bracket, the formally correct
term in the transfer equation.

The full balance equation for radiative losses and gains for the upper
level of a two level atom is given by
\begin{equation}
{n_u}{A_{ul}} + {n_u}{B_{ul}}\bar J - {n_l}{B_{lu}}\bar J \equiv
{n_u}{A_{ul}}{\rho _{ul}} \approx {n_u}{A_{ul}}{P_{ul}}% (1)
\end{equation}
where $A$ and $B$ are the Einstein coefficients, $\bar J$
is the mean intensity averaged over the line,
and $\rho_{\mathrm{ul}}$ is the net radiative bracket, defined as
\begin{equation}
{\rho _{ul}} \equiv 1 - \bar J/S% (2)
\end{equation}
where $S$ is the line source function.
The essence of the EPF is to replace
$\rho_{\mathrm{ul}}$ with the escape probability $P_{ul}$ on the argument that the difference
between $J$ and $S$ is due to photons leaking away from the region.  (\citealp{Elitzur1983}; Sec 2.6) shows that this is exact
if $S$ is constant across the
line-forming region.
In the code $\rho_{\mathrm{ul}}$ is replaced with $P_{ul}$.

\subsection{Redistribution functions }

At low densities, line scattering for a two-level atom is coherent in
the atom's reference frame and the line profile function is described by
the incomplete redistribution function.
At high densities the Stark effect
can broaden the line.
When the radiation density is high, scattering within
excited states can inhibit the broadening of resonance lines such as
L$\beta$ (line
interlocking), destroying the coherence of the scattering process.
In these
cases complete redistribution in a Doppler core more closely describes the
scattering process.

\Cloudy\ uses several escape probability functions to take these processes
into account.
Strong resonance lines are treated with partial redistribution
with a Voigt profile.
Subordinate lines are treated with complete
redistribution in a Doppler core.

\subsection{Incomplete redistribution }

Incomplete redistribution is assumed for resonance transitions such as
C~IV $\lambda$1549 and the \la\ transitions of hydrogen and helium.  Two studies of
line formation using this approximation are those of \citet{Bonihala1979}
and \citet{Hummer1980}.
Both studies suggest escape probabilities of the form
\begin{equation}
{P_l}(\tau ) = {\left\{ {1 + b\left( \tau  \right)\,\tau } \right\}^{
- 1}}% (3)
\end{equation}
but there is substantial disagreement in the form and value of the factor
$b(\tau)$, sometimes by more than a factor of 2.
(This is after due allowance
for the different definitions of line opacities in the two papers.)  \Cloudy\
uses the \citet{Hummer1980} results for \hi, \hei, and \heii\ \la\ and
strong resonance lines such as C~IV $\lambda$1549.
Their tabulated values were fitted
by interpolation.

\subsection{Damping constant }

The damping constant $a$ is given by
\begin{equation}
a = \frac{\Gamma }{{4\pi \;\Delta {\nu _D}}} = \frac{{{\lambda _{cm}}\sum
A }}{{4\pi \,{u_{Dop}}}} = \frac{{{\lambda _{cm}}7.958 \times {{10}^{ -
2}}\sum {A\,} }}{{{u_{Dop}}}} = \frac{{{\lambda _{\mu m}}7.958 \times {{10}^{
- 6}}\sum {A\,} }}{{{u_{Dop}}}}% (4)
\end{equation}
where $\Gamma$ is the inverse lifetime of the level
(the sum of the $A$'s from the
upper level), $\Delta {\nu _D}$
is the Doppler width in frequency units \citep{Mihalas1978},
$\lambda_{cm}$ and $\lambda_{\mu m}$ are
the wavelengths in cm and microns respectively,
and $u_{Dop}$ is the Doppler width in cm~s$^{-1}$.
The ratio $\Gamma \lambda /4\pi $
is stored in the line structures and the $a$'s are evaluated
using this ratio
and the current Doppler width.

\subsection{Background opacity and Destruction probability }

The ratio of continuous to total opacity is $X_c$ parameterized as
\begin{equation}
{X_c} = \frac{{\sum {{\kappa _c}\;{n_c}} }}{{{\kappa _l}{n_l} + \sum
{{\kappa _c}\;{n_c}} }}% (5)
\end{equation}
where the $\kappa_l$'s are the line center absorption opacities and the $n$'s the number
of absorbers.

\subsection{Complete redistribution }

Lines arising from excited states (hydrogen Balmer, Paschen, etc.) and
Lyman lines with $n_u > 2$ are treated assuming complete redistribution in
a Doppler core (i.e., the damping constant $a$ is assumed to be zero).  This
assumption can be changed with the \cdCommand{atom redistribution} command.  If the
total optical depth of the slab is $T$, then the escape probability at a
depth $\tau$
from the illuminated face is given by;
\begin{equation}
{P_{u,l}}(\tau ,\,T,\,{X_c}) = \left[ {1 - {X_c}F({X_c})}
\right]\frac{1}{2}\left[ {{K_2}(\tau ,\,{X_c}) + {K_2}(T - \tau ,\,{X_c})}
\right]\quad ,% (6)
\end{equation}
and the destruction probability is
\begin{equation}
{D_{u,l}}({X_c}) = {X_c}F({X_c}).% (7)
\end{equation}
The function is
\begin{equation}
F({X_c}) = \int_{ - \infty }^\infty  {\frac{{\varphi (x)}}{{{X_c} + \varphi
(x)}}\;dx},% (8)
\end{equation}
where in these expressions (and in this part of the code) the
\emph{mean opacity is used},
and $\varphi(x)\approx \pi^{1/2} \exp(-x^2)$ is the Voigt
function.
$F(X_c)$ is interpolated
from the tables presented by Hummer (1968).
The function
\begin{equation}
{K_2}(\tau ,\,{X_c}) \equiv \frac{1}{{1 - {X_c}F({X_c})}}\int_{ - \infty
}^\infty  {\frac{{{\varphi ^2}(x)}}{{{X_c} + \varphi (x)}}{E_2}\left[ {\left(
{{X_c} + \varphi (x)} \right)\tau } \right]\;d\tau }
\end{equation}
is evaluated numerically.

The complete redistribution escape probabilities are corrected for
finite damping parameters by including a separate wing escape
probability.  The resulting total escape probabilities have been
tested against numerical integrations of the exact formulae in
\cite{Avrett1966,Hummer1982b} -- note the different definitions of
$K_2$ in these papers -- verified by comparison by the tabulations and
fits in these papers.  The overall fit to the escape probability is in
error by at most 25\% in the range $a=10^{-3}\mbox{--}10^3$, with the
largest error at $\tau\simeq 1$, and correct asymptotic behaviour in
the low $\tau$ and high $\tau$ limits.  This level of accuracy is
thought to be sufficient, given the other approximations inherent in
the escape probability method.

\subsection{Masing lines }

A line mases when its optical depth is negative.
The escape probability is (\citealp{Elitzur1992}; p 32, see also
\citealp{1990ApJ...363..628E,1990ApJ...363..638E})
\begin{equation}
{P_{u,l}} = \frac{{1 - \exp \left( { - \tau } \right)}}{\tau }.% (10)
\end{equation}
The code will generate a comment if strong maser action occurs for any
transition.

\subsection{Stark broadening }

Distant collisions with charged particles broaden the upper levels of
lines, and in the limit of very high densities this will make the scattering
process completely non-coherent even for L$\alpha $ (i.e., complete redistribution
obtains).
Cloudy closely follows the treatment of \citet{Puetter1981} in treating
Stark broadening.

\subsection{Net escape probability }

A total escape probability $P_{l, tot}$, given by
\begin{equation}
{P_{u,l}} = \min \left( {{P_{inc}} + {P_{Stark}}\;,\;{P_{com}}} \right),% (11)
\end{equation}
is defined for transitions described by incomplete redistribution.  The
escape probabilities are those for incomplete, Stark, and complete
redistribution respectively.  The total effective escape probability is
not allowed to exceed the complete redistribution value for
$\tau > a^{-1}$.

If $\tau$ is the optical depth in the direction towards the source of ionizing
radiation and $T$ is the total optical depth computed in a previous iteration,
then the escape probability entering the balance equations is
\begin{equation}
{P_{u,l}}\left( {\tau ,T} \right) = \left\{ {{P_{u,l}}\left( \tau  \right)
+ {P_{u,l}}\left( {T - \tau } \right)} \right\}/2.
\end{equation}
In general the total optical depth $T$ is only known after the first iteration,
so more than one iteration must be performed when radiative transfer is
important.

\section{Optical depths and the geometry }

The terms open and closed geometry are defined in a section in Part I.
The treatment of transfer in these two limits is described here.

\subsection{Open geometry }

This is the default.  During the first iteration the line optical depth
is defined using only optical depths accumulated in the inward direction.
This optical depth is initialized to a very small number at the start of
the calculation.
At the end of the first iteration the total optical depth
is set to the optical depth accumulated in the inward direction.
At the
end of subsequent iterations the total optical depth is defined as a mean
of the new and old inward optical depths.

\subsection{Closed geometry overview }

Continuum photons are assumed to interact with gas fully covering the
continuum source.  At the end of the first iteration the total continuum
optical depths are set equal to twice the computed optical depths, and the
inner optical depths reset to the computed optical depths.
The same recipe
is followed on subsequent iterations, except that means of old and newly
computed optical depths are used.

\cdCommand{Closed expanding geometry}  This is the default if the
\cdCommand{sphere} command
is entered.  In this case it is assumed that line photons do not interact
with lines on the ``other'' side of the expanding spherical nebula.  The
treatment of line optical depths is entirely analogous to that described
for an open geometry, since the presence of the distant material has no
effect on line transfer.

\cdCommand{Closed static geometry}  This is assumed if the \cdCommand{sphere static} command
is entered.
In this case line photons from all parts of the spherical shell
do interact.
As a result the optical depth scale is poorly defined on the
first iteration, and more than one iteration is required.
On second and
later iterations the total line optical depth is set to twice the optical
depth of the computed structure, and the optical depth at the illuminated
face of the shell is set to half of this.
The optical depth scale is only
reliably defined after at least a second iteration.

\subsection{Wind }

The model is a large velocity gradient $(v\propto R$ Sobolev approximation) wind.
This is described further in Part 2 of this document.

\section{Collision strengths }

I have tried to follow the Opacity Project notation throughout this
document \citep{Lanzafame1993}.
The energy-specific collision strength
$\Omega_{lu}$ for a transition between upper and lower levels $u$ and $l$ is related to
the excitation cross section $Q_{lu}$~by
\begin{equation}
{Q_{lu}} = \frac{{\pi {\Omega _{lu}}}}{{g_l \, k_{lu}^2}}
\quad [\mathrm{cm}^2]
\end{equation}
where $k_{lu}^2$  is the wavenumber of the collision energy.
If the collisions are with
thermal electrons having a Maxwellian velocity distribution $f(u)$ and velocity
$u$ then the rate coefficient $q_{lu}$ is given by
\begin{equation}
{q_{lu}} = \int_0^\infty  {f\left( u \right)} u{Q_{lu}}\,du = \frac{{2{\pi
^{1/2}}{\hbar ^2}}}{{{g_l}{m_e}}}{a_o}\left( {\frac{{{R_\infty }}}{{kT}}}
\right){\Upsilon _{lu}}\exp \left( { - \frac{{{E_{lu}}}}{{kT}}} \right)\sqrt
{\frac{{2kT}}{{{m_e}}}}\quad
[\mathrm{cm}^3 \mathrm{s}^{-1}].
\end{equation}
$E_{ul}$ is the transition energy in Rydbergs, $a_o$ is the Bohr radius,
\begin{equation}
{a_o} = \frac{{{\hbar ^2}}}{{{m_e}q_e^2}} = 0.529177249 \times {10^{ -
8}}\;{\mathrm{cm}}
\end{equation}
and ${R_\infty }$ is the Rydberg energy.
Then the thermally-averaged collision strength is given by
\begin{equation}
{\Upsilon _{lu}} = \int_0^\infty  {{\Omega _{lu}}\exp \left( { -
\frac{\varepsilon }{{kT}}} \right)\;d\left( {\frac{\varepsilon }{{kT}}}
\right)}.
\end{equation}
The rate coefficient for collisional de-excitation is then given by
\begin{equation}
{q_{ul}} = \frac{\Upsilon }{{{g_u}\sqrt {{T_e}} }}{\left( {\frac{{2\pi
}}{k}} \right)^{1/2}}\frac{{{\hbar ^2}}}{{m_e^{3/2}}} = \frac{{\Upsilon
\,8.6291 \times {{10}^{ - 6}}}}{{{g_u}\sqrt {{T_e}} }}
\quad  [\mathrm{cm}^3 \mathrm{s}^{-1}].
\end{equation}
The rate coefficient for excitation follows from detailed balance:
\begin{equation}
{q_{lu}} = {q_{ul}}\frac{{{g_u}}}{{{g_l}}}\exp \left( { - \chi } \right)
= \frac{{\Upsilon \,8.6291 \times {{10}^{ - 6}}}}{{{g_l}\sqrt {{T_e}} }}\exp
\left( { - \chi } \right)\quad  [\mathrm{cm}^3 \mathrm{s}^{-1}].
\end{equation}

\section{Born approximation }

The Born approximation is valid for energies much larger than the
excitation energy of the transition.
The energy specific collision strength
is given by \citet{Bethe1930} as
\begin{equation}
{\Omega _{lu}} \approx \frac{{4{g_l}{f_{lu}}}}{{{E_{lu}}}}\ln \left(
{\frac{{4\varepsilon }}{{{E_{lu}}}}} \right)
\end{equation}
where $f_{l,u}$ is the absorption oscillator strength of the permitted transition.

\section{The g-bar approximation }

The g-bar or \citet{VanRegemorter1962} approximation relates the collision
strength to the transition probability $A_{ul}$ and wavelength $\lambda$ (in \micron).
Here, the collision strength for the downward transition
${\Upsilon _{ul}}$
is approximately given by
\begin{equation}
\begin{array}{ccl}
 {\Upsilon _{u,l}}& =& \frac{{2\pi }}{{\sqrt 3
}}\frac{{{m^2}{e^2}}}{{{h^3}}}\lambda _{\mu m}^3{10^{ -
12}}{g_u}{A_{u,l}}\bar g \\
& \approx& 2.388 \times {10^{ - 6}}\lambda _{\mu m}^3{g_u}{A_{u,l}}\bar g \\
& \approx& 159{\lambda _{\mu m}}{g_l}{f_{abs}}\bar g \\
 \end{array}
\end{equation}
where $g_u$ and $g_l$ are the statistical weights of the
upper and lower levels
and $f_{abs}$ is the absorption oscillator strength.
For energies of interest
in astrophysical plasmas, where $kT<h\nu$, $\bar g$
is approximately given by
\begin{equation}
\bar g \approx \left\{ {\begin{array}{*{20}{c}}
   \hfill {0.2;} & \hfill {{\mathrm{positive}}\;{\mathrm{ions}}} \\
   \hfill {\left( {{\mathrm{kT/h}}\nu } \right)/10;}\& \hfill {{\mathrm{neutrals}}}
\\
\end{array}} \right.
\end{equation}
(van Regemorter 1962).
These approximations are generally accurate to better
than 1 dex.
\citet{Gaetz1983} give improved forms of the approximation.

\section{The critical density }

The critical density is defined as the density at which the radiative
de-excitation rate $A_{ul}\, P_{ul}$ (where $A$ is the transition
probability and $P$
is the escape probability) equals the collisional de-excitation rate
$q_{ul}n_e$.
Setting
\begin{equation}
{A_{ul}}{P_{ul}} = {C_{ul}} = {q_{ul}}{n_e} = \Upsilon \frac{{8.629 \times
{{10}^{ - 6}}}}{{{g_u}\sqrt {{T_e}} }}{n_e}\quad
  [\mathrm{s}^{-1}]
\end{equation}
where $\Upsilon$ is the thermally averaged collision strength,
the critical density
is given by
\begin{equation}
{n_{crit}} \sim \frac{{{A_{ul}}{P_{ul}}{g_u}\sqrt {{T_e}} }}{{\Upsilon
\,8.629 \times {{10}^{ - 6}}}}\quad
  [\mathrm{cm}^{-3}].
\end{equation}
For an optically allowed transition, in which the g-bar approximation may
apply, this density is approximately given by
\begin{equation}
{n_{crit}} = \frac{{4.8 \times {{10}^{10}}\sqrt {{T_e}} }}{{\lambda _{\mu
m}^3\bar g}}\quad
[\mathrm{cm}^{-3}].
\end{equation}

\section{Line thermalization length }

Line radiative transfer will affect the thermal equilibrium of the gas
when the collision time scale approaches an effective lifetime $\tau\sim
A_{ul} /n_{scat} )^{-1}$, where $A_{ul}$ is the transition probability and
$n_{scat}$ is the number
of scatterings a line photon undergoes before escape.
For permitted metal
lines (which often have optical depths  $\sim10^4 - 10^6$) line thermalization
becomes important at densities $n_e >10^{15} / \tau 10^{10} \mathrm{cm}^{-3}$.  These effects are
important for hydrogen at considerably lower densities due to its greater
abundance.  Additionally, continuum transfer affects the ionization and
thermal equilibrium of the gas at all densities.

\section{Averaging levels into terms  }

\subsection{Collision strengths }

Often cases are encountered in which a multiplet consisting of many lines
can be treated as the equivalent two-level atom with a single transition.
In these cases it is necessary to define ``effective'' collision strengths
and transition probabilities.
If the collision strength from an individual
level $i$ is ${\Upsilon _i}$, and the statistical weights of the level and term are $g_i$ and $g_{tot}$
respectively, then the effective collision strength ${\Upsilon _{eff}}$
is related to ${\Upsilon _i}$  by a simple argument.
The collision rate
$q_i$ is proportional to the ratio
\begin{equation}
{n_i}\,{q_i} \propto {n_i}\frac{{{\Upsilon _i}}}{{{g_i}}}\quad
  [\mathrm{s}^{-1}]
\end{equation}
so that
\begin{equation}
{n_{tot}}\,{q_{tot}} = \sum\limits_i {{n_i}\,{q_i}}  \propto \sum\limits_i
{{n_i}\frac{{{\Upsilon _i}}}{{{g_i}}}}\quad
  [\mathrm{s}^{-1}].
\end{equation}
In many cases it is valid to assume that the levels within the term are
populated according to their statistical weight, viz.,
\begin{equation}
{n_i} = {n_{tot}}\frac{{{g_i}}}{{{g_{tot}}}}
\quad  [\mathrm{cm}^{-3}].
\end{equation}
Then, the effective collision strength ${\Upsilon _{tot}}$
is operationally defined by the relations
\begin{equation}
{n_{tot}}\frac{{{\Upsilon _{tot}}}}{{{g_{tot}}}} = \sum\limits_i
{{n_i}\frac{{{\Upsilon _i}}}{{{g_i}}} = \,} {n_{tot}}\sum\limits_i
{\frac{{{g_i}}}{{{g_{tot}}}}\frac{{{\Upsilon _i}}}{{{g_i}}}}  =
{n_{tot}}\,\frac{{\sum\limits_i {{\Upsilon _i}} }}{{{g_{tot}}}}.
\end{equation}
So, the effective collision strength of the entire multiplet is
\begin{equation}
{\Upsilon _{tot}} = \sum\limits_i {{\Upsilon _i}}.
\end{equation}

\subsection{Transition probabilities }

Under similar circumstances an effective transition probability
$A_{eff}$ may be defined as
\begin{equation}
{n_{tot}}{\kern 1pt} {A_{tot}} = \sum\limits_i {{n_i}{A_i}}  =
{n_{tot}}\sum\limits_i {\frac{{{g_i}}}{{{g_{tot}}}}{A_i}}
\end{equation}
so that the effective transition probability is
\begin{equation}
{A_{tot}} = \sum\limits_i {\frac{{{g_i}}}{{{g_{tot}}}}{A_i}}.
\end{equation}
So collision strengths are added, and transition probabilities averaged.

\section{Level populations with collisions }

Both escape and destruction probabilities enter in the calculation of
a level population and line emissivity.
The escape probability $P_{u,l}$ is
the probability that a line photon will escape in a single scattering
(\citealp{Elitzur1983}; \citealp{Elitzur1984}).
The destruction probability $D_{u,l}$ is
the probability that a line photon will be destroyed in a single scattering.

The line de-excitation rate is given by
\begin{equation}
{\left( {\frac{{d{n_u}}}{{dt}}} \right)_{rad}} = {n_u}{A_{u,l}}\left(
{{P_{u,l}} + {D_{u,l}}} \right) - {n_l}{A_{u,l}}\eta {\gamma _{u,l}} +
{n_u}{C_{ul}} - {n_l}{C_{lu}}\quad
 [\mathrm{cm}^{-3} \mathrm{s}^{-1}]
\end{equation}
where $\eta$ is the photon occupation number of the attenuated external radiation
field, $C$ is the collision rate (s$^{-1}$), and $\gamma _{u,l}$
is the fluorescence probability.

The net emission from a transition between the level $n$ to a lower level
$l$ and escaping to the surface is then
\begin{equation}
4\pi \,j(n,l) = {n_n}{A_{n,l}}\;h{\nu _{n,l}}\;{P_{u,l}}({\tau
_{n,l}})\;f(r)\quad
 [\mathrm{erg~cm}^{-3} \mathrm{s}^{-1}]
\end{equation}
where $f(r)$ is the filling factor.
The total emission from the gas is then
\begin{equation}
e(n,l) = \int_V {{n_n}{A_{n,l}}\;h{\nu _{n,l}}\;{P_{u,l}}({\tau
_{n,l}})\;f(r)} \,dV\quad
 [\mathrm{erg~cm}^{-2} \mathrm{s}^{-1} \mathrm{or\, erg\, s}^{-1}]
\end{equation}
depending on whether the intensity or luminosity case is chosen.
The local
cooling rate (erg cm$^{-3}$~s$^{-1}$) due to the line is related
to the level populations by
\begin{equation}
{\Lambda _{u,l}} = \left( {{n_l}{C_{l,u}} - {n_u}{C_{u,l}}}
\right)\;f(r)\;h\nu \quad
 [\mathrm{erg~cm}^{-3} \mathrm{s}^{-1}]
\end{equation}
and the local flux (cm$^{-2}$ s$^{-1}$) of ``on-the-spot'' (OTS)
photons caused by
line loss (used to compute heating or photoionization rates
for the sources of the background opacity) is
\begin{equation}
{\varphi _{OTS}} = \frac{{{n_u}{A_{u,l}}{D_{u,l}}({X_c})}}{{\sum {{\kappa
_c}\;n(c)} }}.
\end{equation}

The ratio of inward to total line intensity is then given by
\begin{equation}
\frac{{4\pi \,j(in)}}{{4\pi \,j(total)}} = \frac{{{P_{u,l}}\left( \tau
\right)}}{{\left[ {{P_{u,l}}\left( \tau  \right) + {P_{u,l}}\left( {T -
\tau } \right)} \right]}}.
\end{equation}
